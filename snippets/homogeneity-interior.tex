\iflanguage{ngerman}{
  Sei die Ordnung $\ord$ definiert als
}{
  Let the order $\ord$ be defined as
}
\begin{align*}
  \ord \lambda^2 V(x) := {} & 0 \\
            \ord W(x) := {} & 2 \\
 \ord \partial_x A(x) := {} & \ord A(x) + 1,
\end{align*}
\iflanguage{ngerman}{
  wobei $A(x)$ ein rationaler Ausdruck in $V$, $W$ und deren Ableitungen sei,
  für eine genaue Definition siehe \cref{def:order} (letztlich gibt diese
  Ordnung das Skalierungsverhalten bezüglich $x$ an).
}{
  where $A(x)$ is a rational expression in $V$, $W$ and their derivatives, for a
  precise definition cf.\ \cref{def:order} (the order is a measure of the
  scaling properties with regard to $x$).
}

\iflanguage{ngerman}{
  Dann gilt die folgende Gleichung für die Terme $k_{-n}$ der
  Homogenitätsordnung $-n$ in der asymptotischen Entwicklung des Kerns von
  $(\Delta_\lambda + z^2)^{-1}$ auf der Diagonale:
}{
  Then the following holds for the terms $k_{-n}$ of homogeneity order $-n$ in
  the asymptotic expansion of the kernel of $(\Delta_\lambda +
  z^2)^{-1}$ on the diagonal:
}
\begin{equation*}
  \ord k_{-n}(x, x) = n - 1
\end{equation*} 
