\iflanguage{ngerman}{
  Sei $\phi(x)\in\Cinf[0](\Rplus)$ eine Abschneidefunktion, deren Träger in einer
  genügend kleinen Umgebung von $x=0$ liegt. Dann hat die multiparametrische
  Resolventenspur des Sturm-Liouville-Operators
}{
  Let $\phi(x)\in\Cinf[0](\Rplus)$ be a cutoff function that is $0$ outside of a
  sufficiently small neighbourhood of $x=0$. Then the multiparametric
  trace-expansion of the resolvent of the Sturm-Liouville operator
}
\begin{equation*}
  \Delta_\lambda = -\partial_x^2 + \lambda^2 V(x) + W(x)
\end{equation*}
\iflanguage{ngerman}{
  auf $\Rplus$ bezüglich der verallgemeinerten Neumannrandbedingung
  $f(0)\cos\theta + f'(0)\sin\theta = 0$ die folgende asymptotische Entwicklung
  bis zur dritten nicht-verschwindenden Ordnung nahe $x=0$
}{
  on $\mathbb{R_+}$ up to the third non-vanishing order near $x=0$ with the
  generalised Neumann boundary conditions $f(0)\cos\theta + f'(0)\sin\theta = 0$
  is given by
}
\begin{equation*}
  \Tr\left(\phi(x)(\Delta_\lambda + z^2)^{-1}\right) \SimMu
  \frac{1}{(2\mu)^2} + \frac{5\lambda^2}{(2\mu)^5} V'(0) + O(\mu^{-4}),
\end{equation*}
\iflanguage{ngerman}{mit}{with} $\mu^2 := \lambda^2 V(0) + W(0)$.
