\subsection{English Introduction}
% TODO: Die beiden MainTheoremIntros zur Introduction hinzufügen, Order definieren,
% auch in die deutsche Einführung, Beweise mehr ausführen
The theme of this diploma thesis is the asymptotic expansion of the trace of
multi-parametric Sturm-Liouville operators on the closed interval $[0,1]$.
%
Operators of this kind arise on investigation of the resolvent of the
Laplace-Beltrami operator $\Delta$ on a surface of revolution when splitting it
into components with regard to the one-dimensional sphere $\Sphere{1}$. The
surface of revolution is thereby expressed as an infinite direct sum $\Delta =
\bigoplus_{\lambda\in\operatorname{spec}\Delta_\Sphere{1}}\Delta_\lambda$, as an
operator over the direct sum of the eigenspaces
($\times[0,1]$).\ref{sec:laplace-beltrami}
%
The operators $\Delta_\lambda$ are Sturm-Lioville operators with one parameter
$\lambda$ and using $V,W\in\Cinf([0,1])$, $V>0$ and $W>0$ can be written as
\begin{align*}
  (\Delta_\lambda u)(x) = -u''(x) + \lambda^2 V(x) u(x) + W(x) u(x)
\end{align*}
In the given case of the Laplace-Beltrami operator on a surface of revolution
generated by a function $f\in\Cinf([0,1])$ with $f>0$ the functions $V$ and $W$
are given by
\begin{align*}
  V(x) &= \frac1{f(x)^2} \\
  W(x) &= \frac{f''(x)}{2f(x)} - \left(\frac{f'(x)}{2f(x)}\right)^2.
\end{align*}

It is proven in \cite{LV13} that the zeta determinant, (formally) defined for an
operator $A$ by
\begin{align*}
  \zeta(s,A) &:= \sum_{\mu\in\operatorname{spec}A\setminus\{0\}} m(\mu)\mu^{-s}
  \\
  \log\zdet A &= -\zeta'(0,A),
\end{align*}
of the direct sum $\bigoplus_\lambda\Delta_\lambda$ can be calculated as a
regularised sum of the zeta determinants of the $\Delta_\lambda$ with an
additional finite error term:
\begin{align*}
  \log\zdet\Delta = \Regsum_\lambda\zdet\Delta_\lambda + C
\end{align*}
The regularised sum $\Regsum$ is defined by the Hadamard finite part
regularisation of the Euler-MacLaurin formula.\ref{sec:regsum-int}

The goal is now to calculate the error term $C$, which is essentially given by
an integral over a term in the multiparametric asymptotic expansion of
$\Tr(\Delta_\lambda + z^2)^{-2}$ which can be derived from the expansion of
$\Tr(\Delta_\lambda + z^2)^{-1}$. The operator
\begin{align*}
  (A_{\lambda,z}u)(x) := (\Delta_\lambda u)(x) + z^2 u(x)
           = -u''(x) + (\lambda^2 V(x) + z^2) u(x) + W(x) u(x)
\end{align*}
is a \emph{multi-parametric Sturm-Liouville operator} with parameters $\lambda$
and $z$. For $h(\lambda,z)$ being the $-5^{\text{th}}$ term in the common
asymptotic expansion in $(\lambda,z)$ of the trace $\Tr A_{\lambda,z}^{-2}$ the
error term is given by:
\begin{align*}
  C = \Integ[\infty]{0}{\lambda}{h(1,\lambda)\log\lambda}
\end{align*}
To evaluate the integral we calculate the first terms of the asymptotic
expansion of $\Tr(\Delta_\lambda + z^2)^{-1}$ in the interior $(0,1)$ as well as
at the boundaries $0$ and $1$.

Using the theory of parametric pseudo-differential operators we can show the
following
\begin{MainTheoremIntro}
  % TODO: Existenz und explizite Formel (zumindest bis zum Grad 3).

\end{MainTheoremIntro}
Additionally we proof that the resulting homogeneous polynomials in $\lambda$
and $z$ also employ homogeneity with regard to $V$, $W$ and their respective
derivatives.

At the boundary we first split the operator $\Delta_\lambda + z^2$ into simpler
operators using a Neumann series argument. The contributions of those simpler
operators can be estimated such that we only have to investigate the asymptotics
of a finite number (in this particular case 3) of them to calculate the trace
asymptotics up to a given order.
%
To accomplish this we employ Watson's Lemma, which directly gives us the
asymptotics of an exponential integral ($F(x) =
\Integ[T]{0}{t}{\Eto{-tx}f(t)}$). This gives us
\begin{MainTheoremIntro}
  \iflanguage{ngerman}{
  Sei $\phi(x)\in\Cinf[0](\Rplus)$ eine Abschneidefunktion, deren Träger in einer
  genügend kleinen Umgebung von $x=0$ liegt. Dann hat die multiparametrische
  Resolventenspur des Sturm-Liouville-Operators
}{
  Let $\phi(x)\in\Cinf[0](\Rplus)$ be a cutoff function that is $0$ outside of a
  sufficiently small neighbourhood of $x=0$. Then the multiparametric
  trace-expansion of the resolvent of the Sturm-Liouville operator
}
\begin{equation*}
  \Delta_\lambda = -\partial_x^2 + \lambda^2 V(x) + W(x)
\end{equation*}
\iflanguage{ngerman}{
  auf $\Rplus$ bezüglich der verallgemeinerten Neumannrandbedingung
  $f(0)\cos\theta + f'(0)\sin\theta = 0$ die folgende asymptotische Entwicklung
  bis zur dritten nicht-verschwindenden Ordnung nahe $x=0$
}{
  on $\mathbb{R_+}$ up to the third non-vanishing order near $x=0$ with the
  generalised Neumann boundary conditions $f(0)\cos\theta + f'(0)\sin\theta = 0$
  is given by
}
\begin{equation*}
  \Tr\left(\phi(x)(\Delta_\lambda + z^2)^{-1}\right) \SimMu
  \frac{1}{(2\mu)^2} + \frac{5\lambda^2}{(2\mu)^5} V'(0) + O(\mu^{-4}),
\end{equation*}
\iflanguage{ngerman}{mit}{with} $\mu^2 := \lambda^2 V(0) + W(0)$.

\end{MainTheoremIntro}
This formula holds, with minor changes, also for $x=1$. The existence of the
asymptotic expansion is proven in \cite{LV13}.
%
Furthermore we prove in the final Main Theorem, that the asymptotics at the
boundary has the same homogeneity properties as the expansion in the interior,
namely that the terms are again homogeneous in $V$, $W$ and their respective
derivatives.

By applying the first two Main Theorems we can evaluate the error term in the
Lesch-Vertman formula as the integral over a rational function in the
derivatives of $f$ in the interior and at the boundary:
% TODO
