\subsection{English Introduction}
% TODO: Die beiden MainTheoremIntros zur Introduction hinzufügen, Order definieren,
% auch in die deutsche Einführung, Beweise mehr ausführen
The theme of this diploma thesis is the asymptotic expansion of the trace of
multi-parametric Sturm-Liouville operators on the closed interval $[0,1]$.
%
Operators of this kind arise on investigation of the resolvent of the
Laplace-Beltrami operator $\Delta$ on a surface of revolution when splitting it
into components with regard to the one-dimensional sphere $\Sphere{1}$. The
surface of revolution is thereby expressed as an infinite direct sum of
operators $\Delta =
\bigoplus_{\lambda\in\operatorname{spec}\Delta_\Sphere{1}}\Delta_\lambda$ on the
direct sum of the eigenspaces ($\times[0,1]$) (cf.\
\cref{sec:laplace-beltrami}).
%
The operators $\Delta_\lambda$ are Sturm-Lioville operators with one parameter
$\lambda$ and using $V,W\in\Cinf{[0,1]}$, $V>0$ and $W>0$ can be written as
\begin{equation*}
  (\Delta_\lambda u)(x) = -u''(x) + \lambda^2 V(x) u(x) + W(x) u(x).
\end{equation*}
In the given case of the Laplace-Beltrami operator on a surface of revolution
generated by a function $f\in\Cinf{[0,1]}$ with $f>0$ the functions $V$ and $W$
are given by
\begin{align*}
  V(x) &= \frac1{f(x)^2}, \\
  W(x) &= \frac{f''(x)}{2f(x)} - \left(\frac{f'(x)}{2f(x)}\right)^2.
\end{align*}
It is proven in \cite{LV13} that the zeta determinant of the given
Laplace-Beltrami operator on a surface of revolution $\Delta$, (formally)
defined by
\begin{align*}
  \zeta(s,\Delta) := {} &
  \sum_{\mathclap{\mu\in\operatorname{spec}\Delta\setminus\{0\}}}
  m(\mu)\mu^{-s} \text{ for } \Re(s) \\
  \log\zdet \Delta = {} & {-\zeta'}(0,\Delta) \text{ by meromorphic
continuation},
\end{align*}
can be calculated as a regularised sum of the zeta determinants of the
$\Delta_\lambda$ with an additional finite error term:
\begin{equation*}
  \log\zdet\Delta = \Regsum_{\mathclap{\lambda =
  -\infty}}^\infty\log\zdet\Delta_\lambda + C
\end{equation*}
The regularised sum $\Regsum$ is defined by the Hadamard finite part
regularisation of the Euler-MacLaurin formula (cf.\ \cref{sec:regsum-int}).

The goal of this thesis is now to calculate the error term $C$, which is
essentially given by an integral over a homogeneous component in the
multi-parametric asymptotic expansion of $\Tr(\Delta_\lambda + z^2)^{-2}$ which
can be derived from the expansion of $\Tr(\Delta_\lambda + z^2)^{-1}$. The
operator
\begin{align*}
  (A_{\lambda,z}u)(x) := (\Delta_\lambda u)(x) + z^2 u(x)
           = -u''(x) + (\lambda^2 V(x) + z^2) u(x) + W(x) u(x)
\end{align*}
is a \emph{multi-parametric Sturm-Liouville operator} with parameters $\lambda$
and $z$. For $h(\lambda,z)$ being the term of homogeneity order $-5$ in the
common asymptotic expansion in $(\lambda,z)$ of the trace $\Tr
A_{\lambda,z}^{-2}$, $h(1,\,\cdot\,)\log(\,\cdot\,)\in L^1(0,\infty)$ and the
error term is given by:
\begin{align*}
  C = \Integ[\infty]{0}{\lambda}{h(1,\lambda)\log\lambda}
\end{align*}
To evaluate the integral we calculate the first terms of the asymptotic
expansion of the resolvent kernel $k_{(\Delta_\lambda + z^2)^{-1}}$ in the
interior $(0,1)$ as well as at the boundaries $0$ and $1$.

Since we are investigating the parametrix (i.e.\ $(\Delta_\lambda + z^2)^{-1}$
instead of $(\Delta_\lambda + z^2)^{-2}$) we need to find the term of
homogeneity order $-3$ as
\begin{equation*}
  \Tr(\Delta_\lambda + z^2)^{-2} = (-2z)^{-1}\partial_z\Tr(\Delta_\lambda +
  z^2)^{-1},
\end{equation*}
so the first expansion is shifted by $2$ against the other.

Using the theory of parametric pseudo-differential operators we can show the
following
\begin{MainTheoremIntro}[Resolvent in the interior]
  % TODO: Existenz und explizite Formel (zumindest bis zum Grad 3).

\end{MainTheoremIntro}
Additionally we proof that the resulting homogeneous polynomials in $\lambda$
and $z$ also employ homogeneity with regard to $V$, $W$ and their respective
derivatives:
\begin{MainTheoremIntro}[Homogeneity in the interior]
  \iflanguage{ngerman}{
  Sei die Ordnung $\ord$ definiert als
}{
  Let the order $\ord$ be defined as
}
\begin{align*}
  \ord \lambda^2 V(x) := {} & 0 \\
            \ord W(x) := {} & 2 \\
 \ord \partial_x A(x) := {} & \ord A(x) + 1,
\end{align*}
\iflanguage{ngerman}{
  wobei $A(x)$ ein rationaler Ausdruck in $V$, $W$ und deren Ableitungen sei,
  für eine genaue Definition siehe \cref{def:order} (letztlich gibt diese
  Ordnung das Skalierungsverhalten bezüglich $x$ an).
}{
  where $A(x)$ is a rational expression in $V$, $W$ and their derivatives, for a
  precise definition cf.\ \cref{def:order} (the order is a measure of the
  scaling properties with regard to $x$).
}

\iflanguage{ngerman}{
  Dann gilt die folgende Gleichung für die Terme $k_{-n}$ der
  Homogenitätsordnung $-n$ in der asymptotischen Entwicklung des Kerns von
  $(\Delta_\lambda + z^2)^{-1}$ auf der Diagonale:
}{
  Then the following holds for the terms $k_{-n}$ of homogeneity order $-n$ in
  the asymptotic expansion of the kernel of $(\Delta_\lambda +
  z^2)^{-1}$ on the diagonal:
}
\begin{equation*}
  \ord k_{-n}(x, x) = n - 1
\end{equation*} 

\end{MainTheoremIntro}
At the boundary we first split the operator $(\Delta_\lambda + z^2)^{-1}$ into
simpler operators using a Neumann series argument. The contributions of those
simpler operators can be estimated such that we only have to investigate the
asymptotics of a finite number (in this particular case 3) of them to calculate
the trace asymptotics up to a given order.
%
To accomplish this we employ Watson's Lemma, which directly gives us the
asymptotics of an exponential integral ($F(x) =
\Integ[T]{0}{t}{\Eto{-tx}f(t)}$). We get
\begin{MainTheoremIntro}[Resolvent trace at the boundary]
  \iflanguage{ngerman}{
  Sei $\phi(x)\in\Cinf[0](\Rplus)$ eine Abschneidefunktion, deren Träger in einer
  genügend kleinen Umgebung von $x=0$ liegt. Dann hat die multiparametrische
  Resolventenspur des Sturm-Liouville-Operators
}{
  Let $\phi(x)\in\Cinf[0](\Rplus)$ be a cutoff function that is $0$ outside of a
  sufficiently small neighbourhood of $x=0$. Then the multiparametric
  trace-expansion of the resolvent of the Sturm-Liouville operator
}
\begin{equation*}
  \Delta_\lambda = -\partial_x^2 + \lambda^2 V(x) + W(x)
\end{equation*}
\iflanguage{ngerman}{
  auf $\Rplus$ bezüglich der verallgemeinerten Neumannrandbedingung
  $f(0)\cos\theta + f'(0)\sin\theta = 0$ die folgende asymptotische Entwicklung
  bis zur dritten nicht-verschwindenden Ordnung nahe $x=0$
}{
  on $\mathbb{R_+}$ up to the third non-vanishing order near $x=0$ with the
  generalised Neumann boundary conditions $f(0)\cos\theta + f'(0)\sin\theta = 0$
  is given by
}
\begin{equation*}
  \Tr\left(\phi(x)(\Delta_\lambda + z^2)^{-1}\right) \SimMu
  \frac{1}{(2\mu)^2} + \frac{5\lambda^2}{(2\mu)^5} V'(0) + O(\mu^{-4}),
\end{equation*}
\iflanguage{ngerman}{mit}{with} $\mu^2 := \lambda^2 V(0) + W(0)$.

\end{MainTheoremIntro}
This formula holds, with minor changes, also for $x=1$. The existence of the
asymptotic expansion is proven in \cite{LV13}.
%
Furthermore we prove in the final Main Theorem, that the asymptotics at the
boundary has the same homogeneity properties as the expansion in the interior,
namely that the terms are again homogeneous in $V$, $W$ and their respective
derivatives:
\begin{MainTheoremIntro}[Homogeneity at the boundary]
  \iflanguage{ngerman}{
  Mit der selben Definition wie in Hauptsatz~2 gilt für einen Term $A$ der
  asymptotischen Entwicklung der Resolventenspur von $\Delta_\lambda$ am Rand
  mit der Homogenitätsordnung $-n$ in $(\lambda,z)$:
}{
  Using the same definition for $\ord$ as in Main~Theorem~2 we have for a term
  $A$ in the asymptotic expansion of the resolvent trace of $\Delta_\lambda$ at
  the boundary of homogeneity $-n$ in $(\lambda,z)$:
}
\begin{equation*}
  \ord A = n - 2
\end{equation*}

\end{MainTheoremIntro}
By applying the first two Main Theorems we can evaluate the error term in the
Lesch-Vertman formula as the integral over a rational function in the
derivatives of $f$ in the interior and at the boundary:
\begin{align*}
  \Integ[\infty]{0}{\lambda}{
    h(\lambda, 1) \log\lambda
  }
  &=
  % Interior:
  \Int{x}{\Biggl(\frac{(1 + 2 \log(2 V(x))) W(x)}{8 V(x)}
    - \frac{ (1 - 2 \log(2 V(x)))V''(x)}{96 V(x)^{3}} \\
    &\hphantom{=\int_0^1\Biggl(}\mathbin{+} \frac{\left(\frac{5}{3} - 2\log(2
    V(x))\right)\left(V'(x)\right)^{2}}{64 V(x)^{5}}\Biggr) } \\
&\mathbin- 5\frac{(1 - 2\log(2 V(1)))V'(1)}{192 V(1)^3}
+ 5\frac{(1 - 2 \log(2 V(0)))V'(0)}{192 V(0)^3}
\end{align*}
