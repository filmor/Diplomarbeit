\subsection{Conventions}
We follow the conventions used in \cite{Les:PDO}, hence the Fourier transform is
defined as
\begin{align*}
  \hat{u}(\xi) = (\mathcal{F}u)(\xi)
    &:= \Integ{\mathbb{R}}{x}{\Eto{-\mathrm ix\xi} u(x)} \\
  (\mathcal{F}^{-1}\hat{u})(x) &:=
  \frac{1}{2\pi}\Integ{\mathbb{R}}{\xi}{\Eto{\mathrm ix\xi}
    \hat{u}(\xi)}.
\end{align*}
We will always talk about symbols and only mention amplitudes when explicitly
needed. We thus denote the \emph{symbol} of an operator $A$ by $a$ instead of
$\sigma_A$. The kernel of an operator $A$ is denoted by $k_A$, i.e.\ 
\begin{align*}
  (Au)(x) =: \Integ{\mathbb{R}}{y}{k_A(x,y) u(x)},
\end{align*}
where $k_A$ may be a distribution.
