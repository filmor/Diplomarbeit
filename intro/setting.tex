\section{Setting}
We will first motivate our investigation of the asymptotics of multi-parametric
Sturm-Liouville operators by a formula given in \cite{LV13}, which we outline
below. This formula can be used to represent the zeta determinant of an infinite
direct sum of Sturm-Liouville operators by means of a regularised sum of the
zeta determinants of the summed operators and an error term involving a
homogeneous component of the asymptotic expansion of the resolvent of the summed
up operators.

We will first introduce the toy example of a surface of revolution, which indeed
decomposes into an infinite sum of Sturm-Liouville operators, and then introduce
the further notation needed to understand the formula whose error term we aim
to calculate in the end.

\subsection{Laplace-Beltrami Operator on a Surface of Revolution}
\label{sec:laplace-beltrami}
Let $M = [0,1]\times\Sphere{1}$ be a surface of revolution by use of the metric
$g = \mathrm dx^2 \oplus f(x)^2 g_\Sphere{1}$ with $f\in \Cinf([0,1], \Rplus)$
and $g_\Sphere1$ being the standard metric on $\Sphere1$. The associated
Laplace-Beltrami operator is given by
\begin{align}
  \Delta = -\frac{\partial^2}{\partial x^2} -
            \frac{f'(x)}{f(x)}\frac{\partial}{\partial x} +
            \frac{1}{f(x)^2}\Delta_\Sphere1,
  \label{eqn:laplace}
\end{align}
acting on $\Cinf[0]\bigl((0,1)\times\Sphere1\bigr)$. The correct Lebesgue
measure on the space $(M,g)$ is given by $f(x)\,\mathrm dx\, \mathrm
d\!\vol(g_\Sphere1)$. Under the unitary map
\begin{align*}
  \Phi\colon L^2(M, g)&\to L^2([0,1], \mathrm dx)\otimes L^2(\Sphere1,
  g_\Sphere1) \\ (\Phi u)(x) &:= u(x) \sqrt{f(x)},
\end{align*}
the Laplacian $\Delta$ transforms into the operator
\begin{align*}
  \Phi\Delta\Phi^{-1} = -\frac{\partial^2}{\partial x^2} +
  \frac{1}{f(x)^2}\Delta_\Sphere1 + \left( \frac{f''(x)}{2f(x)} -
  \left(\frac{f'(x)}{2f(x)}\right)^2\right).
\end{align*}
For $\lambda\in\mathbb{Z}$ the functions $e_\lambda(x) :=
\frac{1}{\sqrt{2\pi}}\Eto{\mathrm i\lambda x}$ form an orthonormal basis of
$L^2(\Sphere1)$ of eigenfunctions of $\Delta_\Sphere1$ to the eigenvalues
$\lambda^2$, where the eigenvalues $\lambda^2 \neq 0$ have multiplicity 2 while
$\lambda = 0$ has multiplicity 1. Hence we have the decomposition
\begin{align}
  \label{eqn:lpl-decomp}
  \Phi\Delta\Phi^{-1} &= \bigoplus_{\lambda=-\infty}^{\infty} \left[
    -\frac{\partial^2}{\partial x^2} + \frac{\lambda^2}{f(x)^2} + 
     \left( \frac{f''(x)}{2f(x)} -
   \left(\frac{f'(x)}{2f(x)}\right)^2\right)\right]
     =: \bigoplus_{\lambda=0}^{\infty}\Delta_\lambda
\end{align}
into a direct sum of one-dimensional Sturm-Liouville operators $\Delta_\lambda$,
acting on $\Cinf((0,1),\langle e_\lambda\rangle)$, i.e.\ the space of
$\Cinf$-functions mapping the interval $(0,1)$ to the vector space spanned by
the given eigenfunctions $e_\lambda$ for $\lambda\in\mathbb{Z}$.

We consider separated Dirichlet or generalised Neumann boundary conditions for
$\Delta$, where it is straight-forward to see that those also correspond to
separated Dirichlet, resp.\ generalised Neumann boundary conditions for
$\Phi\Delta\Phi^{-1}$ and that the resulting self-adjoint operator is compatible
with the decomposition given in \cref{eqn:lpl-decomp}. Since $\Phi$ is unitary
the transformed operator is unitarily (and thus spectrally) equivalent to
$\Delta$. We will denote the transformed self-adjoint extensions of $\Delta$ and
$\Delta_\lambda$ again by the same name. Since we are considering fixed boundary
conditions this is not too ambiguous.

We will consider operators like $\Delta_\lambda$, where we will set
\begin{align}
  \label{eqn:v-from-f}
  V(x) :={}& f(x)^{-2} \text{ and } \\
  \label{eqn:w-from-f}
  W(x) :={}& \frac{f''(x)}{2f(x)} - \left(\frac{f'(x)}{2f(x)}\right)^2 =
    \partial_x^2 \log\sqrt{f(x)} + \left(\partial_x \log\sqrt{f(x)}\right)^2 \\
  \label{eqn:d_lambda}
  \text{thus}\quad \Delta_\lambda = {}& -\partial^2_x + \lambda^2 V(x) + W(x),
\end{align}
where the non-derivative terms as usual denote the multiplication operators
$(M_\phi u)\colon x \mapsto \phi(x)u(x)$. We won't write this out, for example
in the we use $W(x)$ instead of $M_W$ in the summands of $\Delta_\lambda$,
% TODO Hübscher begründen
although this abuses notation since $x$ is undefined in this context.

In our calculations we will not depend on the specific choice of $V$ and $W$,
but instead only require
\begin{equation*}
  \begin{split}
    &V,W\in\Cinf([0,1]) \\
    &\forall_{x\in[0,1]}\colon V(x) > 0 \\
    &V\text{ and }W\text{ analytic in neighbourhoods of $0$ and $1$}.
  \end{split}
\end{equation*}
If we use an analytic function $f>0$ in \cref{eqn:v-from-f,eqn:w-from-f} this is
given.

\subsection{Regularised Sums and Integrals}
\label{sec:regsum-int}
Before we introduce the Hadamard finite part regularisation that is used to
regularise sums and integrals we will provide some general definitions on
asymptotic analysis based on \cite[Ch.1]{Miller2006}.
\begin{Definition}
  The Landau symbols and asymptotic equality are defined as
  \begin{align*}
    g(x) &= O(f(x))\text{ for $x\to x_0$ }& &:\Leftrightarrow \limsup_{x\to x_0}
    \Abs{\frac{g(x)}{f(x)}} < \infty \\
    g(x) &= o(f(x))\text{ for $x\to x_0$ }& &:\Leftrightarrow \lim_{x\to
    x_0}\Abs{\frac{g(x)}{f(x)}} = 0 \\
    f &\SimAs{x\to x_0} g& &:\Leftrightarrow \lim_{x\to x_0}
    \Abs{\frac{f(x)}{g(x)}} = 1,
    \intertext{where $x_0$ may be $\infty$. As a special case we furthermore define for
    $x\to\infty$}
    f(x) &= O(x^{-\infty})& &:\Leftrightarrow \forall_{n\in\mathbb{N}}\ f(x) =
    O(x^{-n}).
  \end{align*}
\end{Definition}
As done in many other texts we will also use the Landau symbols in actual
calculations, however it is of note that this is only valid in finite sums, for
all other calculations (in particular integrations) we have to be more careful.
\begin{Definition}
  Let $(\phi_n)$ be a sequence of functions with
  \begin{align*}
    \phi_{n+1}(x) = o(\phi_n(x))\text{ for $x\to x_0$},
  \end{align*}
  which is called an \emph{asymptotic sequence}. Then, for a sequence $(a_n)$
  the sum $\sum_{n=0}^N a_n\phi_n(x)$ is an \emph{asymptotic approximation to
  the function $f$} as $x\to x_0$ if
  \begin{equation}
    \label{eqn:def-asymp-exp}
    f(x) - \sum_{n=0}^N a_n\phi_n(x) = o(\phi_N(x)).
  \end{equation}
  If the above is true for all $N\in\mathbb{N}$, the formal infinite sum is
  called an \emph{asymptotic series} and is said to be an \emph{asymptotic
  expansion} of $f$ for $x\to x_0$, which we will denote by
  \begin{equation*}
    f(x) \SimAs{x\to x_0} \sum_{n=0}^\infty a_n\phi_n(x).
  \end{equation*}
  Let $N_0\in\mathbb{N}$. In case \cref{eqn:def-asymp-exp} at least holds for
  all $N < N_0$, we say $f$ has a \emph{partial asymptotic expansion}.
\end{Definition}
Now let $f\in\Cinf(\Rplus)$ be a function with a partial asymptotic expansion
for $x\to\infty$ with regard to the log-polynomial asymptotic series
$x^{-n}\log^k(x)$, were $\log^k(x) := (\log(x))^k$. We denote the components by
$a_{nk}$, thus the expansion is written as
\begin{equation*}
  f(x) \SimAs{x\to\infty} \sum_{n=0}^{N-1}\sum_{k=0}^{M_n}
  a_{nk}\,x^{-n}\log^k(x) + x^{-N}\log^{M_N}f_N(x),
\end{equation*}
where $N\in\mathbb{N}$ is arbitrary and the remainder is $f_N(x) = o(1)$ as
$x\to\infty$. We define the \emph{regularised limit} for $x\to\infty$ as
\begin{equation}
  \Reglim_{x\to\infty} f(x) := a_{00},
\end{equation}
i.e.\ as the constant (or \emph{finite}) part of the expansion.

If for some $N\in\mathbb{N}$ the remainder $(x\mapsto x^{-N}f_N(x))\in
L^1[1,\infty)$, the integral $\Integ[R]{1}{x}{f(x)}$ does also admit an
asymptotic expansion of the form given above and we define its \emph{regularised
integral} as
\begin{equation}
  \Regint{x}{f(x)} := \Reglim_{R\to\infty}\Integ[R]{1}{x}{f(x)}.
\end{equation}
The integral is also the reason why we consider $x^n \log^m(x)$ as
base-functions in the asymptotic expansions, since that way the integral over
the expansion results in a meaningful expansion with the same base-functions.
This approach of regularising potentially infinite integrals is called
\emph{Hadamard finite part} (or French: \textit{partie finie}) regularisation.
In the special case that $f\in L^1[1,\infty)$ the regularised integral coincides
with the usual Lebesgue integral of $f$ over $[1,\infty)$.

Using this we can also give a meaningful definition of a \emph{regularised sum}
over $f$. Given the Bernoulli numbers $B_j$, the Bernoulli polynomials
$B_j(x)$ and $M\in\mathbb{N}$ consider the Euler-MacLaurin summation formula:
\begin{equation}
  \label{eqn:euler-maclaurin}
  \begin{split}
    \sum_{\lambda=1}^N f(\lambda) = {} & \Integ[N]{1}{x}{f(x)}
    + \sum_{k=1}^M \frac{B_{2k}}{(2k)!}\left(f^{(2k-1)}(N) - f^{(2k-1)}(1)\right)
    \\
    &+ \frac1{(2M + 1)!}\Integ[N]{1}{x}{B_{2m+1}(x-[x])f^{(2M+1)}(x)}
    + \frac12(f(1)+f(N))
  \end{split}
\end{equation}
By essentially replacing all of the integrals in this formula by regularised
integrals we arrive at the following definition for the regularised sum:
\begin{align*}
  \Regsum_{\lambda=1}^\infty f(\lambda) := {}& \Reglim_{N\to\infty}
  \sum_{\lambda=1}^N f(\lambda) \\
  = {} & \Regint{x}{f(x)} + \sum_{k=1}^M \frac{B_{2k}}{(2k)!}\left(
  \Reglim_{N\to\infty} f^{(2k-1)}(N) - f^{(2k-1)}(1)\right) \\
  &+ \frac1{(2M + 1)!}\Regint{x}{B_{2m+1}(x - [x])f^{(2M+1)}(x)}
   + \frac12(f(1) + \Reglim_{N\to\infty} f(N))
\end{align*}
The regularised sum over all of $\mathbb{Z}$ is defined by
\begin{equation*}
  \Regsum_{\lambda\in\mathbb{Z}} f(\lambda) := f(0) + \Regsum_{\lambda=1}^\infty
  f(\lambda) + \Regsum_{\lambda=1}^\infty f(-\lambda).
\end{equation*}

The following theorem proven in \cite{LV13} is the basis for the following
considerations:
\begin{Theorem}
  \label{thm:fubini}
  Assume $f\in\Cinf(\Rplus[2])$ is of the form
  \begin{equation*}
    f(x,y) = \sum_{j=0}^{N-1} f_{\alpha_j}(x,y) + F_N(x,y)
  \end{equation*}
  where $N\in\mathbb{N}$ and each $f_{\alpha_j}\in\Cinf(\mathbb{R}_+^2\setminus\{0\})$
  is homogeneous of order $\alpha_j\in\mathbb{C}$ in both variables jointly,
  where the remainder $F_N$ is assumed to be in $L^1[1,\infty)^2$. Then
  \begin{equation*}
    \Regint{x}{\Regint{y}{f(x,y)}} =
    \Regint{y}{\Regint{x}{f(x,y)}} + \Integ[\infty]{0}{y}{f_{-2}(1,y)\log(y)}.
  \end{equation*}
\end{Theorem}
Using this Fubini-like theorem we can (possibly) simplify difficult regularised
integrals, however unlike Fubini's theorem in Lebesgue integral theory we get an
additional error term as an integral over one homogeneous component in this
case.

\subsection{Zeta Determinants}
The zeta function of an operator $\Delta_\lambda$ is defined for $\Re(s) > 0$
sufficiently large by
\begin{equation}
  \zeta(s,\Delta_\lambda) :=
  \sum_{\mathclap{\mu\in\operatorname{spec}\Delta_\lambda\setminus\{0\}}}
  m(\mu)\mu^{-s},
\end{equation}
where $m(\mu)$ is the multiplicity of the eigenvalue $\mu > 0$. Using the
identity % TODO: Ref dafür, in Les98 steht das zwar, aber nicht bewiesen :(
\begin{equation}
  \label{frm:zeta-frm}
  \zeta(s,\Delta_s) = 2 \frac{\sin\pi
  s}{\pi}\,\Regint[0]{z}{z^{1-2s}\Tr(\Delta_\lambda + z^2)^{-1}}
\end{equation}
one can derive a formula for $\log\zdet\Delta_\lambda =
-\zeta'(0,\Delta_\lambda)$ (which is defined by the meromorphic extension to
$\mathbb{C}$ of $\zeta$) for which $s=0$ is a regular point:
\begin{align}
  \log\zdet\Delta_\lambda = -2\,\Regint[0]{z}{z\Tr(\Delta_\lambda + z^2)^{-1}}
\end{align}
However since the resolvent $(\Delta + z^2)^{-1}$ is not trace class,
$\zdet\Delta$ can not be defined in the exact same way. Integrating
\cref{frm:zeta-frm} by parts we get
\begin{equation*}
  \zeta(s,\Delta_\lambda) = 2\frac{\sin\pi
  s}{\pi(1-s)}\,\Regint[0]{z}{z^{3-2s}\Tr(\Delta_\lambda + z^2)^{-2}},
\end{equation*}
and thus
\begin{equation*}
  \log\zdet\Delta_\lambda = -2\, \Regint[0]{z}{z^3\Tr(\Delta_\lambda + z^2)^{-2}},
\end{equation*}
which also works for $\Delta$, since we have an asymptotic expansion for
$\Tr(\Delta_\lambda + z^2)^{-2}$.

\subsection{Objective}
Lesch and Vertman proved in \cite{LV13} the following asymptotic expansion for
the squared resolvent trace of the operator given in \cref{eqn:laplace}:
\begin{align}
  \Tr(\Delta + z^2)^{-2} = \sum_{\mathclap{\lambda=-\infty}}^\infty
  \Tr(\Delta_\lambda + z^2)^{-2} \SimAs{z\to\infty} \sum_{k=2}^{\infty}
  a_kz^{-k}.
\end{align}
\begin{Remark}
  Note that in general the asymptotics of $\Tr(\Delta_\lambda + z^2)^{-2}$ in
  $(\lambda,z)$ is very different from the asymptotics of $\Tr(\Delta +
  z^2)^{-2}$, in particular $\Tr(\Delta_\lambda+z^2)^{-2} = O(z^{-3})$ while
  $\Tr(\Delta + z^2)^{-2} = O(z^{-2})$ for $z\to\infty$.
\end{Remark}
Using this asymptotic expansion and \thref{thm:fubini} they showed, that the
zeta determinant of $\Delta$ is given by a regularised sum of the
zeta determinants of the $\Delta_\lambda$ with an additional error term:
\begin{align}
  \label{eqn:central-thm}
  \log\zdet\Delta = \Regsum_{\mathclap{\lambda=-\infty}}^\infty
  \log\zdet\Delta_\lambda -
  4\Integ[\infty]{0}{\lambda}{h_2(\lambda, 1)\log(\lambda)},
\end{align}
where $h_2(\lambda, z)$ denotes the homogeneous term of degree $-5$ (i.e.\ the
second non-vanishing term) in the polyhomogeneous asymptotic expansion of
$\Tr(\Delta_\lambda + z^2)^{-2}$ as $\Abs{(\lambda, z)}\to\infty$. The ultimate
goal of this thesis is to calculate the error term by calculating the first few
terms in the polyhomogeneous trace-expansion of the resolvent of a
Sturm-Liouville operator, since by formally deriving $\Tr(\Delta_\lambda +
z^2)^{-1}$ we see that
\begin{align}
  \label{frm:shift}
  \partial_z \Tr{(\Delta_\lambda + z^2)}^{-1} &=
        (-2z) \Tr{(\Delta_\lambda + z^2)}^{-2} \\
        \Leftrightarrow \Tr(\Delta_\lambda + z^2)^{-2} &=
        (-2z)^{-1}\partial_z\Tr(\Delta_\lambda + z^2)^{-1},
\end{align}
thus the order of the homogeneous components is shifted by 2 from the resolvent
to the squared resolvent. This means, that to get the component of order $-5$ we
will need to find the homogeneous component in the trace-expansion of the
parametrix of order $-3$.
