\section{Setting}
We will first motivate our investigation of the asymptotics of multi-parametric
Sturm-Liouville operators by a formula given in \cite{LV12}. This formula can be
used to calculate the zeta determinant of an infinite direct sum of
Sturm-Liouville operators by means of a regularised sum of the zeta determinants
of the operators and an error term involving a specific (poly-)homogeneous
component of the asymptotic expansion of the direct sum operator.

We will first introduce the toy example of a surface of revolution, which indeed
decomposes into an infinite sum of Sturm-Liouville operators, and then introduce
the further notation needed to understand the formula, whose error term we aim
to calculate in the end.

\subsection{Laplace-Beltrami operator on a surface of revolution}
Let $M = [0,1]\times\Sphere{1}$ be a surface of revolution by use of the metric
$g = \mathrm dx^2 \oplus f(x)^2 g_\Sphere{1}$ with $f\in \Cinf([0,1],
\Rplus)$. The associated Laplace-Beltrami operator is given by
\begin{align}
  \Delta = -\frac{\partial^2}{\partial x^2} -
            \frac{f'(x)}{f(x)}\frac{\partial}{\partial x} +
            \frac{1}{f(x)^2}\Delta_\Sphere1,
\end{align}
acting on $\Cinf[0]\bigl((0,1)\times\Sphere1\bigr)$. The correct Lebesgue
measure on the space $(M,g)$ is given by $f(x)\,\mathrm dx\, \mathrm
d\!\vol(g_\Sphere1)$. Under the unitary map
\begin{align*}
  \Phi: L^2(M, g)&\to L^2([0,1], \mathrm dx)\otimes L^2(\Sphere1, g_\Sphere1) \\
     (\Phi u)(x) &:= u(x) \sqrt{f(x)},
\end{align*}
the Laplacian $\Delta$ transforms into the operator
% TODO: Proof
\begin{align*}
  \Phi\Delta\Phi^{-1} = -\frac{\partial}{\partial x^2} +
  \frac{1}{f(x)^2}\Delta_\Sphere1 + \left( \frac{f''(x)}{2f(x)} -
  \left(\frac{f'(x)}{2f(x)}\right)^2\right).
\end{align*}
The functions $f_\lambda(x) := \frac{1}{\sqrt{2\pi}}\Eto{\mathrm i\lambda x}$
form for $\lambda\in\mathbb{Z}$ an orthonormal basis of $L^2(\Sphere1)$ of
eigenfunctions of $\Delta_\Sphere1$ to the eigenvalues $\lambda^2$, where the
eigenvalues $\lambda^2 \neq 0$ have multiplicity 2 while $\lambda = 0$ has
multiplicity 1. Hence we have the decomposition
% TODO: Proof
\begin{align}
  \label{eqn:lpl-decomp}
  \Phi\Delta\Phi^{-1} &= \bigoplus_{\lambda=-\infty}^{\infty} \left(
    -\frac{\partial^2}{\partial x^2} + \frac{\lambda^2}{f(x)^2} + 
     \left( \frac{f''(x)}{2f(x)} -
     \left(\frac{f'(x)}{2f(x)}\right)^2\right)\right)
     =: \bigoplus_{\lambda=0}^{\infty}\Delta_\lambda
\end{align}
into a direct sum of one-dimensional Sturm-Liouville operators $\Delta_\lambda$.
We consider separated Dirichlet or generalised Neumann boundary conditions for
$\Delta$, where it is straight-forward to see, that those also correspond to
separated Dirichlet, resp.\ generalised Neumann boundary conditions for
$\Phi\Delta\Phi^{-1}$ and that the resulting self-adjoint operator is compatible
with the given decomposition \eqref{eqn:lpl-decomp}. Since the transformed
operator does indeed behave, from an analytical point of view, the same as the
operator $\Delta$ we will denote the self-adjoint extensions of $\Delta$ and
$\Delta_\lambda$ again by the same name. Since we are considering fixed boundary
conditions this is not too ambiguous.

We will consider operators like $\Delta_\lambda$, where we will set
\begin{align}
  V(x) &:= f(x)^{-2} \\
  W(x) &:= \frac{f''(x)}{2f(x)} - \left(\frac{f'(x)}{2f(x)}\right)^2 =
  \partial_x^2 \ln\sqrt{f(x)} + \left(\partial_x \ln\sqrt{f(x)}\right)^2 \\
  \text{thus}\quad \Delta_\lambda &= -\partial_x + \lambda^2 V(x) + W(x),
\end{align}
where the non-derivative terms as usual denote the multiplication operators
$(M_\phi u)\colon x \mapsto \phi(x)u(x)$. We won't write this out (i.~e.\ in the
summands of $\Delta_\lambda$ we use $W(x)$ instead of $M_W$) unless necessary.

In further calculations the "`multiplication part"' of the operator is of
relevance multiple times, so we will shorten it a bit:
\begin{align}
  \label{def:f-lambda}
  f_{\lambda}(x) := \lambda^2 V(x) + W(x)
\end{align}
Note, that this definition is $\lambda(V,W)(x)$ in \cite{LV13}.

\subsection{Regularised sums and integrals}
% TODO Define asymptotic expansion, verschiedene Basis-Funktionen

% TODO log-homogeneous
Now let $f\in\Cinf(\Rplus)$ be a function with a partial asymptotic expansion at
$x\to\infty$ with regard to the log-polynomial base functions $x^{-N}
\log^\alpha(x)$, were $\log^\alpha(x) := \log^{\alpha-1}(\log(x)), \log^1(x) :=
\log(x)$. We denote the components by $a_{n\alpha}$, thus the expansion is
written as
\begin{align*}
  f(x) \SimAs{x\to\infty} \sum_{j=0}^{N-1} a_{j}\,x^{-j} + x^{-N}f_N(x),
\end{align*}
where $N\in\mathbb{N}$ is arbitrary and the remainder is $f_N(x) = o(1)$ as
$x\to\infty$. We define the \emph{regularised limit} for $x\to\infty$ as
\begin{align}
  \Reglim_{x\to\infty} f(x) := a_{00},
\end{align}
i.~e.\ as the constant part of the expansion.

If for some $N\in\mathbb{N}$ the remainder $(x\mapsto x^{-N}f_N(x))\in
L^1[1,\infty)$, the integral $\Integ[R]{1}{x}{f(x)}$ does also admit an
asymptotic expansion of the form given above and we define its \emph{regularised
integral} as
\begin{align}
  \Regint{x}{f(x)} := \Reglim_{R\to\infty}\Integ[R]{1}{x}{f(x)}.
\end{align}
We also see here the necessity for using $x^n \log^m(x)$ as base-functions in
our asymptotic expansion, as only then the integral over the asymptotic
expansion results in a meaningful expansion for which we can then use the given
limit-definition. This approach of regularising potentially infinite integrals
is called Hadamard finite part (or French: partie finie) regularisation. In the
special case that $f\in L^1[1,\infty)$ the regularised integral coincides with
the usual Lebesgue integral of $f$.

Using this we can also give a meaningful definition of a \emph{regularised sum}
over $f$. Given the Bernoulli numbers $B_j$ and the Bernoulli polynomials
$B_j(x)$ consider the Euler-MacLaurin summation formula:
\begin{align*}
  \sum_{\lambda=1}^N f(\lambda) &= \Integ[N]{1}{x}{f(x)}
  + \sum_{k=1}^M \frac{B_{2k}}{(2k)!}\left(f^{(2k-1)}(N) - f^{(2k-1)}(1)\right)
  \\
  &\mathbin{+} \frac1{(2M + 1)!}\Integ[N]{1}{x}{B_{2m+1}(x-[x])f^{(2M+1)}(x)}
  \\
  &\mathbin{+} \frac12(f(1)+f(N))
\end{align*}
By essentially replacing the integrals by regularised integrals we arrive at the
following definition for the regularised sum:
\begin{align*}
  \Regsum_{\lambda=1}^\infty f(\lambda) := \Reglim_{N\to\infty} \sum_{\lambda=1}^N
  f(\lambda)
\end{align*}
The regularised sum over $\mathbb{Z}$ is defined by
\begin{align*}
  \Regsum_{\lambda\in\mathbb{Z}} f(\lambda) := f(0) + \Regsum_{\lambda=1}^\infty
  f(\lambda) + \Regsum_{\lambda=1}^\infty f(-\lambda)
\end{align*}

The following theorem proven in \cite{LV12} is the basis for the following
considerations:
\begin{Theorem}
  \label{thm:fubini}
  Assume $f\in\Cinf(\Rplus^2)$ is of the form
  \begin{align*}
    f(x,y) = \sum_{j=0}^{N-1} f_{\alpha_j}(x,y) + F_N(x,y)
  \end{align*}
  where $N\in\mathbb{N}$ and each $f_{\alpha_j}\in\Cinf(\mathbb{R}_+^2\setminus\{0\})$
  is homogeneous of order $\alpha_j\in\mathbb{C}$ in both variables jointly,
  where the remainder $F_N$ is assumed to be in $L^1[1,\infty)^2$. Then
  \begin{align}
    \Regint{x}{\Regint{y}{f(x,y)}} =
    \Regint{y}{\Regint{x}{f(x,y)}} + \Integ[\infty]{0}{y}{f_{-2}(1,y)\ln(y)}.
  \end{align}
\end{Theorem}
Using this Fubini-like theorem we can (possibly) simplify difficult regularised
integral, however unlike for Fubini's theorem in usual integral theory we get an
additional error term as an integral over one homogeneous component. 

\subsection{Zeta determinants}
% TODO
% - Definition
% - Im Kontext hier
% Ordnung oder so
% Heuristik für die Formel
    
\subsection{Goal}
Lesch and Vertman proved in \cite{LV13} the following asymptotic expansion for
the resolvent-trace of the operator given in \eqref{eqn:laplace}:
\begin{align}
  \Tr(\Delta + z^2)^{-2} = \sum_{\lambda=-\infty}^\infty \Tr(\Delta_\lambda +
  z^2)^{-2} \SimAs{z\to\infty} \sum_{k=2}^{\infty} a_kz^{-k}.
\end{align}
Using this asymptotic expansion and \thref{thm:fubini} they showed, that the
zeta-determinant of $\Delta$ is given by a regularised sum of the
zeta-determinants of the $\Delta_\lambda$ with an error term:
\begin{align}
  \Zetadet \Delta = \Regsum_{\lambda=-\infty}^\infty
  \ln\operatorname{det}_\zeta\Delta_\lambda -
  4\Integ[\infty]{0}{\lambda}{h_2(\lambda, 1)\ln(\lambda)},
\end{align}
where $h_2(\lambda, z)$ denotes the homogeneous term of degree $-5$ (i.e.\ the
second non-vanishing term) in the polyhomogeneous asymptotic expansion of
$\Tr(\Delta_\lambda + z^2)^{-2}$ as $\Abs{(\lambda, z)}\to\infty$. The ultimate
goal of this thesis is to calculate the error term by calculating the first few
terms in the polyhomogeneous trace-expansion of the resolvent of a
Sturm-Liouville operator.

By formally deriving $\Tr(\Delta_\lambda + z^2)^{-1}$ we see that
\begin{align*}
  \partial_z \Tr{(\Delta_\lambda + z^2)}^{-1} &=
        (-2z) \Tr{(\Delta_\lambda + z^2)}^{-2}
        \Leftrightarrow& \Tr(\Delta_\lambda + z^2)^{-2} =
        (-2z)^{-1}\partial_z\Tr(\Delta_\lambda + z^2)^{-1},
\end{align*}
thus the order of the homogeneous components is shifted by 2. This means, that
to get the component of order $-5$ we will need to find the homogeneous
component in the trace-expansion of the parametrix of order $-3$.
