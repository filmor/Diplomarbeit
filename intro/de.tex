\section{Einführung}
Diese Arbeit beschäftigt sich mit der asymptotischen Entwicklung der Spur
multiparametrischer Sturm-Liouville-Operatoren auf dem abgeschlossenen Intervall
$[0,1]$. Operatoren dieser Art erhält man beispielsweise bei der Untersuchung
des Laplace-Beltrami-Operators auf einer Rotationsfläche wenn dieser bezüglich
$\mathbb{S}^1$ in seine Komponenten zerlegt wird.

% \dots

Bei dieser Methode erhält man, wie in LV gezeigt, einen Fehlerterm in Form eines
Integrals über eine bestimmte Komponentenfunktion der asymptotischen Entwicklung
von $\Tr(\Delta_\lambda + z^2)^{-2}$ für $\left|(\lambda, z)\right|\to\infty$.
Diesen genau zu berechnen ist Ziel dieser Arbeit. Dazu werden im Folgenden
Formeln sowohl für die innere Asymptotik, als auch für die Asymptotik gegen die
Randwerte 0 und 1 hergeleitet. Für ersteres verwenden wir dabei den
multiparametrischen Kalkül der Pseudodifferentialoperatoren, für letzteres eine
neue Herangehensweise basierend auf der sehr einfachen Asymptotik exponentieller
Integrale, also Integrale der Form $F(x) = \Integ[T]{0}{t}{\Eto{-tx}f(t)}$.

Zunächst beweisen wir folgenden Hauptsatz:
% MThm 1: Asymptotik im Inneren
Die Existenz der Entwicklung ist durch einen Satz der Theorie parametrischer
Pseudodifferentialoperatoren gesichert, der ebenfalls bewiesen wird.

Dann zeigen wir mithilfe Watsons Lemmas, dass die erwähnte Asymptotik
exponentieller Integrale behandelt, dass für den Rand folgendes gilt:
\begin{Hauptsatz}
  Seien $V\in C^2([0,1])$ und $W\in C^0([0,1])$ und
  \begin{equation*}
    (\Delta_\lambda u)(x) := -u''(x) + (\lambda^2 V(x) + W(x)) u(x).
  \end{equation*}

  Die Resolvente sei gegeben durch den Operator $(\Delta_\lambda + z^2)^{-1}$.
  Dann hat die Spur der Resolvente die folgende gemeinsame Asymptotik in
  $\lambda$ und $z$:
  \begin{align*}
    \Tr(\Delta_\lambda + z^2) \SimAs{\Abs{(\lambda, z)}\to\infty}
    &\frac{1}{4} \frac1{\lambda^2 V(0) + z^2} \\
    {}+{}&\frac{1}{32} \frac{\lambda^2 V'(0)}{(\lambda^2 V(0) + z^2)^5} \\
    {}+{}&\frac{1}{16} \frac{W(0)}{(\lambda^2 V(0) + z^2)^4}
    + \frac{1}{64} \frac{\lambda^2 (V''(0) + 6V'(0))}{(\lambda^2 V(0) + z^2)^6}
    + \frac{1}{512} \frac{\lambda^4 V'(0)^2}{(\lambda^2 V(0) + z^2)^8}
  \end{align*}
\end{Hauptsatz}
% MThm 2: Asymptotik am Rand
Hier ist die Existenz der asymptotischen Entwicklung per Konstruktion gesichert.
Wir starten mit expliziten Formeln für den Kern des Operators und zeigen, dass
die jeweiligen Spurintegrale nach Watsons Lemma eine asymptotische Entwicklung
besitzen.

Unter Anwendung beider Hauptsätze erhalten wir somit den Fehlerterm in der
Lesch-Vertman-Formel als Integral über eine rationale Funktion in den
Ableitungen von $f$ im Inneren und am Rand:
% Cor:    Explizite Formel für den Fehlerterm in f^(n)

