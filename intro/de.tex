\subsection{Deutsche Einleitung}
% TODO: Special functions erwähnen
Diese Diplomarbeit beschäftigt sich mit der asymptotischen Entwicklung der Spur
multiparametrischer Sturm-Liouville-Operatoren auf dem abgeschlossenen Intervall
$[0,1]$.
%
Operatoren dieser Art erhält man beispielsweise bei der Untersuchung der
Resolvente des Laplace-Beltrami-Operators $\Delta$ auf einer Rotationsfläche
wenn dieser bezüglich der eindimensionalen Sphäre $\mathbb{S}^1$ in seine
Komponenten zerlegt wird, also als eine unendliche direkte Summe $\Delta =
\bigoplus_{\lambda\in\operatorname{spec}\Delta_\Sphere{1}}\Delta_\lambda$ als Operator der
direkten Summe der Eigenräume ($\times[0,1]$) geschrieben
wird.\ref{sec:laplace-beltrami}
%
Die enstehenden Operatoren sind Sturm-Liouville-Operatoren mit einem Parameter
$\lambda$ und lassen sich also für Funktionen $V,W\in\Cinf([0,1])$ mit $V > 0$
und $W > 0$ schreiben als
\begin{align*}
  (\Delta_\lambda u)(x) = -u''(x) + \lambda^2 V(x) u(x) + W(x) u(x).
\end{align*}
Für den Fall des Laplace-Beltrami-Operators auf einer von einer Funktion
$f\in\Cinf([0,1])$ mit $f>0$ erzeugten Rotationsfläche sind $V$ und $W$ gegeben
als
\begin{align*}
  V(x) &= \frac1{f(x)^2} \\
  W(x) &= \frac{f''(x)}{2f(x)} - \left(\frac{f'(x)}{2f(x)}\right)^2.
\end{align*}

In \cite{LV13} wird gezeigt, dass sich die Zetadeterminante, (formal) definiert
als
\begin{align*}
  \zeta(s,A) &:= \sum_{\mu\in\operatorname{spec}A\setminus\{0\}} m(\mu)\mu^{-s}
  \\
  \log\zdet A &= -\zeta'(0,A)
\end{align*}
für einen Operator $A$, der Summe in gewisser Weise als (regularisierte) Summe
der Zetadeterminanten der Summanden zuzüglich eines Fehlerterms schreiben lässt:
\begin{align*}
  \label{eqn:intro-de-lvformel}
  \log\zdet\Delta = \Regsum\zdet\Delta_\lambda + C
\end{align*}
In diesem Kontext ist die regularisierte Summe $\Regsum$ über die
Hadamard-Regularisierung der Euler-MacLaurin-Formel
definiert.\ref{sec:regsum-int}

Das Ziel ist es, den in \eqref{eqn:intro-de-lvformel} angegebenen Fehlerterm zu
berechnen. In \cite{LV13} wird gezeigt, dass obiger Fehlerterm $C$ sich aus der
asymptotischen Entwicklung von $\Tr(\Delta_\lambda + z^2)^{-2}$ berechnen lässt,
welche wiederum leicht auf die asymptotische Entwicklung von $\Tr(\Delta_\lambda
+ z^2)^{-1}$ zurückzuführen ist. Der Operator
\begin{align*}
  (A_{\lambda,z}u)(x) := (\Delta_\lambda u)(x) + z^2 u(x)
           = -u''(x) + (\lambda^2 V(x) + z^2) u(x) + W(x) u(x)
\end{align*}
ist nun ein \emph{multiparametrischer Sturm-Liouville-Operator}. Ist $h(\lambda,
z)$ der Term $-5$ter Ordnung in der gemeinsamen asymptotischen Entwicklung in
$(\lambda,z)$ der Spur $\Tr A_{\lambda,z}^{-2}$, so gilt für den Fehlerterm:
\begin{align*}
  C = \Integ[\infty]{0}{\lambda}{h(1,\lambda)\log\lambda}
\end{align*}
Um diesen berechnen zu können wird die asymptotische Entwicklung der
Resolventenspur $\Tr(\Delta_\lambda + z^2)^{-1}$ sowohl im Inneren, also auf dem
Intervall $(0,1)$, als auch an den Rändern $0$ und $1$ untersucht.

Im Inneren kann die Entwicklung mittels der Theorie parametrischer
Pseudodifferentialoperatoren, die auch die Existenz sichert, explizit rekursiv
bestimmt werden, wir erhalten daraus den 
\begin{Hauptsatz}
  % TODO: Existenz und explizite Formel (zumindest bis zum Grad 3).

\end{Hauptsatz}
Neben der expliziten Bestimmung der Koeffizienten können wir noch zeigen, dass
die Terme der Entwicklung nicht nur eine Homogenität in $(\lambda,z)$, sondern
auch in $V$, $W$ und deren Ableitungen aufweisen.

Am Rand wird der Operator zunächst mit einem Neumannreihenargument in
einfachere Operatoren zerlegt. Die Asymptotik dieser Operatoren lässt sich so nach
oben abschätzen, dass, um den Term $-5$ter Ordnung zu finden, letztlich die
Asymptotik von endlich vielen (um genau zu sein 3) dieser einfacheren Operatoren
bestimmt werden muss. Dazu wird die Asymptotik der Spur mithilfe der sehr
einfachen Asymptotik exponentieller Integrale ($F(x) =
\Integ[T]{0}{t}{\Eto{-tx}f(t)}$), bekannt als Watsons Lemma, bestimmt. Damit
erhalten wir
\begin{Hauptsatz}
  \iflanguage{ngerman}{
  Sei $\phi(x)\in\Cinf[0](\Rplus)$ eine Abschneidefunktion, deren Träger in einer
  genügend kleinen Umgebung von $x=0$ liegt. Dann hat die multiparametrische
  Resolventenspur des Sturm-Liouville-Operators
}{
  Let $\phi(x)\in\Cinf[0](\Rplus)$ be a cutoff function that is $0$ outside of a
  sufficiently small neighbourhood of $x=0$. Then the multiparametric
  trace-expansion of the resolvent of the Sturm-Liouville operator
}
\begin{equation*}
  \Delta_\lambda = -\partial_x^2 + \lambda^2 V(x) + W(x)
\end{equation*}
\iflanguage{ngerman}{
  auf $\Rplus$ bezüglich der verallgemeinerten Neumannrandbedingung
  $f(0)\cos\theta + f'(0)\sin\theta = 0$ die folgende asymptotische Entwicklung
  bis zur dritten nicht-verschwindenden Ordnung nahe $x=0$
}{
  on $\mathbb{R_+}$ up to the third non-vanishing order near $x=0$ with the
  generalised Neumann boundary conditions $f(0)\cos\theta + f'(0)\sin\theta = 0$
  is given by
}
\begin{equation*}
  \Tr\left(\phi(x)(\Delta_\lambda + z^2)^{-1}\right) \SimMu
  \frac{1}{(2\mu)^2} + \frac{5\lambda^2}{(2\mu)^5} V'(0) + O(\mu^{-4}),
\end{equation*}
\iflanguage{ngerman}{mit}{with} $\mu^2 := \lambda^2 V(0) + W(0)$.

\end{Hauptsatz}
Die Formel gilt mit geringen Abwandlungen auch für $x=1$. Hier ist die Existenz
der asymptotischen Entwicklung durch einen Satz aus \cite{LV13} gesichert.
Weiterhin beweisen wir im letzten Hauptsatz, dass die asymptotische
Entwicklung wie für die Resolventenspur im Inneren schon lange bekannt, auch am
Rand aus rationalen Funktionen in $V$, $W$ und deren Ableitungen besteht und die
Terme eine ähnliche Homogenität aufweisen.

Unter Anwendung der beiden ersten Hauptsätze erhalten wir somit letztlich den
Fehlerterm in der Lesch-Vertman-Formel als Integral über eine rationale Funktion
in den Ableitungen von $f$ im Inneren und am Rand:
%TODO: Snippet
\begin{align*}
  \Integ[\infty]{0}{\lambda}{
    h(\lambda, 1) \ln\lambda
  }
  &=
  % Interior:
  \Int{x}{\Biggl(\frac{(1 + 2 \ln(2 V(x))) W(x)}{8 V(x)}
    - \frac{ (1 - 2 \ln(2 V(x)))V''(x)}{96 V(x)^{3}} \\
    &\hphantom{=\int_0^1\Biggl(}\mathbin{+} \frac{\left(\frac{5}{3} - 2\ln(2
    V(x))\right)\left(V'(x)\right)^{2}}{64 V(x)^{5}}\Biggr) } \\
&\mathbin- 5\frac{(1 - 2\ln(2 V(1)))V'(1)}{192 V(1)^3}
+ 5\frac{(1 - 2 \ln(2 V(0)))V'(0)}{192 V(0)^3}
\end{align*}
Zuletzt geben wir noch einen solchen Fehlerterm für $f(x) = x + \epsilon$ an und
betrachten den naiven Grenzwert $\epsilon \to 0$.
