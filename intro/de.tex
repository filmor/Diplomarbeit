\section{Einführung}
Diese Arbeit beschäftigt sich mit der asymptotischen Entwicklung der Spur
multiparametrischer Sturm-Liouville-Operatoren auf dem abgeschlossenen Intervall
$[0,1]$. Operatoren dieser Art erhält man beispielsweise bei der Untersuchung
des Laplace-Beltrami-Operators auf einer Rotationsfläche wenn dieser bezüglich
$\mathbb{S}^1$ in seine Komponenten zerlegt wird.

Bei dieser Methode erhält man, wie in LV gezeigt, einen Fehlerterm in Form eines
Integrals über eine bestimmte Komponentenfunktion der asymptotischen Entwicklung
von $\Tr(\Delta_\lambda + z^2)^{-2}$ für $\left|(\lambda, z)\right|\to\infty$.
Diesen genau zu berechnen ist Ziel dieser Arbeit. Dazu werden im Folgenden
Formeln sowohl für die innere Asymptotik, als auch für die Asymptotik gegen die
Randwerte 0 und 1 hergeleitet. Für ersteres verwenden wir dabei den
multiparametrischen Kalkül der Pseudodifferentialoperatoren, für letzteres eine
neue Herangehensweise basierend auf der sehr einfachen Asymptotik exponentieller
Integrale, also Integrale der Form $F(x) = \Integ[T]{0}{t}{\Eto{-tx}f(t)}$.

Zunächst beweisen wir folgenden Hauptsatz:
\begin{Hauptsatz}
  Sei $\Delta_\lambda := -\partial_x^2 + \lambda^2 V(x) + W(x)$ ein operator wie
  beschrieben und $\mu(x) := \sqrt{\lambda^2V(x)+z^2}$. Dann hat die
  Resolventenspur $\Tr(\Delta_\lambda + z^2)^{-1}$ von $\Delta_\lambda$ die
  folgende asymptotische Entwicklung in $\lambda$ und $z$:
  \begin{align*}
    k_{(\Delta_\lambda + z^2)^{-1}} \SimAs{\Abs{(\lambda, z)}\to\infty}
    \frac12\frac1{\mu(x)^{2}}
   -\frac1{16}\frac{\lambda^2 V''(x)}{\mu(x)^5}
   +\frac{5}{64}\frac{\lambda^4V'(x)^2}{\mu(x)^7}
 -\frac14\frac{W(x)}{\mu(x)^3}
 +O\!\bigl(\Abs{(\lambda, z)}^{-5}\bigr)

  \end{align*}  
\end{Hauptsatz}
Die Existenz der Entwicklung ist durch einen Satz der Theorie parametrischer
Pseudodifferentialoperatoren gesichert. Dann zeigen wir mithilfe Watsons
Lemmas, das die erwähnte Asymptotik exponentieller Integrale behandelt, dass für
den Rand folgendes gilt:
\begin{Hauptsatz}
  Die Resolvente sei gegeben durch den Operator $(\Delta_\lambda + z^2)^{-1}$.
  Dann hat die Spur der Resolvente die folgende gemeinsame Asymptotik in
  $\lambda$ und $z$:
  \begin{align*}
    \left.\Tr(\Delta_\lambda + z^2)^{-1}\right|_{x=0} \SimMu
    \frac{1}{(2\mu)^2} + \frac{5\lambda^2}{(2\mu)^5} V'(0) + O(\mu^{-4}),
  \end{align*}
  mit $\mu^2 := \lambda^2 V(0) + z^2$.
\end{Hauptsatz}
Die Formel gilt mit geringen Abwandlungen auch für $x=1$. Hier ist die Existenz
der asymptotischen Entwicklung durch einen Satz aus \cite{LV13} gesichert.  Wir
starten mit expliziten Formeln für den Kern des Operators und zeigen, dass die
jeweiligen Spurintegrale nach Watsons Lemma eine asymptotische Entwicklung
besitzen. Desweiteren zeigen wir im \textbf{Hauptsatz 3}, dass asymptotische
Entwicklung wie für die Resolventenspur im Inneren schon lange bekannt, auch am
Rand aus rationalen Funktionen in $V$, $W$ und deren Ableitungen besteht.

Unter Anwendung der beiden ersten Hauptsätze erhalten wir somit den Fehlerterm
in der Lesch-Vertman-Formel als Integral über eine rationale Funktion in den
Ableitungen von $f$ im Inneren und am Rand:
\begin{align*}
  \Integ[\infty]{0}{\lambda}{
    h_2(\lambda, 1) \ln\lambda
  }
  &=
  % Interior:
  \Int{x}{\Biggl(\frac{(1 + 2 \ln(2 V(x))) W(x)}{8 V(x)}
    - \frac{ (1 - 2 \ln(2 V(x)))V''(x)}{96 V(x)^{3}} \\
    &\hphantom{=\int_0^1\Biggl(}\mathbin{+} \frac{\left(\frac{5}{3} - 2\ln(2
    V(x))\right)\left(V'(x)\right)^{2}}{64 V(x)^{5}}\Biggr) } \\
&\mathbin- 5\frac{(1 - 2\ln(2 V(1)))V'(1)}{192 V(1)^3}
+ 5\frac{(1 - 2 \ln(2 V(0)))V'(0)}{192 V(0)^3}
\end{align*}
Zuletzt geben wir noch einen solchen Fehlerterm für ein explizites Beispiel an.
