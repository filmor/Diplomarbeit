\subsection{Deutsche Einleitung}
% TODO: Special functions erwähnen
Diese Diplomarbeit beschäftigt sich mit der asymptotischen Entwicklung der Spur
multiparametrischer Sturm-Liouville-Operatoren auf dem abgeschlossenen Intervall
$[0,1]$.
%
Operatoren dieser Art erhält man beispielsweise bei der Untersuchung der
Resolvente des Laplace-Beltrami-Operators $\Delta$ auf einer Rotationsfläche
wenn dieser bezüglich der eindimensionalen Sphäre $\mathbb{S}^1$ in seine
Komponenten zerlegt wird, also als eine unendliche direkte Summe von Operatoren
$\Delta =
\bigoplus_{\lambda\in\operatorname{spec}\Delta_\Sphere{1}}\Delta_\lambda$ auf
der direkten Summe der Eigenräume ($\times[0,1]$) geschrieben wird (siehe
\cref{sec:laplace-beltrami}).
%
Die enstehenden Operatoren sind Sturm-Liouville-Operatoren mit einem Parameter
$\lambda$, lassen sich also für Funktionen $V,W\in\Cinf{[0,1]}$ mit $V > 0$
und $W > 0$ schreiben als
\begin{equation*}
  (\Delta_\lambda u)(x) = -u''(x) + \lambda^2 V(x) u(x) + W(x) u(x).
\end{equation*}
Für den Fall des Laplace-Beltrami-Operators auf einer von einer Funktion
$f\in\Cinf{[0,1]}$ mit $f>0$ erzeugten Rotationsfläche sind $V$ und $W$ gegeben
als
\begin{align*}
  V(x) &= \frac1{f(x)^2}, \\
  W(x) &= \frac{f''(x)}{2f(x)} - \left(\frac{f'(x)}{2f(x)}\right)^2.
\end{align*}
In \cite{LV13} wird gezeigt, dass sich die Zetadeterminante des
Laplace-Beltrami-Operators $\Delta$ auf einer Rotationsfläche, (formal) definiert
als
\begin{align*}
  \zeta(s,\Delta) := {} &
  \sum_{\mathclap{\mu\in\operatorname{spec}\Delta\setminus\{0\}}}
  m(\mu)\mu^{-s} \text{ für } \Re(s) > 1, \\
  \log\zdet \Delta = {} & {-\zeta'}(0,\Delta) \text{ durch meromorphe
Fortsetzung},
\end{align*}
in gewisser Weise als regularisierte Summe der Zetadeterminanten der Summanden
zuzüglich eines Fehlerterms schreiben lässt:
\begin{align*}
  \label[equation]{eqn:intro-de-lvformel}
  \log\zdet\Delta = \Regsum_{\mathclap{\lambda =
  -\infty}}^\infty\log\zdet\Delta_\lambda + C
\end{align*}
In diesem Kontext ist die regularisierte Summe $\Regsum$ über die
Hadamard-Regularisierung der Euler-MacLaurin-Formel
definiert (siehe \cref{sec:regsum-int}).

Das Ziel der vorliegenden Arbeit ist es, den in \cref{eqn:intro-de-lvformel} als
$C$ bezeichneten Fehlerterm zu berechnen. In \cite{LV13} wird gezeigt, dass der
Fehlerterm sich aus der asymptotischen Entwicklung von $\Tr(\Delta_\lambda +
z^2)^{-2}$ berechnen lässt, welche wiederum leicht auf die asymptotische
Entwicklung von $\Tr(\Delta_\lambda + z^2)^{-1}$ zurückzuführen ist. Der
Operator
\begin{align*}
  (A_{\lambda,z}u)(x) := (\Delta_\lambda u)(x) + z^2 u(x)
           = -u''(x) + (\lambda^2 V(x) + z^2) u(x) + W(x) u(x)
\end{align*}
ist nun ein \emph{multiparametrischer Sturm-Liouville-Operator}. Ist $h(\lambda,
z)$ der Term der Homogenitätsordnung $-5$ der gemeinsamen asymptotischen
Entwicklung in $(\lambda,z)$ der Spur $\Tr A_{\lambda,z}^{-2}$, so ist
$h(1,\,\cdot\,)\log(\,\cdot\,)\in L^1(0,\infty)$ und es gilt für den Fehlerterm:
\begin{align*}
  C = \Integ[\infty]{0}{\lambda}{h(1,\lambda)\log\lambda}
\end{align*}
Um diesen berechnen zu können wird das asymptotische Verhalten des
Resolventenkerns des Operators $k_{(\Delta_\lambda + z^2)^{-1}}$ sowohl im
Inneren, also auf dem Intervall $(0,1)$, als auch an den Rändern $0$ und $1$
untersucht.

Da wir hierfür die Parametrix (also $(\Delta_\lambda + z^2)^{-1}$ statt
$(\Delta_\lambda + z^2)^{-2}$) betrachten benötigen wir den Term der
Homogenitätsordnung $-3$, denn
\begin{equation*}
  \Tr(\Delta_\lambda + z^2)^{-2} = (-2z)^{-1}\partial_z\Tr(\Delta_\lambda +
  z^2)^{-1},
\end{equation*}
also ist die eine Entwicklung um $2$ gegenüber der anderen verschoben.

Im Inneren kann die Entwicklung mittels der Theorie parametrischer
Pseudodifferentialoperatoren, die auch die Existenz sichert, explizit rekursiv
bestimmt werden, wir erhalten daraus den 
\begin{Hauptsatz}[Resolvente im Inneren]
  % TODO: Existenz und explizite Formel (zumindest bis zum Grad 3).

\end{Hauptsatz}
Neben der expliziten Bestimmung der Koeffizienten können wir noch zeigen, dass
die Terme der Entwicklung nicht nur eine Homogenität in $(\lambda,z)$, sondern
auch in $V$, $W$ und deren Ableitungen aufweisen:
\begin{Hauptsatz}[Homogenität im Inneren]
  \iflanguage{ngerman}{
  Sei die Ordnung $\ord$ definiert als
}{
  Let the order $\ord$ be defined as
}
\begin{align*}
  \ord \lambda^2 V(x) := {} & 0 \\
            \ord W(x) := {} & 2 \\
 \ord \partial_x A(x) := {} & \ord A(x) + 1,
\end{align*}
\iflanguage{ngerman}{
  wobei $A(x)$ ein rationaler Ausdruck in $V$, $W$ und deren Ableitungen sei,
  für eine genaue Definition siehe \cref{def:order}.
}{
  where $A(x)$ is a rational expression in $V$, $W$ and their derivatives, for a
  precise definition cf.\ \cref{def:order}.
}

\iflanguage{ngerman}{
  Dann gilt die folgende Gleichung für die Terme $k_{-n}$ der
  Homogenitätsordnung $-n$ in der asymptotischen Entwicklung des Kerns von
  $(\Delta_\lambda + z^2)^{-1}$ auf der Diagonale:
}{
  Then the following holds for the terms $k_{-n}$ of homogeneity order $-n$ in
  the asymptotic expansion of the kernel of $(\Delta_\lambda +
  z^2)^{-1}$ on the diagonal:
}
\begin{equation*}
  \ord k_{-n}(x, x) = n - 1
\end{equation*} 

\end{Hauptsatz}
Am Rand wird der Operator zunächst mit einem Neumannreihenargument in
einfachere Operatoren zerlegt. Die Asymptotik dieser Operatoren lässt sich so nach
oben abschätzen, dass, um den Term $-5$ter Ordnung zu finden, letztlich die
Asymptotik von endlich vielen (um genau zu sein 3) dieser einfacheren Operatoren
bestimmt werden muss. Dazu wird die Asymptotik der Spur mithilfe der sehr
einfachen Asymptotik exponentieller Integrale ($F(x) =
\Integ[T]{0}{t}{\Eto{-tx}f(t)}$), bekannt als Watsons Lemma, bestimmt. Wir
erhalten daraus
\begin{Hauptsatz}[Resolventenspur am Rand]
  \iflanguage{ngerman}{
  Sei $\phi(x)\in\Cinf[0](\Rplus)$ eine Abschneidefunktion, deren Träger in einer
  genügend kleinen Umgebung von $x=0$ liegt. Dann hat die multiparametrische
  Resolventenspur des Sturm-Liouville-Operators
}{
  Let $\phi(x)\in\Cinf[0](\Rplus)$ be a cutoff function that is $0$ outside of a
  sufficiently small neighbourhood of $x=0$. Then the multiparametric
  trace-expansion of the resolvent of the Sturm-Liouville operator
}
\begin{equation*}
  \Delta_\lambda = -\partial_x^2 + \lambda^2 V(x) + W(x)
\end{equation*}
\iflanguage{ngerman}{
  auf $\Rplus$ bezüglich der verallgemeinerten Neumannrandbedingung
  $f(0)\cos\theta + f'(0)\sin\theta = 0$ die folgende asymptotische Entwicklung
  bis zur dritten nicht-verschwindenden Ordnung nahe $x=0$
}{
  on $\mathbb{R_+}$ up to the third non-vanishing order near $x=0$ with the
  generalised Neumann boundary conditions $f(0)\cos\theta + f'(0)\sin\theta = 0$
  is given by
}
\begin{equation*}
  \Tr\left(\phi(x)(\Delta_\lambda + z^2)^{-1}\right) \SimMu
  \frac{1}{(2\mu)^2} + \frac{5\lambda^2}{(2\mu)^5} V'(0) + O(\mu^{-4}),
\end{equation*}
\iflanguage{ngerman}{mit}{with} $\mu^2 := \lambda^2 V(0) + W(0)$.

\end{Hauptsatz}
Die Formel gilt mit geringen Abwandlungen auch für $x=1$. Hier ist die Existenz
der asymptotischen Entwicklung durch einen Satz aus \cite{LV13} gesichert.
Weiterhin beweisen wir im letzten Hauptsatz, dass die asymptotische
Entwicklung wie für die Resolventenspur im Inneren schon lange bekannt, auch am
Rand aus rationalen Funktionen in $V$, $W$ und deren Ableitungen besteht und die
Terme eine ähnliche Homogenität aufweisen:
\begin{Hauptsatz}[Homogenität am Rand]
  \iflanguage{ngerman}{
  Mit der selben Definition wie in Hauptsatz~2 gilt für einen Term $A$ der
  asymptotischen Entwicklung der Resolventenspur von $\Delta_\lambda$ am Rand
  mit der Homogenitätsordnung $-n$ in $(\lambda,z)$:
}{
  Using the same definition for $\ord$ as in Main~Theorem~2 we have for a term
  $A$ in the asymptotic expansion of the resolvent trace of $\Delta_\lambda$ at
  the boundary of homogeneity $-n$ in $(\lambda,z)$:
}
\begin{equation*}
  \ord A = n - 2
\end{equation*}

\end{Hauptsatz}

Unter Anwendung der beiden ersten Hauptsätze erhalten wir somit letztlich den
Fehlerterm in der Lesch-Vertman-Formel als Integral über eine rationale Funktion
in den Ableitungen von $f$ im Inneren und am Rand unter Anwendung eines
Integralsatzes aus der Theorie der speziellen Funktionen:
%TODO: Snippet
\begin{align*}
  \Integ[\infty]{0}{\lambda}{
    h(\lambda, 1) \log\lambda
  }
  &=
  % Interior:
  \Int{x}{\Biggl(\frac{(1 + 2 \log(2 V(x))) W(x)}{8 V(x)}
    - \frac{ (1 - 2 \log(2 V(x)))V''(x)}{96 V(x)^{3}} \\
    &\hphantom{=\int_0^1\Biggl(}\mathbin{+} \frac{\left(\frac{5}{3} - 2\log(2
    V(x))\right)\left(V'(x)\right)^{2}}{64 V(x)^{5}}\Biggr) } \\
&\mathbin- 5\frac{(1 - 2\log(2 V(1)))V'(1)}{192 V(1)^3}
+ 5\frac{(1 - 2 \log(2 V(0)))V'(0)}{192 V(0)^3}
\end{align*}
Zuletzt geben wir noch einen solchen Fehlerterm für $f(x) = x + \epsilon$ an und
betrachten den naiven Grenzwert $\epsilon \to 0$.
