\documentclass[paper=a4,twoside,parskip=full]{scrartcl}

\usepackage[utf8]{inputenc}
\usepackage{amsmath,amssymb,amsfonts,fancyhdr,csquotes,tikz,enumerate}
\usepackage[colorlinks=true,linkcolor=blue,citecolor=blue]{hyperref}
\usepackage[standard,thmmarks,hyperref,thref]{ntheorem}
\usepackage[ngerman,english,british]{babel}
% \usepackage[backend=biber]{biblatex}
\usepackage[scale=0.8]{geometry}
\usepackage[T1]{fontenc}
% \usepackage{lmodern}
\usepackage{esint}
% \usepackage{mathpazo}
\usepackage{eulervm}
% \usepackage{fourier}
% TODO: Subequation?

\pagestyle{fancy}
\fancyhf{}
\fancyhead[EL]{\scriptsize\leftmark}
\fancyhead[OR]{\scriptsize\rightmark}
\fancyfoot[EL]{\thepage}
\fancyfoot[OR]{\thepage}
\renewcommand{\headrulewidth}{0.1pt}

% Fix bold headings
\def\bfseries{\fontseries \bfdefault \selectfont \boldmath}

\newtheorem{MainTheorem}{Main Theorem}

\numberwithin{equation}{section}
\numberwithin{Lemma}{section}
\numberwithin{Theorem}{section}
\numberwithin{Remark}{section}

\newcommand{\Thema}{Zur Resolventenspur multiparametrischer
Sturm-Liouville-Operatoren}
\newcommand{\ThemaEn}{On the Resolvent Trace of multi-parametric Sturm-Liouville
Operators}

\author{Benedikt Christian Sauer}
\title{\Thema}

\newcommand{\Eto}[1]{\ensuremath{\mathrm e^{#1}}}
\newcommand{\Integ}[4][]{\ensuremath{\int_{#2}^{#1}\!#4\,\mathrm d{#3}}}
\newcommand{\Int}[2]{\Integ[1]{0}{#1}{#2}}

\newcommand{\Abs}[1]{\ensuremath{\left|#1\right|}}
\newcommand{\Wsum}[2][0]{\ensuremath{\sum_{n=#1}^\infty \frac{#2(0)}{(2\mu)^{n+1}}}}
\newcommand{\Regint}[2]{\ensuremath{\fint_{1}^{\infty}\!#2\,\mathrm d{#1}}}
\newcommand{\Oscint}[2]{\Integ{\mathrm O}{#1}{#2}}
\DeclareMathOperator*{\Reglim}{LIM}
% TODO
\DeclareMathOperator{\Regsum}{\sum}
\newcommand{\Sphere}[1]{\ensuremath{{\mathbb{S}^{#1}}}}
\newcommand{\Cinf}[1][]{\ensuremath{C^\infty_{#1}}}
\newcommand{\Rplus}{\ensuremath{{\mathbb{R}_+}}}

\DeclareMathOperator{\Tr}{Tr}
\DeclareMathOperator{\vol}{vol}
\newcommand{\SimAs}[1]{\ensuremath{\mathrel{\underset{#1}{\sim}}}}
\newcommand{\SimMu}{\ensuremath{\SimAs{\mu\to\infty}}}

\renewcommand{\Re}{\ensuremath{\operatorname{Re}}}
\renewcommand{\Im}{\ensuremath{\operatorname{Im}}}
\renewcommand{\theta}{\vartheta}
\renewcommand{\phi}{\varphi}

\begin{document}
\begin{titlepage}
  \begin{center}
      \textsc{Diplomarbeit} \\
      \bigskip
      \textit{\Thema} \\
      \textit{(\ThemaEn)}
  \end{center}
  \vspace{\stretch{1}}
  \begin{center}
      Angefertigt am \\
      Mathematischen Institut
  \end{center}
  \vspace{\stretch{1}}
  \begin{center}
      Vorgelegt der \\
      Mathematisch-Naturwissenschaftlichen Fakultät der \\
      Rheinischen Friedrich-Wilhelms-Universität Bonn
  \end{center}
  \vspace{\stretch{1}}
  \begin{center}
      Dezember 2012 \\
      \bigskip
      Von \\
      \bigskip
      Benedikt Christian Sauer \\
      \bigskip
      geboren am \\
      30. März 1988 \\
      in \\
      Bonn\,--\,Bad~Godesberg
  \end{center}
\end{titlepage}
\newpage
\tableofcontents
\vfill
\begin{center}
  \large 1.\ Gutachter: Matthias Lesch \\
  \large 2.\ Gutachter: \\
\end{center}
\newpage
\selectlanguage{ngerman}
\section{Einführung}
Diese Arbeit beschäftigt sich mit der asymptotischen Entwicklung der Spur
multiparametrischer Sturm-Liouville-Operatoren auf dem abgeschlossenen Intervall
$[0,1]$. Operatoren dieser Art erhält man beispielsweise bei der Untersuchung
des Laplace-Beltrami-Operators auf einer Rotationsfläche wenn dieser bezüglich
$\mathbb{S}^1$ in seine Komponenten zerlegt wird.

% \dots

Bei dieser Methode erhält man, wie in LV gezeigt, einen Fehlerterm in Form eines
Integrals über eine bestimmte Komponentenfunktion der asymptotischen Entwicklung
von $\Tr(\Delta_\lambda + z^2)^{-2}$ für $\left|(\lambda, z)\right|\to\infty$.
Diesen genau zu berechnen ist Ziel dieser Arbeit. Dazu werden im Folgenden
Formeln sowohl für die innere Asymptotik, als auch für die Asymptotik gegen die
Randwerte 0 und 1 hergeleitet. Für ersteres verwenden wir dabei den
multiparametrischen Kalkül der Pseudodifferentialoperatoren, für letzteres eine
neue Herangehensweise basierend auf der sehr einfachen Asymptotik exponentieller
Integrale, also Integrale der Form $F(x) = \Integ[T]{0}{t}{\Eto{-tx}f(t)}$.

Zunächst beweisen wir folgenden Hauptsatz:
% MThm 1: Asymptotik im Inneren
Die Existenz der Entwicklung ist durch einen Satz der Theorie parametrischer
Pseudodifferentialoperatoren gesichert, der ebenfalls bewiesen wird.

Dann zeigen wir mithilfe Watsons Lemmas, dass für den Rand folgendes gilt:
% MThm 2: Asymptotik am Rand
Hier ist die Existenz der asymptotischen Entwicklung per Konstruktion gesichert.
Wir starten mit expliziten Formeln für den Kern des Operators und zeigen, dass
die jeweiligen Spurintegrale nach Watsons Lemma eine asymptotische Entwicklung
besitzen.

Unter Anwendung beider Hauptsätze erhalten wir somit den Fehlerterm als Polynom
$n$-ten Grades in den Ableitungen von $f$ im Inneren und am Rand:
% Cor:    Explizite Formel für den Fehlerterm in f^(n)


\selectlanguage{british}
\subsection{English Introduction}
The theme of this diploma thesis is the asymptotic expansion of the trace of
multi-parametric Sturm-Liouville operators on the closed interval $[0,1]$.
%
Operators of this kind arise on investigation of the resolvent of the
Laplace-Beltrami operator $\Delta$ on a surface of revolution when splitting it
into components with regard to the one-dimensional sphere $\Sphere{1}$. The
surface of revolution is thereby expressed as an infinite direct sum $\Delta =
\bigoplus_{\lambda\in\operatorname{spec}\Delta_\Sphere{1}}\Delta_\lambda$, as an
operator over the direct sum of the eigenspaces
($\times[0,1]$).\ref{sec:laplace-beltrami}
%
The operators $\Delta_\lambda$ are Sturm-Lioville operators with one parameter
$\lambda$ and using $V,W\in\Cinf([0,1])$, $V>0$ and $W>0$ can be written as
\begin{align*}
  (\Delta_\lambda u)(x) = -u''(x) + \lambda^2 V(x) u(x) + W(x) u(x)
\end{align*}
In the given case of the Laplace-Beltrami operator on a surface of revolution
generated by a function $f\in\Cinf([0,1])$ with $f>0$ the functions $V$ and $W$
are given by
\begin{align*}
  V(x) &= \frac1{f(x)^2} \\
  W(x) &= \frac{f''(x)}{2f(x)} - \left(\frac{f'(x)}{2f(x)}\right)^2.
\end{align*}

It is proven in \cite{LV13} that the zeta determinant, (formally) defined for an
operator $A$ by
\begin{align*}
  \zeta(s,A) &:= \sum_{\mu\in\operatorname{spec}A\setminus\{0\}} m(\mu)\mu^{-s}
  \\
  \log\zdet A &= -\zeta'(0,A),
\end{align*}
of the direct sum $\bigoplus_\lambda\Delta_\lambda$ can be calculated as a
regularised sum of the zeta determinants of the $\Delta_\lambda$ with an
additional finite error term:
\begin{align*}
  \log\zdet\Delta = \Regsum_\lambda\zdet\Delta_\lambda + C
\end{align*}
The regularised sum $\Regsum$ is defined by the Hadamard finite part
regularisation of the Euler-MacLaurin formula.\ref{sec:regsum-int}

The goal is now to calculate the error term $C$, which is essentially given by
an integral over a term in the multiparametric asymptotic expansion of
$\Tr(\Delta_\lambda + z^2)^{-2}$ which can be derived from the expansion of
$\Tr(\Delta_\lambda + z^2)^{-1}$. The operator
\begin{align*}
  (A_{\lambda,z}u)(x) := (\Delta_\lambda u)(x) + z^2 u(x)
           = -u''(x) + (\lambda^2 V(x) + z^2) u(x) + W(x) u(x)
\end{align*}
is a \emph{multi-parametric Sturm-Liouville operator} with parameters $\lambda$
and $z$. For $h(\lambda,z)$ being the $-5^{\text{th}}$ term in the common
asymptotic expansion in $(\lambda,z)$ of the trace $\Tr A_{\lambda,z}^{-2}$ the
error term is given by:
\begin{align*}
  C = \Integ[\infty]{0}{\lambda}{h(1,\lambda)\log\lambda}
\end{align*}
To evaluate the integral we calculate the first terms of the asymptotic
expansion of $\Tr(\Delta_\lambda + z^2)^{-1}$ in the interior $(0,1)$ as well as
at the boundaries $0$ and $1$.

Using the theory of parametric pseudo-differential operators we can show the
following
\begin{MainTheorem}
  % TODO: Existenz und explizite Formel (zumindest bis zum Grad 3).

\end{MainTheorem}
Additionally we proof that the resulting homogeneous polynomials in $\lambda$
and $z$ also employ homogeneity with regard to $V$, $W$ and their respective
derivatives.

At the boundary we first split the operator $\Delta_\lambda + z^2$ into simpler
operators using a Neumann series argument. The contributions of those simpler
operators can be estimated such that we only have to investigate the asymptotics
of a finite number (in this particular case 3) of them to calculate the trace
asymptotics up to a given order.
%
To accomplish this we employ Watson's Lemma, which directly gives us the
asymptotics of an exponential integral ($F(x) =
\Integ[T]{0}{t}{\Eto{-tx}f(t)}$). This gives us
\begin{MainTheorem}
  \iflanguage{ngerman}{
  Sei $\phi(x)\in\Cinf[0](\Rplus)$ eine Abschneidefunktion, deren Träger in einer
  genügend kleinen Umgebung von $x=0$ liegt. Dann hat die multiparametrische
  Resolventenspur des Sturm-Liouville-Operators
}{
  Let $\phi(x)\in\Cinf[0](\Rplus)$ be a cutoff function that is $0$ outside of a
  sufficiently small neighbourhood of $x=0$. Then the multiparametric
  trace-expansion of the resolvent of the Sturm-Liouville operator
}
\begin{equation*}
  \Delta_\lambda = -\partial_x^2 + \lambda^2 V(x) + W(x)
\end{equation*}
\iflanguage{ngerman}{
  auf $\Rplus$ bezüglich der verallgemeinerten Neumannrandbedingung
  $f(0)\cos\theta + f'(0)\sin\theta = 0$ die folgende asymptotische Entwicklung
  bis zur dritten nicht-verschwindenden Ordnung nahe $x=0$
}{
  on $\mathbb{R_+}$ up to the third non-vanishing order near $x=0$ with the
  generalised Neumann boundary conditions $f(0)\cos\theta + f'(0)\sin\theta = 0$
  is given by
}
\begin{equation*}
  \Tr\left(\phi(x)(\Delta_\lambda + z^2)^{-1}\right) \SimMu
  \frac{1}{(2\mu)^2} + \frac{5\lambda^2}{(2\mu)^5} V'(0) + O(\mu^{-4}),
\end{equation*}
\iflanguage{ngerman}{mit}{with} $\mu^2 := \lambda^2 V(0) + W(0)$.

\end{MainTheorem}
This formula holds, with minor changes, also for $x=1$. The existence of the
asymptotic expansion is proven in \cite{LV13}.
%
Furthermore we prove in the final Main Theorem, that the asymptotics at the
boundary has the same homogeneity properties as the expansion in the interior,
namely that the terms are again homogeneous in $V$, $W$ and their respective
derivatives.

By applying the first two Main Theorems we can evaluate the error term in the
Lesch-Vertman formula as the integral over a rational function in the
derivatives of $f$ in the interior and at the boundary:
% TODO

\subsection{Conventions}
%TODO
We follow the conventions used in \cite{Les:PDO}, hence:

Throughout this thesis the following conventions are used:
\paragraph{Fourier Transform}
\begin{align*}
  \hat{u}(\xi) = (\mathcal{F}u)(\xi)
    &:= \Integ{\mathbb{R}}{x}{\Eto{-\mathrm ix\xi} u(x)} \\
  (\mathcal{F}^{-1}\hat{u})(x) &:=
  \frac{1}{2\pi}\Integ{\mathbb{R}}{\xi}{\Eto{\mathrm ix\xi}
    \hat{u}(\xi)}
\end{align*}

We will always talk about symbols and only mention amplitudes when explicitly
needed. We thus denote the \emph{symbol} of an operator $A$ by $a$ instead of
$\sigma_A$. The kernel of an operator $A$ is denoted by $k_A$, i.e.\ 
\begin{align*}
  (Au)(x) =: \Integ{\mathbb{R}}{y}{k_A(x,y) u(x)},
\end{align*}
where $k_A$ may be a distribution.


\section{Setting}
% Multitude of references to ML and BV
% Abschnitt vom Anfang von LV de-facto kopieren, eigene Definitionen benutzen un
% genauer angeben, was was ist und woraus das jeweils folgt.

\subsection{Laplace-Beltrami operator on a surface of revolution}
Let $M = [0,1]\times\Sphere{1}$ be a surface of revolution by use of the metric
$g = \mathrm dx^2 \oplus f(x)^2 g_\Sphere{1}$ with $f\in \Cinf([0,1],
\Rplus)$. The associated Laplace-Beltrami operator is given by
\begin{align}
  \Delta = -\frac{\partial^2}{\partial x^2} -
            \frac{f'(x)}{f(x)}\frac{\partial}{\partial x} +
            \frac{1}{f(x)^2}\Delta_\Sphere1,
\end{align}
acting on $\Cinf[0]\bigl((0,1)\times\Sphere1\bigr)$. The correct Lebesgue
measure on the space $(M,g)$ is given by $f(x)\,\mathrm dx\, \mathrm
d\!\vol(g_\Sphere1)$. Under the unitary map
\begin{align*}
  \Phi: L^2(M, g)&\to L^2([0,1], \mathrm dx)\otimes L^2(\Sphere1, g_\Sphere1) \\
     (\Phi u)(x) &:= u(x) \sqrt{f(x)},
\end{align*}
the Laplacian $\Delta$ transforms into the operator
% TODO: Proof
\begin{align*}
  \Phi\Delta\Phi^{-1} = -\frac{\partial}{\partial x^2} +
  \frac{1}{f(x)^2}\Delta_\Sphere1 + \left( \frac{f''(x)}{2f(x)} -
  \left(\frac{f'(x)}{2f(x)}\right)^2\right).
\end{align*}
The functions $f_\lambda(x) := \frac{1}{\sqrt{2\pi}}\Eto{\mathrm i\lambda x}$
form for $\lambda\in\mathbb{Z}$ an orthonormal basis of $L^2(\Sphere1)$ of
eigenfunctions of $\Delta_\Sphere1$ to the eigenvalues $\lambda^2$, where the
eigenvalues $\lambda^2 \neq 0$ have multiplicity 2 while $\lambda = 0$ has
multiplicity 1. Hence we have the decomposition
% TODO: Proof
\begin{align}
  \label{eqn:lpl-decomp}
  \Phi\Delta\Phi^{-1} &= \bigoplus_{\lambda=-\infty}^{\infty} \left(
    -\frac{\partial^2}{\partial x^2} + \frac{\lambda^2}{f(x)^2} + 
     \left( \frac{f''(x)}{2f(x)} -
     \left(\frac{f'(x)}{2f(x)}\right)^2\right)\right)
     =: \bigoplus_{\lambda=0}^{\infty}\Delta_\lambda
\end{align}
into a direct sum of one-dimensional Sturm-Liouville operators $\Delta_\lambda$.
We consider separated Dirichlet or generalised Neumann boundary conditions for
$\Delta$, where it is straight-forward to see, that those also correspond to
separated Dirichlet, resp.\ generalised Neumann boundary conditions for
$\Phi\Delta\Phi^{-1}$ and that the resulting self-adjoint operator is compatible
with the given decomposition \eqref{eqn:lpl-decomp}. Since the transformed
operator does indeed behave, from an analytical point of view, the same as the
operator $\Delta$ we will denote the self-adjoint extensions of $\Delta$ and
$\Delta_\lambda$ again by the same name. Since we are considering fixed boundary
conditions this is not too ambiguous.

We will consider operators like $\Delta_\lambda$, where we will set
\begin{align}
  V(x) &:= f(x)^{-2} \\
  W(x) &:= \frac{f''(x)}{2f(x)} - \left(\frac{f'(x)}{2f(x)}\right)^2 =
  \partial_x^2 \ln\sqrt{f(x)} + \left(\partial_x \ln\sqrt{f(x)}\right)^2 \\
  \text{thus}\quad \Delta_\lambda &= -\partial_x + \lambda^2 V(x) + W(x),
\end{align}
where the non-derivative terms as usual denote the multiplication operators
$(M_\phi u)\colon x \mapsto \phi(x)u(x)$. We won't write this out (i.~e.\ in the
summands of $\Delta_\lambda$ we use $W(x)$ instead of $M_W$) however we will 

\subsection{Regularised sums and integrals}
% TODO Define asymptotic expansion

% TODO log-homogeneous
Now let $f\in\Cinf(\Rplus)$ be a function with a partial asymptotic expansion
\begin{align*}
  f(x) \SimAs{x\to\infty} \sum_{j=0}^{N-1} a_{j}\,x^{-j} + x^{-N}f_N(x),
\end{align*}
where $N\in\mathbb{N}$ is arbitrary and the remainder $f_N(x) = o(1)$ as
$x\to\infty$. We define the \emph{regularised limit} for $x\to\infty$ as
\begin{align}
  \Reglim_{x\to\infty} f(x) := a_0,
\end{align}
i.~e.\ as the constant part of the expansion.

If for some $N\in\mathbb{N}$ the remainder $(x\mapsto x^{-N}f_N(x))\in
L^1[1,\infty)$, the integral $\Integ[R]{1}{x}{f(x)}$ does also admit an
asymptotic expansion of the form given above and we define its \emph{regularised
integral} as
\begin{align}
  \Regint{x}{f(x)} := \Reglim_{R\to\infty}\Integ[R]{1}{x}{f(x)}.
\end{align}
We also see here the necessity for using $x^n \log^m(x)$ as base-functions in
our asymptotic expansion, as only then the integral over the asymptotic
expansion results in a meaningful expansion for which we can then use the given
limit-definition. This approach of regularising potentially infinite integrals
is called Hadamard finite part (or French: partie finie) regularisation. In the
special case that $f\in L^1[1,\infty)$ the regularised integral coincides with
the usual Lebesgue integral of $f$.
% TODO R\to 0

Using this we can also give a meaningful definition of a \emph{regularised sum}
over $f$. Given the Bernoulli numbers $B_j$ and the Bernoulli polynomials
$B_j(x)$ consider the Euler-MacLaurin summation formula:
\begin{align*}
  \sum_{\lambda=1}^N f(\lambda) &= \Integ[N]{1}{x}{f(x)}
  + \sum_{k=1}^M \frac{B_{2k}}{(2k)!}\left(f^{(2k-1)}(N) - f^{(2k-1)}(1)\right)
  \\
  &\mathbin{+} \frac1{(2M + 1)!}\Integ[N]{1}{x}{B_{2m+1}(x-[x])f^{(2M+1)}(x)}
  \\
  &\mathbin{+} \frac12(f(1)+f(N))
\end{align*}
By essentially replacing the integrals by regularised integrals we arrive at the
following definition for the regularised sum:
\begin{align*}
  \Regsum_{\lambda=1}^\infty := \Reglim_{N\to\infty} \sum_{\lambda=1}^N
  f(\lambda)
\end{align*}
% TODO Auch von -inf bis inf

The following theorem is proven in LV12 and the basis of the further
considerations:
\begin{Theorem}
  Assume $f\in\Cinf(\Rplus^2)$ is of the form
  \begin{align*}
    f(x,y) = \sum_{j=0}^{N-1} f_{\alpha_j}(x,y) + F_N(x,y)
  \end{align*}
  where $N\in\mathbb{N}$ and each $f_{\alpha_j}\in\Cinf(\Rplus^2\setminus{0})$
  is homogeneous of order $\alpha_j\in\mathbb{C}$ in both variables jointly,
  where the remainder $F_N$ is assumed to be in $L^1[1,\infty)^2$. Then
  \begin{align}
    \Regint{x}{\Regint{y}{f(x,y)}} =
    \Regint{y}{\Regint{x}{f(x,y)}} + \Int{y}{f_{-2}(1,y)\ln(y)}
  \end{align}
\end{Theorem}

\subsection{Zeta determinants}
% TODO
    
\subsection{Goal}
Lesch and Vertman proved in (LV) that
\begin{align}
  \Tr(\Delta + z^2)^{-2} = \sum_{\lambda=-\infty}^\infty \Tr(\Delta_\lambda +
  z^2)^{-2} \SimAs{z\to\infty} \sum_{k=2}^{\infty} a_kz^{-k}
\end{align}
and using this asymptotic expansion, that zeta-determinant of $\Delta$ is given
by a regularised sum of the zeta-determinants of the $\Delta_\lambda$ with an
error term:
\begin{align}
  \ln\operatorname{det}_\zeta \Delta = \Regsum_{\lambda=-\infty}^\infty
  \ln\operatorname{det}_\zeta\Delta_\lambda -
  4\Integ[\infty]{0}{\lambda}{h_2(\lambda, 1)\ln(\lambda)},
\end{align}
where $h_2(\lambda, z)$ denotes the homogeneous term of degree $-5$ (i.e.\ the second
non-vanishing term) in the polyhomogeneous asymptotic expansion of
$\Tr(\Delta_\lambda + z^2)^{-2}$ as $\Abs{(\lambda, z)}\to\infty$. The ultimate
goal of this thesis is to calculate the error term by calculating the
first terms in the polyhomogeneous trace-expansion of the resolvent of a
Sturm-Liouville operator.

By formally deriving $\Tr(\Delta_\lambda + z^2)^{-1}$ we see that
\begin{align*}
  \partial_z \Tr{(\Delta_\lambda + z^2)}^{-1} &=
        (-2z) \Tr{(\Delta_\lambda + z^2)}^{-2}
        \Leftrightarrow& \Tr(\Delta_\lambda + z^2)^{-2} =
        (-2z)^{-1}\partial_z\Tr(\Delta_\lambda + z^2)^{-1},
\end{align*}
thus the order of the homogeneous components is shifted by 2. This means, that
to get the component of order $-5$ we will need to find the homogeneous
component in the trace-expansion of the parametrix of order $-3$.

\section{Asymptotics of the Interior Parametrix}
In this section we aim to prove the following Theorem:
\begin{MainTheorem}
  \label{main:interior}
  % TODO: Existenz und explizite Formel (zumindest bis zum Grad 3).

  \begin{Remark}
    Note again, that the order we are looking for here is \emph{not} $-5$ but
    instead $-3$ (i.e.\ the three terms in the middle) since we are dealing with
    $\Tr(\Delta_\lambda + z^2)^{-1}$ instead of $\Tr(\Delta_\lambda + z^2)^{-2}$
    here.
  \end{Remark}
  \begin{Remark}
    For $V\equiv 0$ this expansion has already been proven in
    \cite[Ch 2.1]{Gel'fand1975} (where $z^2 \mapsto \zeta$ and $W(x) \mapsto
    u(x)$) and calculated for even higher orders, but we will restrict ourselves
    in this text to what we need for the particular calculation we want to do in
    the end.
  \end{Remark}
\end{MainTheorem}
To prove this we first note that the asymptotic expansion exists by a classical
theorem on the theory of pseudo-differential operators with parameter, namely
\begin{Theorem}
  Let $U\subset\mathbb{R}^n$ be open, $\Gamma\subset\mathbb{R}^p$ a cone and
  $A\in\mathrm{CL}^m(U;\Gamma)$ (the set of classical pseudo-differential
  operators with parameter in $\Gamma$).

  If $m+n<0$ then for all $\mu\in\Gamma$ the operator $A(\mu)$ is trace-class
  and the kernel on the diagonal $x\mapsto k_A(x, x;\mu)$ is in
  $\mathrm{CS}^{m+n}(\Gamma)$ (the space of classical symbols), with the
  asymptotic expansion
  \begin{equation*}
    k_A(x,x;\mu)\SimAs{\Abs{\mu}\to\infty}
    \sum_{j=0}^{\infty} e_{m-j}\bigl(x,\tfrac{\mu}{\Abs{\mu}}\bigr)
    \Abs{\mu}^{m+n-j}.
  \end{equation*}
  \begin{Proof}
    A proof can be found in \cite[Thm 5.1]{Les:PDO}, where it is shown in the
    slightly more general case of log-polynomial symbols. The idea is to use
    the asymptotic expansion of the symbol, which is given for classical
    operators, and explicitly evaluate the Schwartz kernel on the diagonal by
    means of the formula
    \begin{equation*}
      k_A(x,x;\mu) = \frac{1}{2\pi}\Integ{\mathbb R}{x}{a(x,x;\mu)}.
    \end{equation*}
    A reference for the formula and explanation is given in the next subsection
    as we will use it there for concrete calculations.
  \end{Proof}
\end{Theorem}
Using the existence as well as the product formula for pseudo-differential
operators, we prove the explicit formula in $V$, $W$ and their derivatives.

\subsection{Calculation of the first Coefficients}
The multi-parametric symbol of the operator $(\Delta_\lambda + z^2)$ is given by
\begin{equation}
  \label{eqn:symbol}
  \begin{split}
    a_2(x,\xi,\lambda,z) &= \xi^2 + \lambda^2 V(x) + z^2 \\
    a_0(x,\xi,\lambda,z) &= W(x),
  \end{split}
\end{equation}
with all other $a_n \equiv 0$. To not complicate the following calculations we
suppress the continuous repetition of the parameters, a prime ($a_2'$) denotes
the derivative in $x$.

We use the following product formula for symbols: Let $A$ and $B$ be
pseudo-differential operators with symbols $a$ and $b$, respectively. Then the
complete symbol of $(\Delta_\lambda + z^2)B$ is (modulo $S^\infty$) given by
\begin{equation}
  a * b := \sum_{n=0}^{\infty} \frac{1}{n!}\ \partial_\xi^n a\,D_x^n b,
  \label{eqn:product-formula}
\end{equation}
where $D_x^n := (-i)^n \partial_x^n$. A proof of this formula can be found in
\cite[Thm 3.4]{Shu:POS}.

We are interested in the symbol of the parametrix, so we demand that (as
symbols)
\begin{equation*}
  a * b \sim 1,
\end{equation*}
so the components of the asymptotic expansion of $a * b$ in the symbol spaces
are supposed to be
\begin{equation}
  \label{eqn:ab-as-symbols}
  (a * b)_n \equiv \delta_{n0}
\end{equation}
if $B$ is a parametrix of $A$. Actually this will
also work for $b * a$, i.e.\ $BA$, but this complicates further calculations as
with the chosen order we only get $\xi$-derivatives of $a = a_2 + a_0$ which
are
\begin{equation}
  \partial_{\xi} a = 2\xi \text{ and } \partial_{\xi}^2 a = 2.
\end{equation}
Thus the sum in \cref{eqn:product-formula} collapses and only three terms are
left:
\begin{equation*}
  a * b = (a_2 + a_0)b - 2\xi\mathrm i\,\partial_x b - \partial_x^2 b
  \ \mathrel{\overset{!}{\sim}}\ 1,
\end{equation*}
which gives by use of \cref{eqn:ab-as-symbols}
\begin{equation}
  \label{eqn:int-recursion}
  b_{-n} = \frac1{a_2} \left( 2\xi\mathrm i\, \partial_x b_{-(n-1)}
  + \partial_x^2 b_{-(n-2)} - a_0 b_{-(n-2)} \right),
\end{equation}
with the starting values $b_0 = 0$, $b_{-1} = 0$ and $b_{-2} = a_2^{-1}$ since
the order of the inverse has to be $-2$. We can now inductively derive the first
coefficients of the parametrix of $(\Delta_\lambda + z^2)$:
\begin{equation}
  \label{eqn:coeff-symbol}
  \begin{split}
    b_{-2} &= \frac{1}{a_2} \\
    b_{-3} &= -2\frac{\xi a_2'}{a_2^3}\\
    b_{-4} &= 4\xi^2\left(\frac{a_2''}{a_2^4}
    - 3\frac{(a_2')^2}{a_2^5}\right) - \frac{a_0}{a_2^2}
    - 2\frac{a_2''}{a_2^3} + 2\frac{(a_2')^2}{a_2^4}
  \end{split}
\end{equation}
Since we are ultimately interested in the kernel on the diagonal we will need to
integrate the resulting symbol of the parametrix, as for a pseudo-differential
operator $A$ with symbol $a$ we have
\begin{equation*}
  k_A(x,y) = \frac{1}{2\pi}\Oscint{\xi}{
    \Eto{\mathrm i\left\langle x-y, \xi\right\rangle} a(x,\xi)
  }
  \quad
  \Rightarrow
  \quad
  k(x,x) = \frac{1}{2\pi}\Integ{\mathbb{R}}{\xi}{a(x,\xi)},
\end{equation*}
where $\Oscint{\xi}{\,\cdot\,}$ denotes the oscillatory integral (as defined in
\cite[Def.~1.1]{Shu:POS}) which coincides with the usual integral over
$\mathbb{R}$ for $x = y$. The integrals over $\xi$ can be dealt with in a
uniform way using the beta function as seen in the introduction
(\cref{lem:beta_function_formula}) by letting $n$ be the order of $\xi$ in the
numerator, $m$ the exponent of $a_2$ in the denominator, $a := \mu(x) =
\sqrt{\lambda^2 V(x) + W(x)}$ and $b = 1$. We also see, that for odd $n$ the
terms vanish, as we have an integral over an odd function, while for even $n$ we
get an additional factor of $2$ for integrating over $\mathbb{R}$ instead of
$\Rplus$.

The corresponding terms in the kernel are thus
\begin{equation}
  \begin{split}
    k_{-1}(x,x,\lambda,z) &= \frac{1}{2} \frac{1}{(\lambda^2 V(x) + z^2)^{1/2}} \\
    k_{-2}(x,x,\lambda,z) &= 0 \\
    k_{-3}(x,x,\lambda,z) &= - \frac{1}{16} \frac{\lambda^2 V''(x)}{(\lambda^2
      V(x) + z^2)^{5/2}} + \frac{5}{64} \frac{\lambda^4 V'(x)^2}{(\lambda^2V(x)
      + z^2)^{7/2}} -\frac{1}{4}\frac{W(x)}{(\lambda^2 V(x) + z^2)^{3/2}}.
    \end{split}
  \label{eqn:coeff-kernel}
\end{equation}
We note, that the $(\lambda,z)$-order of the integrated terms increases by $1$
which is reflected by a change of the index.

These are the first three asymptotic components of the kernel of
$(\Delta_\lambda + z^2)^{-1}$ on the open interval $(0,1)$, i.e.\ if we define
$\mu(x) := \sqrt{\lambda^2 V(x) + z^2}$ as above we have
\begin{align*}
  k_{(\Delta_\lambda + z^2)^{-1}} \SimAs{\Abs{(\lambda, z)}\to\infty}
    \frac12\frac1{\mu(x)^{2}}
   -\frac1{16}\frac{\lambda^2 V''(x)}{\mu(x)^5}
   +\frac{5}{64}\frac{\lambda^4V'(x)^2}{\mu(x)^7}
 -\frac14\frac{W(x)}{\mu(x)^3}
 +O\!\bigl(\Abs{(\lambda, z)}^{-5}\bigr)
,
\end{align*}
which proves \thref{main:interior}. The asymptotic order of the remainder of
$-5$ follows from the fact, that by construction every other step in our
recursion formula results in the exponents of $\xi$ in the numerator being odd,
so the $\xi$ integral over $\mathbb{R}$ vanishes, in particular we get $k_{-4} =
0$.

Finally for this to give us all the information we were looking for we need to
carry out the $x$-Integration. Since by \cref{frm:shift,eqn:central-thm} we
know, that we will only differentiate the formula and integrate the result
multiplied with a $\log\lambda$-term, and all intermediate steps are integrable
we can postpone the evaluation until we have enough information on $V$ and $W$
to perform the integration.

\subsection{Homogeneity in $V$, $W$ and their Derivatives}
We see by closely looking on the generating \cref{eqn:int-recursion} that we can
(without carrying out the explicit calculations) show that the terms in the
kernel expansion have a certain structure.  First, we define an order for the
building blocks in the numerator of the terms in the expansion:
\begin{Definition}
  \label{def:order}
  We define an order on terms of $V$ and $W$ by:
  \begin{align*}
    \ord \lambda^2 V(x) &:= 0 \\
              \ord W(x) &:= 2
  \end{align*}
  For $A(x)\neq 0$ and $B(x)$ being rational expressions in $V$, $W$ and their
  derivatives, and $C\in\mathbb{C}$ we also define the following calculation
  rules:
  \begin{align*}
    \ord C &:= 0 \\
    \ord A(x)^{C} &:= C\ord A(x) \\
    \ord \partial_x A(x) &:= \ord(A(x)) + 1 \\
    \ord A(x)B(x) &:= \ord A(x) + \ord B(x)
  \end{align*}
  The sum of two terms $A(x)$ and $B(x)$ has a well-defined order if $\ord A(x)
  = \ord B(x)$, namely $\ord(A(x) + B(x)) = \ord A(x) = \ord B(x)$.
  \begin{Remark}
    The definition can be motivated from our toy example given in the
    introduction where we had
    \begin{align*}
      V(x) = f(x)^{-2} \text{ and } W(x) = \frac{f''(x)}{2f(x)} -
      \left(\frac{f'(x)}{2f(x)}\right)^2.
    \end{align*}
    If we now rescale this as $x\mapsto kx$ we get
    \begin{align*}
      V(x) &\mapsto f(kx)^{-2} = V(kx) \text{ while} \\
      W(x) &\mapsto \frac{k^2 f''(kx)}{2f(x)} -
      \left(\frac{kf'(kx)}{2f(x)}\right)^2 = k^2 W(kx),
    \end{align*}
    $\ord$ thus represents the rescaling exponent. This does also immediately
    give the calculation rules, especially the increase of $\ord$ on
    differentiation comes directly from the fact that
    \begin{equation*}
      \partial^n f\circ(x\mapsto kx)(x) = k^n f(kx)
    \end{equation*}
    (which also proves the well-definedness of $\ord$).
  \end{Remark}
\end{Definition}
We know that by construction, apart from some complex constant, only derivatives
of $\lambda^2 V(x)$ (in form of derivatives of $a_2$), $W(x)$ itself as well as
derivatives of $W(x)$ may appear in the numerator of a term in the asymptotic
expansion. The denominator is uniquely defined by the count of $\lambda^2
V^{(l)}(x)$ terms in the numerator, i.e.\ if there are $k$ factors of $\lambda^2
V^{(l)}$, then the denominator is $(\lambda^2 V(x) + z^2)^{(n+2k-1)/2}$, by
$(\lambda,z)$-homogeneity.

For the numerator we can see, that the order of the expression always adds up to
$n-1$ ($n$ being the $(\lambda,z)$-order, for example if we consider $n=3$ we
have the following possibilities
\begin{align*}
  V''(x), W(x), (V'(x))^2,
\end{align*}
since those are exactly the terms with order $2$ we can produce. This is what we
will prove now:
\begin{MainTheorem}
  Given the previous definition of the order $\ord$ in $V$, $W$ and their
  respective derivatives the following holds for the terms in the asymptotic
  expansion of the kernel of $(\Delta_\lambda + z^2)^{-1}$ on the diagonal:
  \begin{equation}
    \label{eqn:ord-vs-mu-order}
    \ord k_{-n}(x, x) = n - 1
  \end{equation}
  \begin{Proof}
    First we note that the $\xi$-integration does not change the order of the
    expression since
    \begin{equation*}
      \ord \frac{\xi^n}{(\lambda^2 V(x) + z^2 + \xi^2)^m} = \ord (\lambda^2 V(x)
      + z^2)^{n-2m+1} = 0,
    \end{equation*}
    thus the order of the term before and after the integration is the same.
    However, since the $(\lambda,z)$-order changes, the indexation is different,
    i.e.\ $k_{-n}$ corresponds to $b_{-(n+1)}$. We
    thus prove the claim
    \begin{equation*}
      \ord k_{-(n-1)} = \ord b_{-n} = n - 2
    \end{equation*}
    by induction over $n$ on symbol level. We start with $b_{-2} = a_2^{-1} =
    (\lambda^2 V(x) + z^2 + \xi^2)^{-1}$, which clearly has order $0$. Since the
    recursion formula involves the two previous terms we also need to check
    $b_{-3}$:
    \begin{equation*}
      \ord b_{-3} = \ord\left(2\xi \frac{a_2'}{a_2^3}\right) = \ord a_2' = \ord
      \lambda^2 V'(x) = 1
    \end{equation*}
    For the induction step we only need to look closely on the recursion formula
    (\cref{eqn:int-recursion}) and use the given calculation rules:
    \begin{align*}
      &\ord 2\xi i \partial_x b_{-(n-1)} = \ord \partial_x b_{-(n-1)} = ((n-1) - 2) +
      1 = n - 2\\
      &\ord \partial_x^2 b_{-(n-2)} = ((n - 2) - 2) + 2 = n - 2 \\
      &\ord a_0 b_{-(n-2)} = \ord W(x) b_{-(n-2)} = 2 + ((n-2) - 2) = n - 2
    \end{align*}
    Since $b_{-n}$ is the sum of those terms divided by $a_2$, which has order
    $0$ this proves the homogeneity in $V$, $W$ and their derivatives.
  \end{Proof}
\end{MainTheorem}

\section{Asymptotics of the Boundary Parametrix}
By using a summation formula that is derived in \cite[2.2]{LV13}, the existence
and values of the asymptotic components of $\Tr(\Delta_\lambda + z^2)^{-1}$ at
the boundary can be calculated directly. By \cref{frm:shift} we know, that we
only have to consider terms up to order $O(\Abs{(\lambda,z)}^{-3})$ since the
requested asymptotic coefficient is of order $-3$ in the resolvent.  The method
derived in this section can however be used to also extract higher orders.

\subsection{Preliminaries}
We will first briefly step through the process that justifies the formula for
the trace-resolvent we are going to use, based on \cite{LV13}. We denote the
multiplicative part of the operator $\Delta_\lambda$ by 
\begin{equation}
  \label{def:f-lambda}
  \begin{split}
    f_\lambda(x) :={}& \lambda^2 V(x) + W(x)\text{, i.e.} \\
   \Delta_\lambda ={}& {-\partial_x^2} + f_\lambda(x).
  \end{split}
\end{equation}
Without loss of generality we will use $x = 0$ as the model case here, by the
substitution $x\mapsto 1-x$ we get very similar results for $x=1$ which will be
formulated as a Corollary. The parametrix of $\Delta_\lambda + z^2$ near the
boundary $x=0$ can be constructed explicitly. We first consider $L :=
\Delta_\lambda - f_\lambda(x) = -\partial_x^2$ acting on
$C_0^\infty(\mathbb{R})$, which is essentially self-adjoint in
$L^2(\mathbb{R})$, we denote the self-adjoint extension by $\overline L$. Now,
for $\theta \in [0,\pi)$, let $L^\theta$ be $\overline L$ restricted to
\begin{equation*}
  \mathcal{D}(L^\theta) := \left\{ f\in H^1\Rplus\, \middle|\, f(0)\cos\theta +
  f'(0)\sin(\theta) = 0\right\},
\end{equation*}
i.e.\ we employ generalised Neumann boundary conditions. The case $\theta =
\pi$, which is $f'(0) = 0$, is taken out of this although it behaves essentially
the same, we will note it where this matters.

For $\mu\in\mathbb{C}$, $\Re\mu > 0$ one can show, that the resolvent kernel of
$(\overline L + \mu^2)^{-1}$ is given by
\begin{equation*}
  k_{(\overline L + \mu^2)^{-1}}(x,y;\mu) = \frac1{2\mu}\Eto{-\mu\Abs{x-y}},
\end{equation*}
and the kernel of $(L^\theta+\mu^2)^{-1}$ by
\begin{align}
  \label{frm:k-l-theta}
  k_{(L^\theta +\mu^2)^{-1}}(x,y;\mu) = \frac1{2\mu} \left( \Eto{-\mu\Abs{x-y}} +
  C(\mu,\theta)\Eto{-\mu(x+y)}\right), \\
  \label{frm:c-mu-theta}
  \text{with} \quad C(\mu,\theta) :=
  \frac{\mu\sin\theta+\cos\theta}{\mu\sin\theta-\cos\theta}.
\end{align}
However we are interested in the kernel of $(\Delta_\lambda + z^2)^{-1}$ for the
chosen boundary conditions near $x = 0$, namely $(L^\theta + f_\lambda(x) +
z^2)^{-1}$. Let
\begin{equation}
  \label{def:mu}
  \mu^2 := \lambda^2 V(0) + z^2
\end{equation}
and note that $\mu^2 = (\mu(0))^2$ using the definition from
\cref{main:interior}. We will use this abbreviation in this section also as a
substitute for $\Abs{(\lambda,z)}$. Since $V\in C_0^\infty([0,1])$ and $V > 0$
we have $V(0) > 0$ (and $V(1) > 0$), this gives of course the same analytical
results, in particular asymptotics for $\mu\to\infty$ is the same as asymptotics
for $\Abs{(\lambda,z)}\to\infty$.

Now we aim to rewrite the operator-inverse such that we are only inverting
operators that don't depend explicitly on the functions $V$ and $W$, in
particular in terms of $(L+\mu^2)$ and $(L^\theta+\mu^2)$ for which we have
explicit expressions of their inverses' kernels. We assume for now for some
$\delta > 0$ sufficiently small that $\supp V \subset [0,\delta]$ and
$\Abs{V-V(0)}_{\infty}\leq \frac12V(0)$. Algebraically we have
\begin{equation*}
  (L^\theta + \lambda^2 V(x) + W(x) + z^2)^{-1} =
    \bigl(I + (L^\theta + \mu^2)^{-1}(\lambda^2\tilde V(x) + W(x))\bigr)^{-1}
    (L^\theta + \mu^2)^{-1}.
\end{equation*}
Since for $z$ sufficiently large we have (with $\OpNorm{\,\cdot\,}$ being the
operator norm)
\begin{align*}
  \OpNorm{(L^\theta + \mu^2)^{-1}(\lambda^2\tilde V(x) + W(x))} \leq
  \frac{\lambda^2\InfNorm{\tilde V} + \InfNorm{W}}{\lambda^2 V(0) + z^2},
  \text{as } \InfNorm{\tilde V} \leq \frac12V(0)
\end{align*}
we can write this as a converging Neumann series:
\begin{multline}
  \label{frm:neumann-sum}
  \Rightarrow (L^\theta + \lambda^2 V(x) + W(x) + z^2)^{-1} \\ =
    \sum_{j=0}^\infty (-1)^j \left((L^\theta + \mu^2)^{-1}(\lambda^2\tilde V(x)
    + W(x))\right)^j (L^\theta + \mu^2)^{-1}.
\end{multline}
Now, back to our original $V$ and $W$, we introduce cut-off functions $\phi,
\psi\in\Cinf(\Rplus)$ that are $1$ in a neighbourhood of $x = 0$ and $0$ on
$[\delta,\infty)$. Furthermore $\supp\phi\subset\supp\psi$ and
$\supp\psi\,\cap\,\supp\mathrm d\psi = \emptyset$. Setting $W_\psi(x) :=
\psi(x)W(x)$ and $\tilde V_\psi(x) = \psi(x)V(x) - V(0)$ we have by choosing
$\delta$ small enough $\InfNorm{\tilde V_\psi} < \frac12V(0)$, as needed for the
Neumann series expansion. Let
\begin{align}
  \label{frm:r-delta}
  R_\delta &:= \psi(L^\theta + \lambda^2V_\psi(x) + W_\psi(x) + z^2)^{-1}\phi \\
  \label{frm:f-lambda}
  \text{and } f_\lambda(x) &:= \lambda^2\tilde V_\psi(x) + W_\psi(x).
\end{align}
Note that $f_\lambda(x)$ is notated $\lambda(V,W)(x)$ in \cite{LV13}, we'll use
this notation to make the distinction between $\lambda$ as a parameter and the
function (which is of order $2$ in $\lambda$) clearer.

It is shown in \cite[Sec.~2]{LV13}, that the previous choices do not change the
asymptotic behaviour of the resolvent trace at $x=0$ and that for $R_\delta$ the
right-hand-side in \cref{frm:neumann-sum} is well-defined:
\begin{equation*}
  \Tr((L^\theta + \lambda^2 V(x) + W(x) + z^2)^{-1}\phi) =
  \Tr R_\delta + O(\mu^{-\infty}), \quad \text{as } \mu\to\infty.
\end{equation*}
We define
\begin{equation}
  \label{frm:r-theta}
  R^\theta := (L^\theta + \mu^2)^{-1},
\end{equation}
for which we already know the kernel by \cref{frm:k-l-theta}:
\begin{equation*}
  k_{R^\theta}(x,y;\mu) = \frac1{2\mu} \left( \Eto{-\mu\Abs{x-y}} +
  C(\mu,\theta)\Eto{-\mu(x+y)}\right)
\end{equation*}
We denote by $R_-$ and $R_+$ the operators belonging to the kernels
\begin{align*}
  k_{R_-}(x,y;\mu) := \tfrac1{2\mu}\Eto{-\mu\Abs{x-y}} \text{ and }
  k_{R_+}(x,y;\mu) := \tfrac1{2\mu}C(\mu,\theta)\Eto{-\mu(x+y)},
\end{align*}
such that $R^\theta = \tfrac1{2\mu}(R_- + R_+)$. Now we can write our Neumann
series as
\begin{equation*}
  R_\delta = \sum_{j=0}^\infty (-1)^j M_\psi (R^\theta
  M_{f_\lambda})^j R^\theta M_\phi,
\end{equation*}
where we have for once explicitly written out $M_f$ to emphasise that we mean
the multiplication operator, whose kernel is given by the Dirac
delta-distribution as
\begin{equation*}
  k_{M_f}(x,y) = \delta(x-y)f(x).
\end{equation*}
Now we insert our newly defined operators $R_-$ and $R_+$ into the sum and split
of the terms that only contain $R_-$. To simplify the expressions we define an
abbreviation for the compositions. Let $n\in\mathbb{N}$ and $\sigma \in
\{+,-\}^n$, i.e.\ a tuple of symbols $+$ and $-$, which we will denote when
given explicitly by, e.g.\ $++--+$. Then we recursively (starting with the
already defined $R_-$ and $R_+$) define
\begin{equation}
  R_\sigma := R_{\sigma_1} M_{f_\lambda} R_{(\sigma_2,\ldots,\sigma_{n})},
\end{equation}
for which we can also give the kernels explicitly by use of the general formula
$k_{AB}(x,y) = \Integ{\mathbb{R}}{z}{k_A(x,z)k_B(z,y)}$, where we, by slight
abuse of notation, reuse the variable $z$ in the following to denote the
integration variable.  This won't lead to ambiguity as the resolvent variable
$z$ is hidden in $\mu$ for the whole section. We have
\begin{align}
  \label{eqn:kernel-plus}
  k_{R_{(+,\sigma_1,\ldots,\sigma_n)}}(x,y) &=
  \frac1{2\mu}C(\mu,\theta)\Int{z}{\Eto{-\mu(x+z)}f_\lambda(z)
  k_{R_\sigma}(z,y)} \\
  \label{eqn:kernel-minus}
  k_{R_{(-,\sigma_1,\ldots,\sigma_n)}}(x,y) &=
  \frac1{2\mu}\Int{z}{\Eto{-\mu\Abs{x-y}}f_\lambda(z) k_{R_\sigma}(z,y)}.
\end{align}

We can now express, by use of $R_\theta = R_- + R_+$, $R_\delta$ in the
following way:
\begin{align}
  R_\delta = \sum_{j=0}^\infty (-1)^j \psi(x)
  \ \sum_{\mathclap{\sigma\in\{+,-\}^{j+1}}} R_\sigma \phi(x)
\end{align}
The two last results we have to cite are the fact, that terms with $\sigma =
(-,\ldots,-)$ don't contribute to the asymptotics, thus we can rewrite the
formula as
\begin{equation}
  R^{(j)} = (-1)^j \sum_{\mathclap{\substack{ \sigma\in\{+,-\}^{j+1} \\
  \sigma\neq(-,\ldots,-) }}} \psi R_\sigma \phi
  \quad \Rightarrow \quad R^0_\delta =
  \sum_{j=0}^\infty R^{(j)},
\end{equation}
and the following theorem (\cite[Prop.~2.2]{LV13}), that will be central to the
further considerations:
\begin{Theorem}
  \label{thm:r-expansion}
  Let $M\in\mathbb{N}$, $\alpha,\beta\in\mathbb{N}_0$ be fixed. For $\mu_0$
  sufficiently large there exist constants $C>0$, $q\in(0,1)$ such that for
  $N\geq M$ and $\mu \geq \mu_0$
  \begin{equation*}
    \Norm[\mathrm{tr}]{\partial_\lambda^\alpha\partial^\beta_z R^{(N)}}
    \leq CNq^{N-M}\mu^{-M-\alpha-\beta-3/2}
  \end{equation*}
  and thus for $\mu\to\infty$
  \begin{equation*}
    \Norm[\mathrm{tr}]{\partial^\alpha_\lambda\partial^\beta_z\sum_{j=M}^\infty
    R^{(j)}} = O(\mu^{-M-\alpha-\beta-3/2}),
  \end{equation*}
  where $\|\cdot\|_{\mathrm{tr}}$ denotes the trace norm
  ($\|A\|_\mathrm{tr} = \Tr\bigl((A^*A)^{1/2}\bigr) = \Tr\Abs{A}$).
\end{Theorem}
This tells us that since we are aiming to find the asymptotics up to order $2$
we need to consider $R^{(j)}$ up to $j=1$, since already the trace of $R^{(2)}$
is $O(\mu^{-7/2})$, i.e.\ $o(\mu^{-3})$.

\subsection{Calculations}
Equipped with the definitions and theorems of the preceding subsection we can
explicitly calculate the asymptotics of the resolvent trace, which we will do
now for $x=0$. The case $x=1$ is identical to this one if we employ the
transformation $x\mapsto 1-x$, which results in a sign change for odd
derivatives of $f_\lambda(x)$, and of course evaluation at $x=1$ instead of $0$.
From the above considerations we know that we will have to calculate the
asymptotics of $3$ traces.

In the following calculations we define analogous to $R_\sigma$ the factor
\begin{equation}
  C_{(\sigma_1,\ldots,\sigma_n)} := (-1)^{n+1}\Bigl(\frac{1}{2\mu}\Bigr)^n
  \prod_{ \substack{ 1\le i \le n \\ \sigma_i = - }} C(\mu,\theta).
\end{equation}
For $c$ being the count of minuses in $\sigma$ we can see that
$\lim_{\mu\to\infty} C_\sigma = (-1)^{n+c+1} (2\mu)^{-n}$, since $C(\mu,\theta)
\to -1$. For $\theta = \pi$ we'd get a different result here, namely $C_\sigma
\to (-1)^{n+1}(2\mu)^{-n}$ since $C(\mu,\pi)\to +1$.

The calculation of all terms with $\sigma=(+,\ldots,+)\in\{-,+\}^j$ is
straightforward:
\begin{equation}
  \label{frm:r-plusplus}
  \begin{split}
    \Tr R_{+\ldots+} &=
    C_{+\ldots+}
    \Int{x}{\left(
      \Integ{[0,1]^{j-1}}{z}{
        \Eto{-\mu((x + z_1) + (z_1 + z_2) + \ldots + (z_{j-1} + x))}
        \prod_{i=1}^{j-1} f_\lambda(z_i)
      }\right)
    } \\
    &= C_{+\ldots+}
    \Int{x}{\Eto{-2\mu x}} \left(\Int{z}{\Eto{-2\mu
    z}f_\lambda(z)}\right)^{j-1} \\
    &= C_{+\ldots+}
    \left(\frac1{2\mu} - \frac{\Eto{-2\mu}}{2\mu}\right)
    \left(\Int{z}{\Eto{-2\mu z}f_\lambda(z)}\right)^{j-1} \\
    &\SimMu (-2\mu)^{-j-1}
    \left(\Wsum{f_\lambda^{(n)}}\right)^{j-1} = (-1)^{j-1} (2\mu)^{-2j}
    \left(\sum_{n=0}^\infty\frac{f_\lambda^{(n)}(0)}{(2\mu)^n}\right)^{j-1}.
  \end{split}
\end{equation}
The last transformation uses Watson's lemma as well as the fact that we can
apply it simultaneously to multiple integrals to determine the common
asymptotics, which is ensured by the finiteness of $F(x) =
\Int{x}{\Eto{-tx}f(t)}$ using \cref{lem:continuity_of_watson_integrals}.

From this calculation we immediately get the first term we were looking for:
\begin{equation}
  \label{eqn:tr1}
  \Tr R^{(1)} = \Tr R_+ \SimAs{\mu\to\infty} (2\mu)^{-2}
\end{equation}

For higher $j$ terms we need to employ both Watson's lemma and our triangle
integration formula (\cref{lem:triangle-integration}). The process (which can
also be used for higher orders) is the following:
\begin{enumerate}
  \item Write down the whole integral with one large exponential
  \item Combine all terms that appear trivially in the exponent (i.e. not in the
    absolute value of a difference) and split of the $\Int{x}{\Eto{-2\mu
    x}f_\lambda(x)}$-term
  \item For each absolute value of the form $\Abs{x-y}$ in the exponent split
    the corresponding integral in $x > y$ and $x < y$
  \item The resulting parameter in the integral limits can be coped with using
    the integration formula, which flattens the integral
  \item The integral is now in the form of multiple exponential integrals
    multiplied together, so we apply Watson's lemma on each of them
\end{enumerate}
That way we derive a finite product of infinite sums as the asymptotic
expansion. To get an actual coefficient from that we use combinatorics (we could
of course just combine the sum using the Cauchy product formula, but that will
just result in a much more complicated term).

From \cref{frm:r-plusplus} we know, that 
\begin{equation*}
  \Tr R_{++} = -(2\mu)^{-4}\sum_{n=0}^\infty \frac{f^{(n)}_\lambda(0)}{(2\mu)^n},
\end{equation*}
leading to two contributions. Now we only need to calculate $\Tr R_{+-}$ for
$j=1$, since $\Tr R_{+-} = \Tr R_{-+}$:
\begin{align*}
  \Tr R_{+-} &= C_{+-} \Int{x}{
      \Int{z}{
        \Eto{-\mu(x+z+\Abs{z-x})}
        f_\lambda(z)
      }
    } \\
    &\text{split the integral at $x$} \\
    &= C_{+-} \Int{x}{
      \left(
        \Integ[x]{0}{z}{
          \Eto{-\mu(x+z+x-z)} f_\lambda(z)
        }
      + \Integ[1]{x}{z}{
          \Eto{-\mu(x+z-x+z)} f_\lambda(z)
        }
      \right)
    } \\
    &= C_{+-} \left(\Int{x}{\Integ[x]{0}{z}{\Eto{-2\mu x} f_\lambda(z)}}
    + \Int{x}{\Integ[1]{x}{z}{\Eto{-2\mu z}f_\lambda(z)}} \right) \\
    &= C_{+-} \left(
      \Int{x}{\Eto{-2\mu x} F_\lambda(x)}
      + \Int{x}{\Eto{-2\mu x} xf_\lambda(x)}
      \right) \\
      % Result:
    &\SimAs{\mu\to\infty} %
      (2\mu)^{-2}\left(
      \Wsum{F_\lambda^{(n)}} + \Wsum{nf_\lambda^{(n-1)}}\right)
      = (2\mu)^{-3} \sum_{n=1}^\infty\frac{(n+1)f_\lambda^{(n-1)}(0)}{(2\mu)^n},
\end{align*}
since $F_\lambda(0) = 0$. We also used the identity $(xf(x))^{(n)} = xf^{(n)}(x)
+ nf^{(n-1)}(x)$ (which results in the $nf_\lambda^{(n-1)}(0)$ term as we
evaluate at $0$) as well as the triangle integration
(\thref{lem:triangle-integration}) with $g \equiv 1$.  For the symmetry reasons
given above this is also the asymptotics of $\Tr R_{-+}$, so we have:
\begin{equation}
  \label{eqn:tr2}
  \begin{split}
    \Tr R^{(2)} &\SimAs{\mu\to\infty} (2\mu)^{-4} \sum_{n=0}^\infty
  \frac{2(n+2)f_\lambda^{(n)}(0) - f^{(n)}_\lambda(0)}{(2\mu)^n} \\ &=
    (2\mu)^{-4} \sum_{n=0}^\infty \frac{(2n + 3)f_\lambda^{(n)}(0)}{(2\mu)^n}
  \end{split}
\end{equation}
Now that we have calculated $\Tr R^{(1)}$ and $\Tr R^{(2)}$ we only need to sum
them up and match coefficients to get the first three terms in the asymptotic
expansion of $\Tr(\Delta_\lambda + z^2)^{-1}$. In particular we have to deal
with the fact that $f_\lambda(x) = \lambda^2 V_\psi(x) + W_\psi(x)$ is of mixed
order.  Furthermore we note that, as it was for the interior parametrix, the
coefficients of the trace expansion are again a homogeneous polynomial in $V$,
$W$ and their respective derivatives, which will be proven in the next section.
In contrast to the interior parametrix the even integral orders do not vanish by
construction. Also the starting order is $-2$ instead of $-1$.

We get the following three terms of the asymptotic expansion of $\Tr R$:
\begin{align}
  (\Tr R)_{-2} &= \frac{1}{4\mu^2} = \frac{1}{4} (\lambda^2V(0) + z^2)^{-1} \\
  (\Tr R)_{-3} &= \frac{5\lambda^2}{(2\mu)^5} V'(0) =
  \frac{5}{32}\frac{\lambda^2 V'(0)}{(\lambda^2 V(0) + z^2)^{5/2}}
\end{align}

We have thus proven the following theorem
\begin{MainTheorem}
  \label{main:boundary}
  \iflanguage{ngerman}{
  Sei $\phi(x)\in\Cinf[0](\Rplus)$ eine Abschneidefunktion, deren Träger in einer
  genügend kleinen Umgebung von $x=0$ liegt. Dann hat die multiparametrische
  Resolventenspur des Sturm-Liouville-Operators
}{
  Let $\phi(x)\in\Cinf[0](\Rplus)$ be a cutoff function that is $0$ outside of a
  sufficiently small neighbourhood of $x=0$. Then the multiparametric
  trace-expansion of the resolvent of the Sturm-Liouville operator
}
\begin{equation*}
  \Delta_\lambda = -\partial_x^2 + \lambda^2 V(x) + W(x)
\end{equation*}
\iflanguage{ngerman}{
  auf $\Rplus$ bezüglich der verallgemeinerten Neumannrandbedingung
  $f(0)\cos\theta + f'(0)\sin\theta = 0$ die folgende asymptotische Entwicklung
  bis zur dritten nicht-verschwindenden Ordnung nahe $x=0$
}{
  on $\mathbb{R_+}$ up to the third non-vanishing order near $x=0$ with the
  generalised Neumann boundary conditions $f(0)\cos\theta + f'(0)\sin\theta = 0$
  is given by
}
\begin{equation*}
  \Tr\left(\phi(x)(\Delta_\lambda + z^2)^{-1}\right) \SimMu
  \frac{1}{(2\mu)^2} + \frac{5\lambda^2}{(2\mu)^5} V'(0) + O(\mu^{-4}),
\end{equation*}
\iflanguage{ngerman}{mit}{with} $\mu^2 := \lambda^2 V(0) + W(0)$.

  \begin{Corollary}
    \label{cor:boundary}
    Using the previous definitions we also get the expansion near $1$ by
    considering a new cutoff function $\psi$ on $(0,1]$ that is $0$ outside of a
    sufficiently small neighbourhood of $x=1$:
    \begin{equation*}
      \Tr\left(\psi(x)(\Delta_\lambda + z^2)^{-1}\right) \SimMu
      \frac{1}{(2\mu)^2} - \frac{5\lambda^2}{(2\mu)^5} V'(1) + O(\mu^{-4})
    \end{equation*}
    \begin{Proof}
      Apply \thref{main:boundary} to the operator
      \begin{equation*}
        \tilde\Delta_\lambda = \Delta_\lambda\circ(x\mapsto 1-x).
      \end{equation*}
      That way all analytical properties are kept, but $\partial_x^n|_{x=0}
      \mapsto (-1)^n \partial_x^n|_{x=1}$, so a sign is introduced for odd
      derivatives and we evaluate at $x=1$.
    \end{Proof}
  \end{Corollary}
\end{MainTheorem}

\section{Asymptotics of $\Tr{(\Delta_\lambda + z^2)^{-2}}$}
By formal derivation we see that
\begin{align*}
    \partial_z\Tr{(\Delta_\lambda + z^2)}^{-1} &=
        (2z) \Tr{(\Delta_\lambda + z^2)}^{-2}.
\end{align*}
Since we saw in the previous sections that
\begin{align*}
    \Tr{(\Delta_\lambda + z^2)}^{-1}
    \ \SimAs{\Abs{(z,\lambda)}\to\infty}\ 
\end{align*}

\section{Compilation}
Now that we have calculated the coefficients we only need to integrate the
homogeneous part with the logarithm. To do that we use the Mellin transform,
since
\begin{align*}
    \left.\frac\partial{\partial z}\right|_{z=0}\Integ[\infty]{0}{x}{x^z f(x)}
    &= \Integ[\infty]{0}{x}{\ln{(x)} f(x)}
\end{align*}
if one of the integrals exist. We thus calculate:
\begin{align*}
    \Integ[\infty]{0}{\lambda}{\frac{\lambda^{n + z}}{\left(1 +
    \lambda^2\right)^{\frac{n+5}{2}}}} &=
    % Substituiere \lambda^2 = y
    \frac12\Integ[\infty]{0}{y}{\frac{y^{\frac{n+z+1}{2}}}{(1 +
    y)^{\frac{n+5}{2}}}} \\
    % Betafunktion
    &= \frac12 B\bigl(\tfrac{n+z+1}{2}, \tfrac{4-z}{2}\bigr)
\end{align*}
where $B(a,b)$ is the beta function. % Benutztes Integral mit ref!
% a = n+z+1/2, b = n+5/2 - a

The function is symmetric, and its
derivative is $\partial_a B(a,b) = B(a,b) (\psi(a) - \psi(a+b))$, where
$\psi(x) = \partial_x \ln\Gamma(x)$ is the digamma function, so we get
\begin{align*}
    \left.\frac12\frac\partial{\partial z}\right|_{z=0} B\bigl(\tfrac{n+z+1}{2},
    \tfrac{4-z}{2}\bigr)
    &= \left.\frac14 B\bigl(\tfrac{n+z+1}{2}, \tfrac{4-z}{2}\bigr)
    \left(\psi\bigl(\tfrac{n+z+1}{2}\bigr) - \psi\bigl(\tfrac{n+5}{2}\bigr) -
    \psi\bigl(\tfrac{4-z}{2}\bigr) +
    \psi\bigl(\tfrac{n+5}{2}\bigr)\right)\right|_{z=0} \\
    &= \frac14 B\bigl(\tfrac{n+1}{2}, 2\bigr)
    \left(\psi\bigl(\tfrac{n+1}{2}\bigr) - \psi\bigl(2\bigr)\right)
\end{align*}

$B\bigl(\tfrac{n+1}{2}, 2\bigr)$ can be evaluated using the formula given above
to be $\tfrac{4}{(n+1)(n+3)}$. Using the recurrence relation $\psi(x + 1) =
\psi(x) + \tfrac{1}{x}$ and the special values $\psi(1) = -\gamma$ and
$\psi(\tfrac{1}{2}) = -\gamma - 2\ln 2$ ($\gamma$ being the Euler-Mascheroni
constant) we arrive at the following explicit formula for the integral:
\begin{align*}
    \Integ[\infty]{0}{\lambda}{\frac{\lambda^n}{(1 +
    \lambda^2)^{\frac{n+5}{2}}}\ln{\lambda}} = \frac{1}{(n+1)(n+3)}
        \begin{cases} 
            -1 - 2\ln 2 + \sum_{k=1}^{\frac n 2}\frac{2}{2k - 1} & n\text{
            even} \\
            -1 + \sum_{k=1}^{\frac{n-1}{2}}\frac 1 k & n\text{ odd}
        \end{cases}
\end{align*}

\section{Summary}
We have given methods and formulas to calculate the resolvent-trace asymptotics
for Sturm-Lioville operators on a closed interval, where our results for the
interior are in line with the prior work in \cite[Gel'fand1975]. The methods for
the evaluation of the boundary asymptotics could be expanded, especially the
proof of the homogeneity can potentially be used for an algorithmical approach
on the problem.

\appendix
\section{Special Functions}
\subsection{Gamma Function}
\subsection{Digamma Function}
\subsection{Beta Function}

\section{Proof of Watson's lemma}
\label{sec:proof-watson}

\newpage
\selectlanguage{ngerman}
\pagestyle{plain}
\begin{center}
    \textbf{\large Eidesstattliche Erklärung}
\end{center}
\vspace{2cm}
Hiermit erkläre ich, Benedikt Christian Sauer, an Eides statt, dass ich die
Diplomarbeit \textit{"`\Thema"'} selbstständig verfasst und keine anderen als
die angegebenen Hilfsmittel benutzt sowie Zitate kenntlich gemacht habe. \\
\vspace{2cm} \\
Bonn, den \today
\end{document}
