\documentclass[paper=a4,twoside,parskip=full]{scrartcl}

\usepackage[utf8]{inputenc}
\usepackage{amsmath,amssymb,amsfonts,fancyhdr,csquotes,tikz,enumerate}
\usepackage[colorlinks=true,linkcolor=blue,citecolor=blue]{hyperref}
\usepackage[standard,thmmarks,hyperref,thref]{ntheorem}
\usepackage[ngerman,english,british]{babel}
% \usepackage[backend=biber]{biblatex}
\usepackage[scale=0.75]{geometry}
\usepackage[T1]{fontenc}
% \usepackage{lmodern}
\usepackage{slashed}
% \usepackage{mathpazo}
\usepackage{eulervm}
% \usepackage{fourier}
% TODO: Subequation?

\pagestyle{fancy}
\fancyhf{}
\fancyhead[EL]{\scriptsize\leftmark}
\fancyhead[OR]{\scriptsize\rightmark}
\fancyfoot[EL]{\thepage}
\fancyfoot[OR]{\thepage}
\renewcommand{\headrulewidth}{0.1pt}

% Fix bold headings
\def\bfseries{\fontseries \bfdefault \selectfont \boldmath}

\newtheorem{MainTheorem}{Main Theorem}
\newtheorem{Hauptsatz}{Hauptsatz}

\numberwithin{equation}{section}
\numberwithin{Lemma}{section}
\numberwithin{Theorem}{section}
\numberwithin{Remark}{section}

\newcommand{\Thema}{Zur Resolventenspur multiparametrischer
Sturm-Liouville-Operatoren}
\newcommand{\ThemaEn}{On the Resolvent Trace of multi-parametric Sturm-Liouville
Operators}

\author{Benedikt Christian Sauer}
\title{\Thema}

\newcommand{\Eto}[1]{\ensuremath{\mathrm e^{#1}}}

\newcommand{\BaseInteg}[5]{\ensuremath{#5_{#2}^{#1}\!#4\,\mathrm d{#3}}}
\newcommand{\Integ}[4][]{\BaseInteg{#1}{#2}{#3}{#4}{\int}}
\newcommand{\Int}[2]{\Integ[1]{0}{#1}{#2}}

\newcommand{\Abs}[1]{\ensuremath{\left|#1\right|}}
\newcommand{\Wsum}[2][0]{\ensuremath{\sum_{n=#1}^\infty \frac{#2(0)}{(2\mu)^{n+1}}}}

\newcommand{\Oscint}[2]{\Integ{\mathrm O}{#1}{#2}}
\DeclareMathOperator*{\Reglim}{LIM}

\declareslashed{}{-}{0}{0}{\int}
\declareslashed{\mathop}{-}{-0.1}{-0.03}{\sum}
\newcommand{\Regsum}{\slashed{\sum}}
\newcommand{\Regint}[2]{\BaseInteg{\infty}{1}{#1}{#2}{\slashed{\int}}}

\newcommand{\Sphere}[1]{\ensuremath{{\mathbb{S}^{#1}}}}
\newcommand{\Cinf}[1][]{\ensuremath{C^\infty_{#1}}}
\newcommand{\Rplus}{\ensuremath{{\mathbb{R}_+}}}

\DeclareMathOperator{\Tr}{Tr}
\DeclareMathOperator{\vol}{vol}
\newcommand{\SimAs}[1]{\mathrel{\underset{#1}{\sim}}}
\newcommand{\SimMu}{\SimAs{\mu\to\infty}}

\let\Re=\relax
\let\Im=\relax
\DeclareMathOperator{\Re}{Re}
\DeclareMathOperator{\Im}{Im}
\renewcommand{\theta}{\vartheta}
\renewcommand{\phi}{\varphi}

\begin{document}
\begin{titlepage}
  \begin{center}
      \textsc{Diplomarbeit} \\
      \bigskip
      \textit{\Thema} \\
      \textit{(\ThemaEn)}
  \end{center}
  \vspace{\stretch{1}}
  \begin{center}
      Angefertigt am \\
      Mathematischen Institut
  \end{center}
  \vspace{\stretch{1}}
  \begin{center}
      Vorgelegt der \\
      Mathematisch-Naturwissenschaftlichen Fakultät der \\
      Rheinischen Friedrich-Wilhelms-Universität Bonn
  \end{center}
  \vspace{\stretch{1}}
  \begin{center}
      Dezember 2012 \\
      \bigskip
      Von \\
      \bigskip
      Benedikt Christian Sauer \\
      \bigskip
      geboren am \\
      30. März 1988 \\
      in \\
      Bonn\,--\,Bad~Godesberg
  \end{center}
\end{titlepage}
\newpage
\tableofcontents
\vfill
\begin{center}
  \large 1.\ Gutachter: Matthias Lesch \\
  \large 2.\ Gutachter: \\
\end{center}
\newpage
\selectlanguage{ngerman}
\subsection{Deutsche Einleitung}
% TODO: Special functions erwähnen
Diese Diplomarbeit beschäftigt sich mit der asymptotischen Entwicklung der Spur
multiparametrischer Sturm-Liouville-Operatoren auf dem abgeschlossenen Intervall
$[0,1]$.
%
Operatoren dieser Art erhält man beispielsweise bei der Untersuchung der
Resolvente des Laplace-Beltrami-Operators $\Delta$ auf einer Rotationsfläche
wenn dieser bezüglich der eindimensionalen Sphäre $\mathbb{S}^1$ in seine
Komponenten zerlegt wird, also als eine unendliche direkte Summe von Operatoren
$\Delta =
\bigoplus_{\lambda\in\operatorname{spec}\Delta_\Sphere{1}}\Delta_\lambda$ auf
der direkten Summe der Eigenräume ($\times[0,1]$) geschrieben wird (siehe
\cref{sec:laplace-beltrami}).
%
Die enstehenden Operatoren sind Sturm-Liouville-Operatoren mit einem Parameter
$\lambda$, lassen sich also für Funktionen $V,W\in\Cinf{[0,1]}$ mit $V > 0$
und $W > 0$ schreiben als
\begin{equation*}
  (\Delta_\lambda u)(x) = -u''(x) + \lambda^2 V(x) u(x) + W(x) u(x).
\end{equation*}
Für den Fall des Laplace-Beltrami-Operators auf einer von einer Funktion
$f\in\Cinf{[0,1]}$ mit $f>0$ erzeugten Rotationsfläche sind $V$ und $W$ gegeben
als
\begin{align*}
  V(x) &= \frac1{f(x)^2}, \\
  W(x) &= \frac{f''(x)}{2f(x)} - \left(\frac{f'(x)}{2f(x)}\right)^2.
\end{align*}
In \cite{LV13} wird gezeigt, dass sich die Zetadeterminante des
Laplace-Beltrami-Operators $\Delta$ auf einer Rotationsfläche, (formal) definiert
als
\begin{align*}
  \zeta(s,\Delta) := {} &
  \sum_{\mathclap{\mu\in\operatorname{spec}\Delta\setminus\{0\}}}
  m(\mu)\mu^{-s} \text{ für } \Re(s) > 1, \\
  \log\zdet \Delta = {} & {-\zeta'}(0,\Delta) \text{ durch meromorphe
Fortsetzung},
\end{align*}
in gewisser Weise als regularisierte Summe der Zetadeterminanten der Summanden
zuzüglich eines Fehlerterms schreiben lässt:
\begin{align*}
  \label[equation]{eqn:intro-de-lvformel}
  \log\zdet\Delta = \Regsum_{\mathclap{\lambda =
  -\infty}}^\infty\log\zdet\Delta_\lambda + C
\end{align*}
In diesem Kontext ist die regularisierte Summe $\Regsum$ über die
Hadamard-Regularisierung der Euler-MacLaurin-Formel
definiert (siehe \cref{sec:regsum-int}).

Das Ziel der vorliegenden Arbeit ist es, den in \cref{eqn:intro-de-lvformel} als
$C$ bezeichneten Fehlerterm zu berechnen. In \cite{LV13} wird gezeigt, dass der
Fehlerterm sich aus der asymptotischen Entwicklung von $\Tr(\Delta_\lambda +
z^2)^{-2}$ berechnen lässt, welche wiederum leicht auf die asymptotische
Entwicklung von $\Tr(\Delta_\lambda + z^2)^{-1}$ zurückzuführen ist. Der
Operator
\begin{align*}
  (A_{\lambda,z}u)(x) := (\Delta_\lambda u)(x) + z^2 u(x)
           = -u''(x) + (\lambda^2 V(x) + z^2) u(x) + W(x) u(x)
\end{align*}
ist nun ein \emph{multiparametrischer Sturm-Liouville-Operator}. Ist $h(\lambda,
z)$ der Term der Homogenitätsordnung $-5$ der gemeinsamen asymptotischen
Entwicklung in $(\lambda,z)$ der Spur $\Tr A_{\lambda,z}^{-2}$, so ist
$h(1,\,\cdot\,)\log(\,\cdot\,)\in L^1(0,\infty)$ und es gilt für den Fehlerterm:
\begin{align*}
  C = \Integ[\infty]{0}{\lambda}{h(1,\lambda)\log\lambda}
\end{align*}
Um diesen berechnen zu können wird das asymptotische Verhalten des
Resolventenkerns des Operators $k_{(\Delta_\lambda + z^2)^{-1}}$ sowohl im
Inneren, also auf dem Intervall $(0,1)$, als auch an den Rändern $0$ und $1$
untersucht.

Da wir hierfür die Parametrix (also $(\Delta_\lambda + z^2)^{-1}$ statt
$(\Delta_\lambda + z^2)^{-2}$) betrachten benötigen wir den Term der
Homogenitätsordnung $-3$, denn
\begin{equation*}
  \Tr(\Delta_\lambda + z^2)^{-2} = (-2z)^{-1}\partial_z\Tr(\Delta_\lambda +
  z^2)^{-1},
\end{equation*}
also ist die eine Entwicklung um $2$ gegenüber der anderen verschoben.

Im Inneren kann die Entwicklung mittels der Theorie parametrischer
Pseudodifferentialoperatoren, die auch die Existenz sichert, explizit rekursiv
bestimmt werden, wir erhalten daraus den 
\begin{Hauptsatz}[Resolvente im Inneren]
  % TODO: Existenz und explizite Formel (zumindest bis zum Grad 3).

\end{Hauptsatz}
Neben der expliziten Bestimmung der Koeffizienten können wir noch zeigen, dass
die Terme der Entwicklung nicht nur eine Homogenität in $(\lambda,z)$, sondern
auch in $V$, $W$ und deren Ableitungen aufweisen:
\begin{Hauptsatz}[Homogenität im Inneren]
  \iflanguage{ngerman}{
  Sei die Ordnung $\ord$ definiert als
}{
  Let the order $\ord$ be defined as
}
\begin{align*}
  \ord \lambda^2 V(x) := {} & 0 \\
            \ord W(x) := {} & 2 \\
 \ord \partial_x A(x) := {} & \ord A(x) + 1,
\end{align*}
\iflanguage{ngerman}{
  wobei $A(x)$ ein rationaler Ausdruck in $V$, $W$ und deren Ableitungen sei,
  für eine genaue Definition siehe \cref{def:order}.
}{
  where $A(x)$ is a rational expression in $V$, $W$ and their derivatives, for a
  precise definition cf.\ \cref{def:order}.
}

\iflanguage{ngerman}{
  Dann gilt die folgende Gleichung für die Terme $k_{-n}$ der
  Homogenitätsordnung $-n$ in der asymptotischen Entwicklung des Kerns von
  $(\Delta_\lambda + z^2)^{-1}$ auf der Diagonale:
}{
  Then the following holds for the terms $k_{-n}$ of homogeneity order $-n$ in
  the asymptotic expansion of the kernel of $(\Delta_\lambda +
  z^2)^{-1}$ on the diagonal:
}
\begin{equation*}
  \ord k_{-n}(x, x) = n - 1
\end{equation*} 

\end{Hauptsatz}
Am Rand wird der Operator zunächst mit einem Neumannreihenargument in
einfachere Operatoren zerlegt. Die Asymptotik dieser Operatoren lässt sich so nach
oben abschätzen, dass, um den Term $-5$ter Ordnung zu finden, letztlich die
Asymptotik von endlich vielen (um genau zu sein 3) dieser einfacheren Operatoren
bestimmt werden muss. Dazu wird die Asymptotik der Spur mithilfe der sehr
einfachen Asymptotik exponentieller Integrale ($F(x) =
\Integ[T]{0}{t}{\Eto{-tx}f(t)}$), bekannt als Watsons Lemma, bestimmt. Wir
erhalten daraus
\begin{Hauptsatz}[Resolventenspur am Rand]
  \iflanguage{ngerman}{
  Sei $\phi(x)\in\Cinf[0](\Rplus)$ eine Abschneidefunktion, deren Träger in einer
  genügend kleinen Umgebung von $x=0$ liegt. Dann hat die multiparametrische
  Resolventenspur des Sturm-Liouville-Operators
}{
  Let $\phi(x)\in\Cinf[0](\Rplus)$ be a cutoff function that is $0$ outside of a
  sufficiently small neighbourhood of $x=0$. Then the multiparametric
  trace-expansion of the resolvent of the Sturm-Liouville operator
}
\begin{equation*}
  \Delta_\lambda = -\partial_x^2 + \lambda^2 V(x) + W(x)
\end{equation*}
\iflanguage{ngerman}{
  auf $\Rplus$ bezüglich der verallgemeinerten Neumannrandbedingung
  $f(0)\cos\theta + f'(0)\sin\theta = 0$ die folgende asymptotische Entwicklung
  bis zur dritten nicht-verschwindenden Ordnung nahe $x=0$
}{
  on $\mathbb{R_+}$ up to the third non-vanishing order near $x=0$ with the
  generalised Neumann boundary conditions $f(0)\cos\theta + f'(0)\sin\theta = 0$
  is given by
}
\begin{equation*}
  \Tr\left(\phi(x)(\Delta_\lambda + z^2)^{-1}\right) \SimMu
  \frac{1}{(2\mu)^2} + \frac{5\lambda^2}{(2\mu)^5} V'(0) + O(\mu^{-4}),
\end{equation*}
\iflanguage{ngerman}{mit}{with} $\mu^2 := \lambda^2 V(0) + W(0)$.

\end{Hauptsatz}
Die Formel gilt mit geringen Abwandlungen auch für $x=1$. Hier ist die Existenz
der asymptotischen Entwicklung durch einen Satz aus \cite{LV13} gesichert.
Weiterhin beweisen wir im letzten Hauptsatz, dass die asymptotische
Entwicklung wie für die Resolventenspur im Inneren schon lange bekannt, auch am
Rand aus rationalen Funktionen in $V$, $W$ und deren Ableitungen besteht und die
Terme eine ähnliche Homogenität aufweisen:
\begin{Hauptsatz}[Homogenität am Rand]
  \iflanguage{ngerman}{
  Mit der selben Definition wie in Hauptsatz~2 gilt für einen Term $A$ der
  asymptotischen Entwicklung der Resolventenspur von $\Delta_\lambda$ am Rand
  mit der Homogenitätsordnung $-n$ in $(\lambda,z)$:
}{
  Using the same definition for $\ord$ as in Main~Theorem~2 we have for a term
  $A$ in the asymptotic expansion of the resolvent trace of $\Delta_\lambda$ at
  the boundary of homogeneity $-n$ in $(\lambda,z)$:
}
\begin{equation*}
  \ord A = n - 2
\end{equation*}

\end{Hauptsatz}

Unter Anwendung der beiden ersten Hauptsätze erhalten wir somit letztlich den
Fehlerterm in der Lesch-Vertman-Formel als Integral über eine rationale Funktion
in den Ableitungen von $f$ im Inneren und am Rand unter Anwendung eines
Integralsatzes aus der Theorie der speziellen Funktionen:
%TODO: Snippet
\begin{align*}
  \Integ[\infty]{0}{\lambda}{
    h(\lambda, 1) \log\lambda
  }
  &=
  % Interior:
  \Int{x}{\Biggl(\frac{(1 + 2 \log(2 V(x))) W(x)}{8 V(x)}
    - \frac{ (1 - 2 \log(2 V(x)))V''(x)}{96 V(x)^{3}} \\
    &\hphantom{=\int_0^1\Biggl(}\mathbin{+} \frac{\left(\frac{5}{3} - 2\log(2
    V(x))\right)\left(V'(x)\right)^{2}}{64 V(x)^{5}}\Biggr) } \\
&\mathbin- 5\frac{(1 - 2\log(2 V(1)))V'(1)}{192 V(1)^3}
+ 5\frac{(1 - 2 \log(2 V(0)))V'(0)}{192 V(0)^3}
\end{align*}
Zuletzt geben wir noch einen solchen Fehlerterm für $f(x) = x + \epsilon$ an und
betrachten den naiven Grenzwert $\epsilon \to 0$.

\selectlanguage{british}

% Main results:
% Asymptotische Entwicklung von Tr(\Delta_\lambda + z²)^{-1}
%            -/-                                       ^{-2}
% Application

\subsection{Conventions}
We follow the conventions used in \cite{Les:PDO}, hence the Fourier transform is
defined as
\begin{align*}
  \hat{u}(\xi) = (\mathcal{F}u)(\xi)
    &:= \Integ{\mathbb{R}}{x}{\Eto{-\mathrm ix\xi} u(x)} \\
  (\mathcal{F}^{-1}\hat{u})(x) &:=
  \frac{1}{2\pi}\Integ{\mathbb{R}}{\xi}{\Eto{\mathrm ix\xi}
    \hat{u}(\xi)}.
\end{align*}
We will always talk about symbols and only mention amplitudes when explicitly
needed. We thus denote the \emph{symbol} of an operator $A$ by $a$ instead of
$\sigma_A$. The kernel of an operator $A$ is denoted by $k_A$, i.e.\ 
\begin{align*}
  (Au)(x) =: \Integ{\mathbb{R}}{y}{k_A(x,y) u(x)},
\end{align*}
where $k_A$ may be a distribution.

\section{Setting}
% Multitude of references to ML and BV

\subsection{Integration Lemmas}
% TODO Umformulieren
During the course of this thesis we will see some recurring patterns in the
integrals that need to be solved. We will thus list some explicit formulas for
them now and prove the simpler ones here.

The first one is a classical result of asymptotic analysis and is the basis of
most further results on the asymptotics of exponential integrals.
\begin{Lemma}[Watson]
  Let $\phi\colon [0,1] \to \mathbb{C}, \phi(t) = t^\sigma g(t)$ such that $g$ is
analytic in some neighbourhood of $t=0$, $\sigma > -1$, $\beta > 0$, and
\begin{equation*}
    \exists_{C, b > 0} \forall_{t > 0} \left|\phi(t)\right| < C \Eto{bt}.
\end{equation*}
Then the exponential integral
\begin{align*}
    F(x) := \Integ[T]{0}{t}{\Eto{-\beta\,xt}\phi(t)}
\end{align*}
is finite for all $x \geq 0$ and has the asymptotic expansion in terms of the
gamma function $\Gamma$ (see Appendix~\ref{app:gamma})
\begin{equation*}
    F(x)\SimAs{x\to\infty}\sum_{n=0}^{\infty}
    \frac{g^{(n)}(0)}{(\beta x)^{n+\sigma+1}} \frac{\Gamma(n+\sigma+1)}{n!}
\end{equation*}

  \begin{proof}
    Since the proof is a bit lengthy and not needed to understand the following
    it can be found in Appendix~\ref{sec:proof-watson}.
  \end{proof}
%    \begin{Remark}
%        We actually don't need analyticity but only the existence of as many
%        derivatives as degrees we want and additionally a finiteness condition
%        on .
%    \end{Remark}
\end{Lemma}
The next lemma considers the continuity properties of exponential integrals like
the ones that are object of Watson's Lemma:
\begin{Lemma}
  \label{lem:continuity_of_watson_integrals}
  Let $f\colon [0,T)\to\mathbb{R}$ (where $T$ may be $\infty$) be an integrable
  function. Then the exponential parameter integral
  \begin{align*}
    F(x) = \Integ[T]{0}{t}{\Eto{-tx} f(t)}
  \end{align*}
  is continuous in $x$.
  \begin{Proof}
    The statement follows immediately from a classical result of Lebesgue
    integration theory on the continuity of parameter-dependent integrals
    (\cite[p282f]{Konigsberger2004}). We only need to check, that there is an
    integrable majorant for all $x$, which is simply $f(t)$ in this case, and
    that the whole integrand is continuous in $x$ for fixed $t$ which is
    obvious.
  \end{Proof}
\end{Lemma}
The third lemma is a quite general formula for nested integrals, in which the
upper limit of the inner integral is the integration variable of the outer one:
\begin{Lemma}
    \label{lem:triangle-integration}
    Let $f,g\colon[0,1]\to\mathbb{R}$ be continuous. Then the following formula
    holds:
    \begin{align*}
        \Integ[1]{0}{x}{g(x)\Integ[1]{x}{y}{f(y)}} =
        \Integ[1]{0}{y}{f(y)\Integ[y]{0}{x}{g(x)}}
    \end{align*}
    \begin{proof}
        Let $F(x) := \Integ[x]{0}{y}{f(y)}$ and $G(y) := \Integ[y]{0}{x}{g(x)}$.
        Then we see using partial integration:
        \begin{align*}
            \Integ[1]{0}{x}{g(x)\Integ[1]{x}{y}{f(y)}} &=
            \Integ[1]{0}{x}{g(x)F(1)} - \Integ[1]{0}{x}{g(x)F(x)} \\
            &= G(1)F(1) - \left(\Bigl.G(y)F(y)\Bigr|_0^1 -
            \Integ[1]{0}{y}{G(y)f(y)}\right) \\
            &= \Integ[1]{0}{y}{G(y)f(y)} =
            \Integ[1]{0}{y}{f(y)\Integ[y]{0}{x}{g(x)}},
        \end{align*}
        which proves the given formula.
    \end{proof}
    \begin{Remark}
      The special case $f\equiv g$ results in the formula
      \begin{align*}
        \Integ[1]{0}{x}{f(x)\Integ[1]{x}{y}{f(y)}} = 
        \Integ[1]{0}{y}{f(y)\Integ[y]{0}{x}{f(x)}} = 
        \frac12 F(1)^2.
      \end{align*}
    \end{Remark}
\end{Lemma}

\begin{Remark}
  This is in fact only Fubini's theorem for the integral of $(x,y)\to g(x)f(y)$
  over a right-angled triangle.
%  \begin{center}
%    \begin{tikzpicture}
%      \draw (0,0)
%      -- node[anchor=east] {$y$} (0,1)
%      -- node[anchor=south] {$x$} (1,1)
%      -- (0,0);
%    \end{tikzpicture}
%  \end{center}
\end{Remark}
Finally we give a formula for a particular kind of integral that comes up a lot
in the final calculations:
\begin{Lemma}
  \label{lem:beta_function_formula}
  Let $a, b > 0$ and $m > n$. Then
  \begin{align*}
    \Integ[\infty]{0}{x}{\frac{x^n}{(a^2 + b^2 x^2)^{m}}} =
    \frac{a^{n-2m+1}}{2b^{n+1}}B\bigl(\tfrac{n+1}{2}, \tfrac{2m - n -
    1}{2}\bigr),
  \end{align*}
  where $B$ denotes the beta function.
  \begin{Proof}
    \begin{align*}
      \Integ[\infty]{0}{x}{\frac{x^n}{(a^2 + b^2 x^2)^{m}}} 
      &= \frac{a^{n-2m+1}}{b^{n+1}} \Integ[\infty]{0}{y}{\frac{y^n}{(1 +
  y^2)^m}}
        &\text{with $y = bx / a$} \\
        &= \frac{a^{n - 2m + 1}}{2b^{n+1}}
        \Integ[\infty]{0}{z}{\frac{z^{\frac{n-1}{2}}}{(1 + z)^n}}
        &\text{with $z = \sqrt{y}$}\\
        &= \frac{a^{n-2m+1}}{2b^{n+1}}B\bigl(\tfrac{n+1}{2}, \tfrac{2m - n -
      1}{2}\bigr)
    \end{align*}
    The formula used in the last step,
    \begin{align*}
      B(a,b) = \Integ[\infty]{0}{z}{\frac{z^{a-1}}{(1+z)^{a+b}}},
    \end{align*}
    is proven in the Appendix~\ref{sec:beta_function}.
  \end{Proof}
\end{Lemma}

\section{Asymptotics of the Interior Parametrix}
In this section we aim to prove the following Theorem:
\begin{MainTheorem}
  Let $\Delta_\lambda := -\partial_x^2 + \lambda^2V(x)+W(x)$ be an operator as
  introduced beforehand and $\mu(x) := \sqrt{\lambda^2 V(x) + z^2}$ (thus
  $\mu(x)$ has the same order as $\lambda$ and $z$, since $V(x) > 0$ is given).
  Then the trace of the resolvent of $\Delta_\lambda$ has the following common
  asymptotic expansion in its parameter $\lambda$ and the resolvent parameter
  $z$:
  \begin{align*}
      \Tr(\Delta_\lambda + z^2)^{-1} \\ \SimAs{\Abs{(\lambda, z)}\to\infty} &
   \Int{x}{\left(
      \frac12\frac1{\mu(x)^{2}}
     -\frac1{16}\frac{\lambda^2 V''(x)}{\mu(x)^5}
     +\frac{5}{64}\frac{\lambda^4V'(x)^2}{\mu(x)^7}
     -\frac14\frac{W(x)}{\mu(x)^3}
     +O\left(\Abs{(\lambda, z)}^{-5}\right)
   \right)}

  \end{align*}
  \begin{Remark}
    For $V\equiv 0$ this expansion has already been proven in \cite{gelfand}
    (where $z^2 \mapsto \zeta$ and $W(x) \mapsto u(x)$) and calculated for even
    higher orders, but we will restrict ourselves in this text to what we need
    for the particular calculation we want to do in the end.  The
    Appendix~\ref{app:source-code} contains the source code for a simple Python
    program that calculates the asymptotic components (apart from the
    $x$-Integration) of the resolvent trace, as well as a list of the
    coefficients to higher order.
  \end{Remark}
\end{MainTheorem}
To do this we first note that the asymptotic expansion exists by a classical
theorem on the theory of pseudo-differential operators with paramater, namely
\begin{Theorem}
  Let $M$ be a closed $n$-dimensional smooth manifold, $E$ and $F$ smooth
  vector bundles over $M$, and $A\in\mathrm{CL}^m(M,E; F)$ (the set of
  classical pseudo-differential operators).

  If $m+n<0$ then for all $x\in M$ $A(x)$ is trace-class and $\Tr A(\cdot)\in
  \mathrm{CS}^{m+n}(\Gamma)$, with
  \begin{align*}
    \Tr A(x)\SimAs{\left|x\right|\to\infty}
    \sum_{j=0}^{\infty} e_{m-j}\bigl(\tfrac{x}{\left|x\right|}\bigr)
    \left|x\right|^{m+n-j}
  \end{align*}
  \begin{Proof}
    A proof can be found in \cite{Lesch10}.
  \end{Proof}
\end{Theorem}
Using that as well as the product formula for pseudo-differential operators, we
prove the explicit formula in $V$, $W$ and their derivatives.

\subsection{Calculation of the first coefficients}
The multi-parametric symbol of the operator $(\Delta_\lambda + z^2)$ is
\begin{align}
  a_2(x,\xi,\lambda,z) &= \xi^2 + \lambda^2 V(x) + z^2 \\
  a_0(x,\xi,\lambda,z) &= W(x),
  \label{eqn:symbol}
\end{align}
with all other $a_n \equiv 0$. To not complicate the following calculations we
suppress the continuous repetition of the parameters, a prime ($a_2'$) denotes
the derivative in $x$.

We use the following product formula for symbols: Let $A$ and $B$ be
pseudo-differential operators with symbols $a$ and $b$, respectively. Then the
complete symbol of $(\Delta_\lambda + z^2)B$ is (modulo $S^\infty$) given by
\begin{align}
  a * b := \sum_{n=0}^{\infty} \frac{1}{n!}\ \partial_\xi^n a\,D_x^n b.
  \label{eqn:product-formula}
\end{align}
A proof of this formula can be found in \cite[Thm 3.4]{Shu:POS}.

We are interested in the symbol of the parametrix, so we demand
\begin{align*}
  a * b \sim 1,
\end{align*}
so the asymptotic expansion of $a * b$ in the symbol spaces is supposed to be
$(a * b)_n \equiv \delta_{n0}$ if $B$ is a parametrix of $A$. Actually this will
also work for $b * a$, i.~e.\ $BA$, but this complicates further calculations as
with the chosen order we only have $\xi$-derivatives of $a_n$ which are
\begin{align}
  \partial_{\xi} a = 2\xi \text{ and } \partial_{\xi}^2 a = 2.
\end{align}
Thus the above sum collapses and only three terms are left:
\begin{align}
  a * b = (a_2 + a_0)b - 2\xi\mathrm i\,\partial_x b - \partial_x^2 b
\end{align}

From the known asymptotics of $a$ we can now inductively derive the first
coefficients of the parametrix of $(\Delta_\lambda + z^2)$:
\begin{align*}
  b_{-2} &= \frac{1}{a_2} \\
  b_{-3} &= -2\frac{\xi a_2'}{a_2^3}\\
  b_{-4} &= 4\frac{\xi^2 a_2''}{a_2^4}
  - 12\frac{\xi^2 (a_2')^2}{a_2^5} - \frac{a_0}{a_2^2}
  - 2\frac{a_2''}{a_2^3} + 2\frac{(a_2')^2}{a_2^4} \\
  \label{eqn:coeff-symbol}
\end{align*}

Since we are ultimately interested in the kernel on the diagonal we will need to
integrate the resulting symbol for the parametrix, since for a
pseudo-differential operator $A$ with symbol $a$ we have
\begin{align*}
  k_A(x,y) &= \frac{1}{2\pi}\Oscint{\xi}{
    \Eto{\mathrm i\left\langle x-y, \xi\right\rangle} a(x,\xi)
  }
  \quad
  \Rightarrow
  \quad
  k(x,x) = \frac{1}{2\pi}\Integ{\mathbb{R}}{\xi}{a(x,\xi)},
\end{align*}
where $\Oscint{\xi}\cdot$ denotes the oscillating integral (as defined in
\cite[]{Shu:POS}), which coincides with the usual integral over $\mathbb{R}$ for
$x = y$. The integrals over $\xi$ can be handled with in a uniform way using the
beta function as seen in the introduction (\ref{lem:beta_function_formula}) by
setting $n = \text{order of $\xi$ in the numerator}$, $m = \tfrac12\text{order
of $a_2$ in the denominator}$, $a=\mu(x)=\sqrt{\lambda^2 V(x) + W(x)}$ and $b =
1$. We also see, that for odd $n$ the terms vanish, as we have an integral over
an odd function, while for even $n$ we get an additional factor of $2$ for
integrating over $\mathbb{R}$ instead of $\Rplus$.

The corresponding terms in the kernel are thus
\begin{align}
  k_{b_{-1}}(x,x,\lambda,z) &= \frac{1}{2} \frac{1}{\lambda^2 V(x) + z^2} \\
  k_{b_{-2}}(x,x,\lambda,z) &= 0 \\
  k_{b_{-3}}(x,x,\lambda,z) &= - \frac{1}{16} \frac{\lambda^2 V''(x)}{(\lambda^2
    V(x) + z^2)^{5/2}} + \frac{5}{64} \frac{\lambda^4 V'(x)^2}{(\lambda^2V(x)
    + z^2)^{7/2}} -\frac{1}{4}\frac{W(x)}{(\lambda^2 V(x) + z^2)^{3/2}}.
  \label{eqn:coeff-kernel}
\end{align}

These are the first three asymptotic components of the kernel of
$(\Delta_\lambda + z^2)^{-1}$ on the open interval $(0,1)$, i.e.\ if we define
$\mu(x) := \sqrt{\lambda^2 V(x) + z^2}$ as above we have
\begin{align*}
    \Tr(\Delta_\lambda + z^2)^{-1} \\ \SimAs{\Abs{(\lambda, z)}\to\infty} &
   \Int{x}{\left(
      \frac12\frac1{\mu(x)^{2}}
     -\frac1{16}\frac{\lambda^2 V''(x)}{\mu(x)^5}
     +\frac{5}{64}\frac{\lambda^4V'(x)^2}{\mu(x)^7}
     -\frac14\frac{W(x)}{\mu(x)^3}
     +O\left(\Abs{(\lambda, z)}^{-5}\right)
   \right)}
.
\end{align*}
% TODO: Oben allgemein angeben, folgt ja aus der Rekursionsformel
% Was heißt das?!
The asymptotic order of the remainder of $-5$ follows from the fact, that by
construction every other step in our recursion formula results in the exponents
of $\xi$ in the numerator being odd, so the $\xi$ integral over $\mathbb{R}$
vanishes, in particular $k_{b_{-4}} = 0$.

Finally for this to give us all the information we were looking for we need to
carry out the $x$-Integration. Evaluating this integral will be postponed until
we have done the necessary formal steps to get to $\Tr(\Delta_\lambda +
z^2)^{-2}$, and evaluate the $(\lambda,z)$-integral. Since by
Equation~\eqref{frm:shift} and Equation~\eqref{eqn:central-theorem} we know,
that we will only differentiate and integrate with a $\ln\lambda$-term, and all
intermediate steps are integrable we can indeed postpone the evaluation until we
have enough information on $V$ and $W$ to perform the integration.

\section{Asymptotics of the exterior parametrix}
By using the summation formula derived in LV12 the existence and values of the
asymptotic components of the specific operator we're looking at can be
calculated directly. From Lemma bla we know, that we only have to consider $j\in
{0, 1, 2}$ since the requested asymptotic coefficient is of order $-5$ % TODO

The calculation for $j=0$ is straightforward:
\begin{align}
    \Tr R^{(0)} =& \Integ[1]{0}{x}{k_\theta(x,x;\mu) - k_\mathbb{R}(x,x;\mu)} \\
    =& -\frac1{2\mu} C(\mu,\theta) \Integ[1]{0}{x}{\Eto{-2\mu x}} \\
%    =& -\frac1{2\mu} C(\mu,\theta) \left( \frac1{2\mu} - \frac{\Eto{-2\mu}}{2\mu} \right) \\
    \SimAs{\mu\to\infty}& -\frac1{4\mu^2}
    \label{eqn:jeq0}
\end{align}

For $j=1$ and $j=2$ we employ the following Lemma:
\begin{Lemma}[Watson]
    Let $\phi\colon [0,1] \to \mathbb{C}, \phi(t) = t^\sigma g(t)$ such that $g$ is
analytic in some neighbourhood of $t=0$, $\sigma > -1$, $\beta > 0$, and
\begin{equation*}
    \exists_{C, b > 0} \forall_{t > 0} \left|\phi(t)\right| < C \Eto{bt}.
\end{equation*}
Then the exponential integral
\begin{align*}
    F(x) := \Integ[T]{0}{t}{\Eto{-\beta\,xt}\phi(t)}
\end{align*}
is finite for all $x \geq 0$ and has the asymptotic expansion in terms of the
gamma function $\Gamma$ (see Appendix~\ref{app:gamma})
\begin{equation*}
    F(x)\SimAs{x\to\infty}\sum_{n=0}^{\infty}
    \frac{g^{(n)}(0)}{(\beta x)^{n+\sigma+1}} \frac{\Gamma(n+\sigma+1)}{n!}
\end{equation*}

    \begin{proof}
        See Appendix~\ref{sec:proof-watson}.
    \end{proof}
    \begin{Remark}
        We actually don't need analyticity but only the existence of as many
        derivatives as degrees we want and additionally a finiteness condition
        on
    \end{Remark}
\end{Lemma}

% j = 1, ++
% TODO: 1/2µ einbauen
The first term for $j=1$ is as follows
\begin{align*}
    \Tr R^{(1)}_{++} &= C(\mu,\theta)^2
        \Integ[1]{0}{x}{
            \Integ[1]{0}{z}{\Eto{-\mu(x+z)}
                \lambda(z)\Eto{-\mu(z+x)}
            }
        }
        \\
        &= C(\mu,\theta)^2
            \Integ[1]{0}{x}{
                \Eto{-2\mu x}
            }
            \Integ[1]{0}{z}{\Eto{-2\mu z}\lambda(z)}
        \\
        &= C(\mu, \theta)^2 \left(
            \frac1{2\mu} - \frac{\Eto{-2\mu}}{2\mu}
           \right)
           \Integ[1]{0}{z}{\Eto{-2\mu z}\lambda(z)}
        \\
        &\SimAs{\mu\to\infty}
            \frac{(-1)^2}{2\mu}
            \sum_{n=0}^{\infty}\frac{\lambda^{(n)}(0)}{(2\mu)^{n+1}}
\end{align*}

% j = 1, +-
The second term for $j=1$ is
\begin{align*}
    \Tr R^{(1)}_{+-} &= -C(\mu,\theta)
        \Integ[1]{0}{x}{
            \Integ[1]{0}{z}{\Eto{-\mu(x+z)}
                \lambda(z)\Eto{-\mu\left|z-x\right|}
            }
        } \\
        &= -C(\mu,\theta)
            \Integ[1]{0}{x}{\left(
                \Integ[x]{0}{z}{
                    \Eto{-\mu(x+z+x-z)} \lambda(z)
                }
                +
                \Integ[1]{x}{z}{
                    \Eto{-\mu(x+z-x+z)} \lambda(z)
                }
                \right)
            } \\
        &= -C(\mu,\theta) \left(
            \Integ[1]{0}{x}{\Eto{-2\mu x} \Lambda(x)}
            + \Integ[1]{0}{x}{\Eto{-2\mu x}\lambda(x)x}
            \right) \\
        &\SimAs{\mu\to\infty}\ 
            \sum_{n=0}^\infty \frac{\Lambda^{(n)}(0)}{(2\mu)^{n+1}}
            + \sum_{n=0}^\infty \frac{n\lambda^{(n-1)}(0)}{(2\mu)^{n+1}} \\
        &= \sum_{n=0}^\infty \frac{(n+1)\lambda^{(n-1)}}{(2\mu)^{n+1}}
\end{align*}
% TODO: Beweis, dass Int_0^1 Int_y^1 f(x) dx dy = Int_0^1 x f(x) dx
% TODO: (x*f(x))^(n) = x*f^(n) + n*f^(n-1)
where $\Lambda(x) := \Integ[x]{0}{z}{\lambda(z)}$.

\section{Asymptotics of $\Tr{(\Delta_\lambda + z^2)^{-2}}$}
By formal derivation we see that
\begin{align*}
    \partial_z\Tr{(\Delta_\lambda + z^2)}^{-1} &=
        (2z) \Tr{(\Delta_\lambda + z^2)}^{-2}
        .
\end{align*}
Since we saw in the previous sections that
\begin{align*}
    \Tr{(\Delta_\lambda + z^2)}^{-1}
        \ \SimAs{\left|(z,\lambda)\right|\to\infty}\ 
\end{align*}

\section{Compilation}
Now that we have calculated the coefficients we only need to integrate the
homogeneous part with the logarithm. To do that we use the Mellin transform,
since
\begin{align*}
    \left.\frac\partial{\partial z}\right|_{z=0}\Integ[\infty]{0}{x}{x^z f(x)}
    &= \Integ[\infty]{0}{x}{\ln{(x)} f(x)}
\end{align*}
if one of the integrals exist. We thus calculate:
\begin{align*}
    \Integ[\infty]{0}{\lambda}{\frac{\lambda^{n + z}}{\left(1 +
    \lambda^2\right)^{\frac{n+5}{2}}}} &=
    % Substituiere \lambda^2 = y
    \frac12\Integ[\infty]{0}{y}{\frac{y^{\frac{n+z+1}{2}}}{(1 +
    y)^{\frac{n+5}{2}}}} \\
    % Betafunktion
    &= \frac12 B\bigl(\tfrac{n+z+1}{2}, \tfrac{4-z}{2}\bigr)
\end{align*}
where $B(a,b)$ is the beta function. % Benutztes Integral mit ref!
% a = n+z+1/2, b = n+5/2 - a

The function is symmetric, and its
derivative is $\partial_a B(a,b) = B(a,b) (\psi(a) - \psi(a+b))$, where
$\psi(x) = \partial_x \ln\Gamma(x)$ is the digamma function, so we get
\begin{align*}
    \left.\frac12\frac\partial{\partial z}\right|_{z=0} B\bigl(\tfrac{n+z+1}{2},
    \tfrac{4-z}{2}\bigr)
    &= \left.\frac14 B\bigl(\tfrac{n+z+1}{2}, \tfrac{4-z}{2}\bigr)
    \left(\psi\bigl(\tfrac{n+z+1}{2}\bigr) - \psi\bigl(\tfrac{n+5}{2}\bigr) -
    \psi\bigl(\tfrac{4-z}{2}\bigr) +
    \psi\bigl(\tfrac{n+5}{2}\bigr)\right)\right|_{z=0} \\
    &= \frac14 B\bigl(\tfrac{n+1}{2}, 2\bigr)
    \left(\psi\bigl(\tfrac{n+1}{2}\bigr) - \psi\bigl(2\bigr)\right)
\end{align*}

$B\bigl(\tfrac{n+1}{2}, 2\bigr)$ can be evaluated using the formula given above
to be $\tfrac{4}{(n+1)(n+3)}$. Using the recurrence relation $\psi(x + 1) =
\psi(x) + \tfrac{1}{x}$ and the special values $\psi(1) = -\gamma$ and
$\psi(\tfrac{1}{2}) = -\gamma - 2\ln 2$ ($\gamma$ being the Euler-Mascheroni
constant) we arrive at the following explicit formula for the integral:
\begin{align*}
    \Integ[\infty]{0}{\lambda}{\frac{\lambda^n}{(1 +
    \lambda^2)^{\frac{n+5}{2}}}\ln{\lambda}} = \frac{1}{(n+1)(n+3)}
        \begin{cases} 
            -1 - 2\ln 2 + \sum_{k=1}^{\frac n 2}\frac{2}{2k - 1} & n\text{
            even} \\
            -1 + \sum_{k=1}^{\frac{n-1}{2}}\frac 1 k & n\text{ odd}
        \end{cases}
\end{align*}

\subsection{Examples}
Finally we want to actually make use of our fine formula and evaluate it for
some selected test functions. To do that we first calculate $V$ and $W$ from the
given $f$ and then evaluate the $x$-integral left in the formula of the error
term.

\subsubsection{Linear function}
The first canonical example we can use is a linear function
\begin{align}
  f(x) &:= mx + \epsilon \\
  V(x) &= (mx + \epsilon)^{-2} \\
  W(x) &= -\frac m 2 V(x)
\end{align}

% TODO: Auswerten wenn die finale Formel da ist!

\section{Summary}
We have given methods and formulas to calculate the resolvent-trace asymptotics
for Sturm-Lioville operators on a closed interval, where our results for the
interior are in line with the prior work in \cite[Gel'fand1975]. The methods for
the evaluation of the boundary asymptotics could be expanded, especially the
proof of the homogeneity can potentially be used for an algorithmical approach
on the problem.

\appendix
\section{Special Functions}
\subsection{Gamma Function}
\subsection{Digamma Function}
\subsection{Beta Function}

\section{Proof of Watson's Lemma}
\label{sec:proof-watson}
\begin{Theorem}
  Let $\phi\colon [0,1] \to \mathbb{C}, \phi(t) = t^\sigma g(t)$ such that $g$ is
analytic in some neighbourhood of $t=0$, $\sigma > -1$, $\beta > 0$, and
\begin{equation*}
    \exists_{C, b > 0} \forall_{t > 0} \left|\phi(t)\right| < C \Eto{bt}.
\end{equation*}
Then the exponential integral
\begin{align*}
    F(x) := \Integ[T]{0}{t}{\Eto{-\beta\,xt}\phi(t)}
\end{align*}
is finite for all $x \geq 0$ and has the asymptotic expansion in terms of the
gamma function $\Gamma$ (see Appendix~\ref{app:gamma})
\begin{equation*}
    F(x)\SimAs{x\to\infty}\sum_{n=0}^{\infty}
    \frac{g^{(n)}(0)}{(\beta x)^{n+\sigma+1}} \frac{\Gamma(n+\sigma+1)}{n!}
\end{equation*}

  \begin{Proof}
    The proof due to \cite{Miller2006} uses the Taylor series expansion of
    $g(t)$ with a remainder term in the integrand.

    Since we only assume differentiability in some neighbourhood of $t=0$ we
    need to estimate the integral for some $0 < s < T$, with $s$ being
    sufficiently small such that $g\in C^{\infty}[0,s]$ is analytic. We
    can now split up the integral $F(x)$ as
    \begin{equation*}
      F(x) = \Integ[s]{0}{t}{\Eto{-xt}\phi(t)} +
      \Integ[T]{s}{t}{\Eto{-xt}\phi(t)}.
    \end{equation*}

    The integral over $\left[s,T\right]$ can be easily estimated as
    \begin{equation*}
      \Abs{\Integ[T]{s}{t}{\Eto{-xt}\phi(t)}}
      \le \Eto{-xs} \Integ[T]{s}{t}{\Abs{\phi(t)}}
      \le \Eto{-xs}\Integ[T]{0}{t}{\Abs{\phi(t)}}.
    \end{equation*}
    Since we assumed this integral to be finite and $s>0$ we have for
    $x\to\infty$
    \begin{equation*}
      \Integ[T]st{\Eto{-xt}\phi(t)} = O(x^{-\infty}),
    \end{equation*}
    so we see that this term doesn't contribute to the asymptotic expansion.
    For the first term, the integral over $[0,s]$ we use the taylor expansion of
    $g$ with remainder
    \begin{equation*}
      g(t) = \sum_{n=0}^{N} \frac{g^{(n)}(0)}{n!}t^n + r_N(t).
    \end{equation*}
    We thus have
    \begin{equation*}
      \Integ[s]0t{\Eto{-xt}t^\sigma g(t)} = \sum_{n=0}^N
      \frac{g^{(n)}(0)}{n!} \Integ[s]0t{\Eto{-xt}t^{\sigma+n}} +
      \Integ[s]0t{\Eto{-xt}t^\sigma r_N(t)},
    \end{equation*}
    where we can estimate the remainder term as
    \begin{align*}
      \left|\Integ[s]0t{\Eto{-xt}t^\sigma r_N(t)}\right| &\le
      \Integ[s]0t{\Eto{-xt}t^\sigma\left|r_N(t)\right|} \\
      &\le \sup_{\tau\in\left[0,s\right]}
      \frac{\bigl|g^{(N+1)}(\tau)\bigr|}{(N+1)!}\Integ[s]0t{\Eto{-xt}t^{\sigma+N+1}}.
    \end{align*}
    It thus suffices to estimate the asymptotic behaviour of integrals of
    the type $F_\alpha(x) := \Integ[s]0t{\Eto{-xt}t^\alpha}$ for
    $x\to\infty$. We have
    \begin{align*}
      F_\alpha(x) &= \Integ[\infty]{0}{t}{\Eto{-xt}t^\alpha} -
      \Integ[\infty]{s}{t}{\Eto{-xt}t^\alpha} \\
      &= x^{-(\alpha+1)}\Integ[\infty]{s}{\tau}{\Eto{-\tau}\tau^\alpha} -
      \Integ[\infty]{s}{t}{\Eto{-xt}t^\alpha} \\
      &= x^{-(\alpha+1)}\Gamma(\alpha + 1) - 
      \Integ[\infty]{s}{t}{\Eto{-xt}t^\alpha}.
    \end{align*}
    For the second term we can use the Hölder equation in the special case
    $p=q=2$, where we split the exponential as $\Eto{-xt} =
    \Eto{-xt/2}\Eto{-xt/2}$, to get
    \begin{equation*}
      \Integ[\infty]{s}{t}{\Eto{-xt}t^\alpha} \le
      \sqrt{\Integ[\infty]{s}{t}{\Eto{-xt}}} \cdot
      \sqrt{\Integ[\infty]{s}{t}{\Eto{-xt}t^{2\alpha}}}
      = \Eto{-xs/2} \sqrt{\Integ[\infty]{s}{t}{\Eto{-xt}t^{2\alpha}}}.
    \end{equation*}
    The second term in the product decreases with $x$, and since we are
    interested in the case $x\to\infty$ we can assume $x>1$ to estimate the
    square of the second factor as
    \begin{equation*}
      \Integ[\infty]{s}{t}{\Eto{-xt}t^{2\alpha}} \underset{x>1}\le
      \Integ[\infty]{s}{t}{\Eto{-t}t^{2\alpha}} =: C_{s,\alpha}
    \end{equation*}
    and thus since $s>0$ we have the following asymptotic behaviour for
    $x\to\infty$:
    \begin{equation*}
      \Integ[\infty]{s}{t}{\Eto{-xt}t^\alpha} \le \Eto{-xs/2}
      \sqrt{\Integ[\infty]{s}{t}{\Eto{-t}t^{2\alpha}}}
      \le C_{s,\alpha}\,\Eto{-xs/2} = O(x^{-\infty}),
    \end{equation*}
    thus this term doesn't contribute to the asymptotic expansion. Using this we
    have
    \begin{equation*}
      F_\alpha(x) = x^{-(\alpha+1)}\,\Gamma(\alpha + 1) + O(x^{-\infty}),
    \end{equation*}
    and inserting $\alpha = \sigma + N + 1$ for the remainder and $\alpha =
    \sigma + n$ for the main term we arrive at
    \begin{align*}
      F(x) &= \sum_{n=0}^N \frac{g^{(n)}(0)}{n!} F_{\sigma+n}
            + F_{\sigma + N + 1} \\
           &= \sum_{n=0}^N \frac{g^{(n)}(0)}{n!}
            \Gamma(\sigma + n +1)\,x^{-(\sigma + n + 1)} + \Gamma(\sigma +
              N + 2)\,x^{-(\sigma + N + 2)} \\
           &= \sum_{n=0}^N \frac{g^{(n)}(0)}{n!}\Gamma(\sigma + n +
            1)\,x^{-(\sigma + n + 1)} + O\bigl(x^{-(\sigma + N + 2)}\bigr).
    \end{align*}
    This proves the statement.
  \end{Proof}
  \begin{Remark}
    A different, more general approach on this is taken in \cite[Ex.~53,
    p.104]{Estrada1993}, where the Lemma is proven by use of the expansion of
    $f(\lambda x)$ for $f(x) = H(x)\Eto{-x}$, $H$ being the Heaviside function.
  \end{Remark}
\end{Theorem}

\newpage
\selectlanguage{ngerman}
\pagestyle{plain}
\begin{center}
    \textbf{\large Eidesstattliche Erklärung}
\end{center}
\vspace{2cm}
Hiermit erkläre ich, Benedikt Christian Sauer, an Eides statt, dass ich die
Diplomarbeit \textit{"`\Thema"'} selbstständig verfasst und keine anderen als
die angegebenen Hilfsmittel benutzt sowie Zitate kenntlich gemacht habe. \\
\vspace{2cm} \\
Bonn, den \today
\end{document}
