\section{Asymptotics for the interior parametrix}
In this section we aim to prove the following Theorem:
% TODO: Existenz und explizite Formel (zumindest bis zum Grad 3).

To do this we first show that the asymptotic expansion exists and, using that as
well as the product formula for pseudo-differential operators, prove the
explicit formula in $V$, $W$ and their derivatives.
\subsection{Existence}
By the following classical theorem an elliptic pseudo-differential operator with
parameter has an asymptotic expansion in its parameter:
\begin{Theorem}
    Let $M$ be a closed $n$-dimensional smooth manifold, $E$ and $F$ smooth
    vector bundles over $M$, and $A\in\mathrm{CL}^m(M,E; F)$.

    If $m+n<0$ then for all $x\in M$ $A(x)$ is trace-class and $\Tr A(\cdot)\in
    \mathrm{CS}^{m+n}(\Gamma)$, with
    \begin{align*}
        \Tr A(x)\SimAs{\left|x\right|\to\infty}
        \sum_{j=0}^{\infty} e_{m-j}\!\left(\frac{x}{\left|x\right|}\right)
                            \left|x\right|^{m+n-j}
    \end{align*}
\end{Theorem}

\begin{proof}
\end{proof}

\subsection{Calculation of the first coefficients}
The multi-parametric symbol of the operator $(\Delta_\lambda + z^2)$ is
\begin{align}
  % TODO: Array
    a_2(x,\xi,\lambda,z) &= \xi^2 + \lambda^2 V(x) + z^2 \\
    a_0(x,\xi,\lambda,z) &= W(x),
    \label{eqn:symbol}
\end{align}
with all other $a_n \equiv 0$.

% TODO REF: Shubin Thm 3.4, p 27
We use the following product formula for symbols. Let $B$ be a
pseudo-differential operator with symbol $a$ and $b$. Then the complete symbol
of $(\Delta_\lambda + z^2)B$ is (modulo $S^\infty$)
\begin{align}
  a * b := \sum_{n=0}^{\infty} \frac{1}{n!}\ \partial_\xi^n a\,D_x^n b
  \label{eqn:product-formula}
\end{align}
We are interested in the symbol of the parametrix, so we demand
\begin{align*}
  a * b \sim 1,
\end{align*}
so the asymptotic expansion of $(a*b)$ in the symbol spaces is supposed to be
$(a * b)_n \equiv \delta_{n0}$. Actually this will also work for $(b * a)$,
i.~e.\ $BA$, but this complicates further calculations as with the chosen order
we only have $\xi$-derivatives of $a_n$ which are
\begin{align}
  \partial_{\xi}^1 a = 2\xi \text{ and } \partial_{\xi}^2 a = 2.
\end{align}
Thus the above sum collapses and only three terms are left:
\begin{align}
  a * b = ab + 2\xi\,D_x b + 2D_x^2 b.
\end{align}

From the known asymptotics of $a$ we can now inductively derive the first
coefficients of the parametrix of $(\Delta_\lambda + z^2)$.
\begin{align}
    b_{-2}(x,\xi,\lambda,z) &= (a_2(x,\xi,\lambda,z))^{-1} \\
    b_{-3}(x,\xi,\lambda,z) &= \\
    b_{-4}(x,\xi,\lambda,z) &=
    \label{eqn:coeff-symbol}
\end{align}

% This variant is chosen deliberately, since we thus have to calculate fewer
% $x$-derivatives.

Since we are ultimately interested in the kernel on the diagonal we will need to
integrate the resulting symbol for the parametrix, since
\begin{align*}
    k(x,y) &= \frac{1}{2\pi}\Integ{\mathrm{O}}{\xi}{
        \Eto{\mathrm i\left\langle x-y, \xi\right\rangle} a(x,\xi)
        } \\
    \Rightarrow k(x,x) &= \frac{1}{2\pi}\Integ{\mathbb{R}}{\xi}{a(x,\xi)},
\end{align*}
where $\int_{O}$ denotes the oscillating integral (REF), which coincides with
the usual integral over $\mathbb{R}$ for $x = y$.

The corresponding terms in the kernel are thus
\begin{align}
    k_{b_{-1}}(x,x,\lambda,z) &= \\
    k_{b_{-2}}(x,x,\lambda,z) &= \\
    k_{b_{-3}}(x,x,\lambda,z) &= \\
    \label{eqn:coeff-kernel}
\end{align}

To now get the coefficients of the kernel of $(\Delta_\lambda + z^2)^{-2}$ we
formally derive by $z$


