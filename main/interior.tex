\section{Asymptotics of the interior parametrix}
In this section we aim to prove the following Theorem:
\begin{MainTheorem}
  Let $\Delta_\lambda := -\partial_x^2 + \lambda^2V(x) $ be an operator as
  introduced beforehand and $\mu(x) := \sqrt{\lambda^2 V(x) + z^2}$ (thus
  $\mu(x)$ has the same order as $\lambda$ and $z$, since $V(x) > 0$ is given).
  Then the trace of the resolvent of $\Delta_\lambda$ has the following common
  asymptotic expansion in its parameter $\lambda$ and the resolvent parameter
  $z$:
  \begin{align*}
  \Tr(\Delta_\lambda + z^2)^{-1} \SimAs{\Abs{(\lambda, z)}\to\infty}&
   \Int{x}{\left(
      \frac12\frac1{\mu(x)^{2}}
     -\frac1{16}\frac{\lambda^2 V''(x)}{\mu(x)^5}
     +\frac{5}{64}\frac{\lambda^4V'(x)^2}{\mu(x)^7}
     -\frac14\frac{W(x)}{\mu(x)^3}
     +O\left(\Abs{(\lambda, z)}^{-5}\right)
   \right)}.
  \end{align*}
  \begin{Remark}
    % TODO Zusätzliche Ordnung
    For $V\equiv 0$ this expansion has already been proven in (DIKII, Gel'fand)
    (where $z^2 \mapsto \zeta$ and $W(x) \mapsto u(x)$) and calculated for even
    higher orders, but we will restrict us in this text to (one more than) what
    we need for the particular calculation we want to do in the end.  The
    Appendix (ref) contains the source code for a simple Python program that
    calculates the asymptotic components (apart from the $x$-Integration) of the
    resolvent trace, as well as a list of the coefficients to higher order.
  \end{Remark}
\end{MainTheorem}
To do this we first show that the asymptotic expansion exists and, using that as
well as the product formula for pseudo-differential operators, prove the
explicit formula in $V$, $W$ and their derivatives.
\subsection{Existence}
By the following classical theorem an elliptic pseudo-differential operator with
parameter has an asymptotic expansion in its parameter:
\begin{Theorem}
    Let $M$ be a closed $n$-dimensional smooth manifold, $E$ and $F$ smooth
    vector bundles over $M$, and $A\in\mathrm{CL}^m(M,E; F)$.

    If $m+n<0$ then for all $x\in M$ $A(x)$ is trace-class and $\Tr A(\cdot)\in
    \mathrm{CS}^{m+n}(\Gamma)$, with
    \begin{align*}
        \Tr A(x)\SimAs{\left|x\right|\to\infty}
        \sum_{j=0}^{\infty} e_{m-j}\!\left(\frac{x}{\left|x\right|}\right)
                            \left|x\right|^{m+n-j}
    \end{align*}
\end{Theorem}

\begin{Proof}
  % TODO Gehört das hier rein oder reicht ein Verweis?
\end{Proof}

\subsection{Calculation of the first coefficients}
The multi-parametric symbol of the operator $(\Delta_\lambda + z^2)$ is
\begin{align}
  a_2(x,\xi,\lambda,z) &= \xi^2 + \lambda^2 V(x) + z^2 \\
  a_0(x,\xi,\lambda,z) &= W(x),
  \label{eqn:symbol}
\end{align}
with all other $a_n \equiv 0$. To not complicate the following calculations we
suppress the continuous repetition of the parameters, a prime ($a_2'$) denotes
the derivative in $x$.

% TODO REF: Shubin Thm 3.4, p 27
We use the following product formula for symbols: Let $A$ and $B$ be
pseudo-differential operators with symbols $a$ and $b$, respectively. Then the
complete symbol of $(\Delta_\lambda + z^2)B$ is (modulo $S^\infty$) given by
\begin{align}
  a * b := \sum_{n=0}^{\infty} \frac{1}{n!}\ \partial_\xi^n a\,D_x^n b
  \label{eqn:product-formula}
\end{align}
We are interested in the symbol of the parametrix, so we demand
\begin{align*}
  a * b \sim 1,
\end{align*}
so the asymptotic expansion of $a * b$ in the symbol spaces is supposed to be
$(a * b)_n \equiv \delta_{n0}$ if $B$ is a parametrix of $A$. Actually this will
also work for $b * a$, i.~e.\ $BA$, but this complicates further calculations as
with the chosen order we only have $\xi$-derivatives of $a_n$ which are
\begin{align}
  \partial_{\xi} a = 2\xi \text{ and } \partial_{\xi}^2 a = 2.
\end{align}
Thus the above sum collapses and only three terms are left:
\begin{align}
  a * b = (a_2 + a_0)b - 2\xi\mathrm i\,\partial_x b - \partial_x^2 b.
\end{align}

From the known asymptotics of $a$ we can now inductively derive the first
coefficients of the parametrix of $(\Delta_\lambda + z^2)$:
\begin{align*}
  b_{-2} &= \frac{1}{a_2} \\
  b_{-3} &= -2\frac{\xi a_2'}{a_2^3}\\
  b_{-4} &= 4\frac{\xi^2 a_2''}{a_2^4}
  - 12\frac{\xi^2 (a_2')^2}{a_2^5} - \frac{a_0}{a_2^2}
  - 2\frac{a_2''}{a_2^3} + 2\frac{(a_2')^2}{a_2^4} \\
  \label{eqn:coeff-symbol}
\end{align*}

Since we are ultimately interested in the kernel on the diagonal we will need to
integrate the resulting symbol for the parametrix, since for a
pseudo-differential operator $A$ with symbol $a$ we have
\begin{align*}
  k_A(x,y) &= \frac{1}{2\pi}\Oscint{\xi}{
    \Eto{\mathrm i\left\langle x-y, \xi\right\rangle} a(x,\xi)
  }
  \quad
  \Rightarrow
  \quad
  k(x,x) = \frac{1}{2\pi}\Integ{\mathbb{R}}{\xi}{a(x,\xi)},
\end{align*}
where $\Oscint{\xi}\cdot$ denotes the oscillating integral (REF), which
coincides with the usual integral over $\mathbb{R}$ for $x = y$. The integrals
over $\xi$ can be handled with in a uniform way using the beta function as seen
in the introduction (\ref{lem:beta_function_formula}) by setting $n =
\text{order of $\xi$ in the numerator}$, $m = \tfrac12\text{order of $a_2$ in
the denominator}$, $a=\mu(x)=\sqrt{\lambda^2 V(x) + W(x)}$ and $b = 1$. We also
see, that for odd $n$ the terms vanish, as we have an integral over an odd
function, while for even $n$ we get an additional factor of $2$ for integrating
over $\mathbb{R}$ instead of $\Rplus$.

The corresponding terms in the kernel are thus
\begin{align}
  k_{b_{-1}}(x,x,\lambda,z) &= \frac{1}{2} \frac{1}{\lambda^2 V(x) + z^2} \\
  k_{b_{-2}}(x,x,\lambda,z) &= 0 \\
  k_{b_{-3}}(x,x,\lambda,z) &= - \frac{1}{16} \frac{\lambda^2 V''(x)}{(\lambda^2
    V(x) + z^2)^{5/2}} + \frac{5}{64} \frac{\lambda^4 V'(x)^2}{(\lambda^2V(x)
    + z^2)^{7/2}} -\frac{1}{4}\frac{W(x)}{(\lambda^2 V(x) + z^2)^{3/2}}
  \label{eqn:coeff-kernel}
\end{align}

These are the first three asymptotic components of the kernel of
$(\Delta_\lambda + z^2)^{-1}$ on the open interval $(0,1)$, i.~e.\ if we define
$\mu(x) := \sqrt{\lambda^2 V(x) + z^2}$ as above we have
\begin{align*}
  \Tr(\Delta_\lambda + z^2)^{-1} \SimAs{\Abs{(\lambda, z)}\to\infty}&
   \Int{x}{\left(
      \frac12\frac1{\mu(x)^{2}}
     -\frac1{16}\frac{\lambda^2 V''(x)}{\mu(x)^5}
     +\frac{5}{64}\frac{\lambda^4V'(x)^2}{\mu(x)^7}
     -\frac14\frac{W(x)}{\mu(x)^3}
     +O\left(\Abs{(\lambda, z)}^{-5}\right)
   \right)}.
\end{align*}
% TODO: Oben allgemein angeben, folgt ja aus der Rekursionsformel
The asymptotic order of the remainder of $-5$ follows from the fact, that by
construction every other step in our recursion formula results in the exponents
of $\xi$ in the numerator being odd, so the $\xi$ integral over $\mathbb{R}$
vanishes, in particular $k_{b_{-4}} = 0$.

Finally for this to give us all the information we were looking for we need to
carry out the $x$-Integration. Evaluating this integral will be postponed until
we have done the necessary formal steps to get to $\Tr(\Delta_\lambda +
z^2)^{-2}$, and evaluate the $(\lambda,z)$-integral. We will justify this
\textit{a posteriori}.
