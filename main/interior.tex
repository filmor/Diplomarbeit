\section{Asymptotics for the interior parametrix}
In this section we aim to prove the following Theorem:
% TODO: Existenz und explizite Formel (zumindest bis zum Grad 3).

To do this we first show that the asymptotic expansion exists and, using that as
well as the product formula for pseudo-differential operators, prove the
explicit formula in $V$, $W$ and their derivatives.
\subsection{Existence}
By the following classical theorem an elliptic pseudo-differential operator with
parameter has an asymptotic expansion in its parameter:
\begin{theorem}
    Let $M$ be a closed $n$-dimensional smooth manifold, $E$ and $F$ smooth
    vector bundles over $M$, and $A\in\mathrm{CL}^m(M,E; F)$.

    If $m+n<0$ then for all $x\in M$ $A(x)$ is trace-class and $\Tr A(\cdot)\in
    \mathrm{CS}^{m+n}(\Gamma)$, with
    \[
        \Tr A(x)\SimAs{\left|x\right|\to\infty}
        \sum_{j=0}^{\infty} e_{m-j}\!\left(\frac{x}{\left|x\right|}\right)
                            \left|x\right|^{m+n-j}
    \]
\end{theorem}

\begin{Proof}
\end{Proof}

\subsection{Calculation of the first coefficients}
The multi-parametric symbol of the operator $(\Delta_\lambda + z^2)$ is
\begin{align}
  % TODO: Array
    a_2(x,\xi,\lambda,z) &= \xi^2 + \lambda^2 V(x) + z^2 \\
    a_0(x,\xi,\lambda,z) &= W(x),
    \label{eqn:symbol}
\end{align}
with all other $a_n \equiv 0$.

% TODO REF: Shubin Thm 3.4, p 27
We use the following product formula for symbols. Let $B$ be a
pseudo-differential operator with symbol $a$ and $b$. Then the complete symbol
of $(\Delta_\lambda + z^2)B$ is (modulo $S^\infty$)
\begin{align}
  a * b := \sum_{n=0}^{\infty} \frac{1}{n!}\ \partial_\xi^n a\,D_x^n b
  \label{eqn:product-formula}
\end{align}
We are interested in the symbol of the parametrix, so we demand
\begin{align*}
  a * b \sim 1,
\end{align*}
so the asymptotic expansion of $(a*b)$ in the symbol spaces is supposed to be
$(a * b)_n \equiv \delta_{n0}$. Actually this will also work for $(b * a)$,
i.~e.\ $BA$, but this complicates further calculations as with the chosen order
we only have $\xi$-derivatives of $a_n$ which are
\begin{align}
  \partial_{\xi} a = 2\xi \text{ and } \partial_{\xi}^2 a = 2.
\end{align}
Thus the above sum collapses and only three terms are left:
\begin{align}
  a * b = (a_2 + a_0)b - 2\xi\mathrm i\,\partial_x b - \partial_x^2 b.
\end{align}
We will now split up $b$ in its asymptotic components.

From the known asymptotics of $a$ we can now inductively derive the first
coefficients of the parametrix of $(\Delta_\lambda + z^2)$.
\begin{align*}
  b_{-2} &= \frac{1}{a_2} \\
  b_{-3} &= -2\frac{\xi a_2'}{a_2^3}\\
  b_{-4} &= -4\frac{\xi^2 a_2''}{a_2^4}
  + 12\frac{\xi^2 (a_2')^2}{a_2^5} - \frac{a_0}{a_2^2} +
  2\frac{a_2''}{a_2^3} -4\frac{(a_2')^2}{a_2^4}
  \label{eqn:coeff-symbol}
\end{align*}

Since we are ultimately interested in the kernel on the diagonal we will need to
integrate the resulting symbol for the parametrix, since
\begin{align*}
  k(x,y) &= \frac{1}{2\pi}\Oscint{\xi}{
    \Eto{\mathrm i\left\langle x-y, \xi\right\rangle} a(x,\xi)
  }
  \quad
  \Rightarrow
  \quad
  k(x,x) = \frac{1}{2\pi}\Integ{\mathbb{R}}{\xi}{a(x,\xi)},
\end{align*}
where $\int_{O}$ denotes the oscillating integral (REF), which coincides with
the usual integral over $\mathbb{R}$ for $x = y$. The integrals can be handled
with in a uniform way using the beta function as seen in the introduction.
% TODO

The corresponding terms in the kernel are thus
\begin{align}
  k_{b_{-1}}(x,x,\lambda,z) &= \frac{1}{2} \frac{1}{\lambda^2 V(x) + z^2} \\
  k_{b_{-2}}(x,x,\lambda,z) &= 0 \\
  k_{b_{-3}}(x,x,\lambda,z) &= \frac{1}{4} \frac{\lambda^2 V''(x)}{(\lambda^2
    V(x) + z^2)^{5/2}} - \frac{25}{64} \frac{\lambda^4 V'(x)^2}{(\lambda^2V(x)
    + z^2)^{7/2}} -\frac{1}{4}\frac{W(x)}{(\lambda^2 V(x) + z^2)^{3/2}}
  \label{eqn:coeff-kernel}
\end{align}

These are the first three asymptotic components of the kernel of
$(\Delta_\lambda + z^2)^{-1}$ on the open interval $(0,1)$, i.e. if we define
$\mu(x) := \sqrt{\lambda^2 V(x) + z^2}$ we have
\begin{align*}
  \Tr(\Delta_\lambda + z^2)^{-1} \SimAs{\Abs{(\lambda, z)}\to\infty}&
   \Int{x}{\left(
      \frac12\frac1{\mu(x)^{2}}
     +\frac14\frac{\lambda^2 V''(x)}{\mu(x)^5}
     -\frac{25}{64}\frac{\lambda^4V'(x)^2}{\mu(x)^7}
     -\frac14\frac{W(x)}{\mu(x)^3}
     +O\left(\Abs{(\lambda, z)}^{-5}\right)
   \right)}.
\end{align*}
The asymptotic order of the remainder of $-5$ follows from the fact, that by
construction every other stepin our recursion formula results in the exponents
of $\xi$ in the numerator being odd, so the $\xi$ integral over $\mathbb{R}$
vanishes, in particular $k_{b_{-4}} = 0$.

Finally for this to give us all the information we were looking for we need to
carry out the $x$-Integration.

Evaluating this integral will be postponed until we have done the necessary
formal steps to get to $\Tr(\Delta_\lambda + z^2)^{-2}$, and evaluate the 
$(\lambda,z)$-integral. We will justify this a posteriori.
