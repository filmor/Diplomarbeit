\section{Asymptotics of the Interior Parametrix}
In this section we aim to prove the following Theorem:
\begin{MainTheorem}
  \label{main:interior}
  % TODO: Existenz und explizite Formel (zumindest bis zum Grad 3).

  \begin{Remark}
    For $V\equiv 0$ this expansion has already been proven in
    \cite[Ch 2.1]{Gel'fand1975} (where $z^2 \mapsto \zeta$ and $W(x) \mapsto
    u(x)$) and calculated for even higher orders, but we will restrict ourselves
    in this text to what we need for the particular calculation we want to do in
    the end.  The Appendix~\ref{app:source-code} contains the source code for a
    simple Python program that calculates the asymptotic components (apart from
    the $x$-Integration) of the resolvent trace, as well as a list of the
    coefficients to higher order.
  \end{Remark}
\end{MainTheorem}
To do this we first note that the asymptotic expansion exists by a classical
theorem on the theory of pseudo-differential operators with paramater, namely
\begin{Theorem}
  Let $M$ be a closed $n$-dimensional smooth manifold and
  $\Gamma\subset\mathbb{R}^p$ a cone and $A\in\mathrm{CL}^m(M;\Gamma)$ (the set
  of classical pseudo-differential operators with parameter).

  If $m+n<0$ then for all $x\in M$, then for $\mu\in\Gamma$ $A(\mu)$ is
  trace-class and $\Tr A(\cdot)\in \mathrm{CS}^{m+n}(\Gamma)$ (the space of
  classical symbols), with
  \begin{align*}
    \Tr A(\mu)\SimAs{\Abs{\mu}\to\infty}
    \sum_{j=0}^{\infty} e_{m-j}\bigl(\tfrac{\mu}{\Abs{\mu}}\bigr)
    \Abs{\mu}^{m+n-j}.
  \end{align*}
  \begin{Proof}
    A proof can be found in \cite[Thm 5.1]{Les:PDO}, where it is shown in the
    slightly more general case of log-polynomial asymptotics. The idea is first
    prove a corresponding asymptotic expansion of the kernel on the diagonal by
    use of the asymptotic expansion of the symbol, which is given for classical
    operators. The theorem follows then by integrating the expansion of the
    kernel over $x$, which is well-defined as $M$ is compact.
  \end{Proof}
\end{Theorem}
Using that as well as the product formula for pseudo-differential operators, we
prove the explicit formula in $V$, $W$ and their derivatives.

\subsection{Calculation of the first coefficients}
The multi-parametric symbol of the operator $(\Delta_\lambda + z^2)$ is
\begin{align}
  a_2(x,\xi,\lambda,z) &= \xi^2 + \lambda^2 V(x) + z^2 \\
  a_0(x,\xi,\lambda,z) &= W(x),
  \label{eqn:symbol}
\end{align}
with all other $a_n \equiv 0$. To not complicate the following calculations we
suppress the continuous repetition of the parameters, a prime ($a_2'$) denotes
the derivative in $x$.

We use the following product formula for symbols: Let $A$ and $B$ be
pseudo-differential operators with symbols $a$ and $b$, respectively. Then the
complete symbol of $(\Delta_\lambda + z^2)B$ is (modulo $S^\infty$) given by
\begin{align}
  a * b := \sum_{n=0}^{\infty} \frac{1}{n!}\ \partial_\xi^n a\,D_x^n b,
  \label{eqn:product-formula}
\end{align}
where $D_x^n := (-i)^n \partial_x^n$. A proof of this formula can be found in
\cite[Thm 3.4]{Shu:POS}.

We are interested in the symbol of the parametrix, so we demand
\begin{align*}
  a * b \sim 1,
\end{align*}
so the asymptotic expansion of $a * b$ in the symbol spaces is supposed to be
$(a * b)_n \equiv \delta_{n0}$ if $B$ is a parametrix of $A$. Actually this will
also work for $b * a$, i.e.\ $BA$, but this complicates further calculations as
with the chosen order we only have $\xi$-derivatives of $a = a_2 + a_0$ which
are
\begin{align}
  \partial_{\xi} a = 2\xi \text{ and } \partial_{\xi}^2 a = 2.
\end{align}
Thus the sum in \eqref{eqn:product-formula} collapses and only three terms are
left:
\begin{align*}
  a * b &= (a_2 + a_0)b - 2\xi\mathrm i\,\partial_x b - \partial_x^2 b
  \ \mathrel{\overset{!}{\sim}}\ 1, \\
  \newtag
  \label{eqn:int-recursion}
  \Rightarrow b_{-n} &= \frac1{a_2} \left( 2\xi\mathrm i\, \partial_x b_{-(n-1)}
  + \partial_x^2 b_{-(n-2)} - a_0 b_{-(n-2)} \right),
\end{align*}
with the starting values $b_0 = 0$, $b_{-1} = 0$ and $b_{-2} = a_2^{-1}$.
%TODO Warum?
We can now inductively derive the first coefficients of the parametrix of
$(\Delta_\lambda + z^2)$:
\begin{align*}
  \label{eqn:coeff-symbol}
  b_{-2} &= \frac{1}{a_2} \\
  b_{-3} &= -2\frac{\xi a_2'}{a_2^3}\\
  b_{-4} &= 4\xi^2\left(\frac{a_2''}{a_2^4}
  - 3\frac{(a_2')^2}{a_2^5}\right) - \frac{a_0}{a_2^2}
  - 2\frac{a_2''}{a_2^3} + 2\frac{(a_2')^2}{a_2^4}
\end{align*}
Since we are ultimately interested in the kernel on the diagonal we will need to
integrate the resulting symbol for the parametrix, since for a
pseudo-differential operator $A$ with symbol $a$ we have
\begin{align*}
  k_A(x,y) &= \frac{1}{2\pi}\Oscint{\xi}{
    \Eto{\mathrm i\left\langle x-y, \xi\right\rangle} a(x,\xi)
  }
  \quad
  \Rightarrow
  \quad
  k(x,x) = \frac{1}{2\pi}\Integ{\mathbb{R}}{\xi}{a(x,\xi)},
\end{align*}
where $\Oscint{\xi}{\,\cdot\,}$ denotes the oscillating integral (as defined in
\cite[]{Shu:POS}), which coincides with the usual integral over $\mathbb{R}$ for
$x = y$. The integrals over $\xi$ can be handled with in a uniform way using the
beta function as seen in the introduction (\ref{lem:beta_function_formula}) by
setting $n = (\text{order of $\xi$ in the numerator})$, $m = \tfrac12(\text{order
of $a_2$ in the denominator})$, $a=\mu(x)=\sqrt{\lambda^2 V(x) + W(x)}$ and $b =
1$. We also see, that for odd $n$ the terms vanish, as we have an integral over
an odd function, while for even $n$ we get an additional factor of $2$ for
integrating over $\mathbb{R}$ instead of $\Rplus$.

The corresponding terms in the kernel are thus
\begin{align}
  k_{-2}(x,x,\lambda,z) &= \frac{1}{2} \frac{1}{\lambda^2 V(x) + z^2} \\
  k_{-3}(x,x,\lambda,z) &= 0 \\
  k_{-4}(x,x,\lambda,z) &= - \frac{1}{16} \frac{\lambda^2 V''(x)}{(\lambda^2
    V(x) + z^2)^{5/2}} + \frac{5}{64} \frac{\lambda^4 V'(x)^2}{(\lambda^2V(x)
    + z^2)^{7/2}} -\frac{1}{4}\frac{W(x)}{(\lambda^2 V(x) + z^2)^{3/2}}.
  \label{eqn:coeff-kernel}
\end{align}

These are the first three asymptotic components of the kernel of
$(\Delta_\lambda + z^2)^{-1}$ on the open interval $(0,1)$, i.e.\ if we define
$\mu(x) := \sqrt{\lambda^2 V(x) + z^2}$ as above we have
\begin{align*}
  k_{(\Delta_\lambda + z^2)^{-1}} \SimAs{\Abs{(\lambda, z)}\to\infty}
    \frac12\frac1{\mu(x)^{2}}
   -\frac1{16}\frac{\lambda^2 V''(x)}{\mu(x)^5}
   +\frac{5}{64}\frac{\lambda^4V'(x)^2}{\mu(x)^7}
 -\frac14\frac{W(x)}{\mu(x)^3}
 +O\!\bigl(\Abs{(\lambda, z)}^{-5}\bigr)
,
\end{align*}
which proves \thref{main:interior}.  The asymptotic order of the remainder of
$-5$ follows from the fact, that by construction every other step in our
recursion formula results in the exponents of $\xi$ in the numerator being odd,
so the $\xi$ integral over $\mathbb{R}$ vanishes, in particular $k_{b_{-4}} =
0$.

Finally for this to give us all the information we were looking for we need to
carry out the $x$-Integration. Evaluating this integral will be postponed until
we have done the necessary formal steps to get to $\Tr(\Delta_\lambda +
z^2)^{-2}$, and evaluate the $(\lambda,z)$-integral. Since by
Equation~\eqref{frm:shift} and Equation~\eqref{eqn:central-theorem} we know,
that we will only differentiate and integrate with a $\log\lambda$-term, and all
intermediate steps are integrable we can indeed postpone the evaluation until we
have enough information on $V$ and $W$ to perform the integration.

\begin{Remark}
  We can see by closely looking on the generating
  Equation~\eqref{eqn:int-recursion} that we can, without doing the explicit
  calculations show, that the terms in the kernel expansion have a certain
  structure, we will consider $k_{-n}$.

  First, we define an order for possible components in the numerator of the
  terms in the expansion (where $A(x)$ and $B(x)$ are expressions in $V$, $W$
  and their derivatives, and $C\in\mathbb{C}$):
  \begin{align}
    &\ord \lambda^2 V(x) := 0 \\
    &\ord W(x) := 2 \\
    &\ord \partial_x A(x) := \ord(A(x)) + 1 \\
    &\ord CA(x) := \ord(A(x)) \\
    &\ord A(x)B(x) := \ord A(x) + \ord B(x) \\
    &\ord C := 0
  \end{align}
  We know by construction, that apart from some complex constant only
  derivatives of $\lambda^2 V(x)$ (in form of derivatives of $a_2$), $W(x)$ as
  well as derivatives of $W(x)$ may appear in the numerator. The denominator is
  uniquely defined by the count of $\lambda^2 V^{(l)}(x)$ terms in the
  numerator, i.e.\ if there are $k$ factors of $\lambda^2 V^{(l)}$, then the
  denominator is $(\lambda^2 V(x) + z^2)^{n+2k-1}$.

  For the numerator we can see, that the order of the expression always adds up
  to $n-2$, for example if we consider $n=4$ we have the following
  possibilities
  \begin{align*}
    V''(x), W(x), (V'(x))^2,
  \end{align*}
  since those are exactly the terms with order $2$ we can produce.

  \begin{Proof}
    % TODO: \xi verarbeiten (ist aber nicht viel, einfach an den entsprechenden
    %       Stellen einfügen
    % TODO: "Summe" von zwei Mengen
    We prove this claim by induction over $n$ on symbol level. It suffices to
    prove, that if for $n$ we have all possible terms of order $n-2$, the
    recursion formula \eqref{frm:int-recursion} again yields all possible terms
    for $(n+1)-2$. We can specify each term by two finite sets $S_V$ and $S_W$
    of tuples of the derivation order and their respective multiplicities:
    \begin{align*}
      A_{S_V,S_W}(x) &= \prod_{(p,\pi)\in S_V} (\lambda^2 V^{(p)}(x))^\pi
      \prod_{(r,\rho)\in S_W} (W^{(r)}(x))^\rho \\
      \Rightarrow \ord A_{S_V,S_W}(x) &= \sum_{(p,\pi)\in S_V} p\pi +
      \sum_{(r,\rho)\in S_W} (2+r)\rho
    \end{align*}
    We know from the previous considerations, that each such term has the
    denominator
    \begin{align*}
      D_{S_V}(x) &= \prod_{(p,\pi)\in S_V} (\lambda^2 V(x) + z^2)^\pi.
    \end{align*}
    By construction we have $\ord D_{S_V} = 0$. Now we need to differentiate the
    quotient $A_{S_V,S_W}(x) / D_{S_V}(x)$ to see, how the sets $S_V$ and $S_W$
    change:
    \begin{align*}
      \frac{\mathrm d}{\mathrm dx}\frac{A(x)}{D(x)} &= \frac{\partial_x
        A(x)}{D(x)} - \frac{A(x)}{D(x)} \frac{\partial_x D(x)}{D(x)}
    \end{align*}
    It suffices to calculate the derivatives of $A$ and $D$.

    Let $\gamma = \sum_{(p,\pi)\in S_V}\pi$, then
    \begin{align*}
      % TODO:
      \partial_x D_{S_V}(x) &= -\gamma (\lambda^2 V(x) + z^2)^{\gamma - 1}
      (\lambda^2 V'(x)) \\
      \Rightarrow \ord \partial_x D_{S_V} &= \ord (\lambda^2 V(x) + z^2)^{\gamma - 1}
      + \ord \lambda^2 V'(x) = 1
    \end{align*}
    For $A$ we consider one particular term for $(p,\pi)$, where $f\in\{V,W\}$:
    \begin{align*}
      \partial_x (f^{(p)}(x))^\pi &= \pi (f^{(p)})^{\pi-1}
      (f^{(p+1)}) \\
      \Rightarrow \{(p,\pi)\} &\mapsto \{(p,\pi-1), (p+1,1)\} \\
      \Rightarrow \ord \partial_x (V^{(p)}(x))^\pi &= p(\pi-1) + (p+1)1 = p\pi +
      1 \\
      \Rightarrow \ord \partial_x (W^{(r)}(x))^\rho &= (2 + r)(\rho-1) +
      (2+r+1)1 = (2+r)\rho + 1.
    \end{align*}
    We thus have shown, that
    \begin{align*}
      \ord \partial_x\frac{A_{S_V,S_W}}{D_{S_V}(x)} = \ord
      \frac{A_{S_V,S_W}(x)}{D_{S_V}(x)} + 1,
    \end{align*}
    which proves our first claim, namely that the order of each term $k_{-n}$ is
    $n-2$, since we start with $k_{-2}$ which has order $0$.

    Now for the second claim we have a closer look on the produced sets $S_V$
    and $S_W$. We have seen, that $S_V$ and $S_W$ transform under
    differentiation by
    \begin{align*}
      \{(p,\pi)\} \mapsto \{(p,\pi-1),(p+1,1)\},
    \end{align*}
    where in a set $S$ for each $p$ the corresponding $\pi$ are to be summed up
    after each differentiation step, i.e.\
    $\{(p,\pi-1),(p,1)\}\mapsto\{(p,\pi)\})$. Thus deriving the result again we
    get
    \begin{align*}
      \{(p,\pi-1), (p+1,1)\} &\mapsto \{(p,\pi-2),(p+1,1),(p+1,0),(p+2,1)\} \\
                             &\mapsto \{(p,\pi-2), (p+1,1), (p+2,1)\}.
    \end{align*}
    Now we only have to show, that if $b_{-(n-1)}$ and $b_{-(n-2)}$ are sums
    such that all possible $S_V$ and $S_W$ for which $\ord A_{S_V,S_W} = n-2$
    appear, then this does also hold for $b_{-n}$.
    % TODO
  \end{Proof}
\end{Remark}
