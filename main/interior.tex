\section{Asymptotics of the Interior Parametrix}
In this section we aim to prove the following Theorem:
\begin{MainTheorem}
  \label{main:interior}
  % TODO: Existenz und explizite Formel (zumindest bis zum Grad 3).

  \begin{Remark}
    Note again, that the order we are looking for here is \emph{not} $-5$ but
    instead $-3$ (i.e.\ the three terms in the middle) since we are dealing with
    $\Tr(\Delta_\lambda + z^2)^{-1}$ instead of $\Tr(\Delta_\lambda + z^2)^{-2}$
    here.
  \end{Remark}
  \begin{Remark}
    For $V\equiv 0$ this expansion has already been proven in
    \cite[Ch 2.1]{Gel'fand1975} (where $z^2 \mapsto \zeta$ and $W(x) \mapsto
    u(x)$) and calculated for even higher orders, but we will restrict ourselves
    in this text to what we need for the particular calculation we want to do in
    the end.
  \end{Remark}
\end{MainTheorem}
To prove this we first note that the asymptotic expansion exists by a classical
theorem on the theory of pseudo-differential operators with parameter, namely
\begin{Theorem}
  Let $U\subset\mathbb{R}^n$ be open, $\Gamma\subset\mathbb{R}^p$ a cone and
  $A\in\mathrm{CL}^m(U;\Gamma)$ (the set of classical pseudo-differential
  operators with parameter in $\Gamma$).

  If $m+n<0$ then for all $\mu\in\Gamma$ the operator $A(\mu)$ is trace-class
  and the kernel on the diagonal $x\mapsto k_A(x, x;\mu)$ is in
  $\mathrm{CS}^{m+n}(\Gamma)$ (the space of classical symbols), with the
  asymptotic expansion
  \begin{equation*}
    k_A(x,x;\mu)\SimAs{\Abs{\mu}\to\infty}
    \sum_{j=0}^{\infty} e_{m-j}\bigl(x,\tfrac{\mu}{\Abs{\mu}}\bigr)
    \Abs{\mu}^{m+n-j}.
  \end{equation*}
  \begin{Proof}
    A proof can be found in \cite[Thm 5.1]{Les:PDO}, where it is shown in the
    slightly more general case of log-polynomial symbols. The idea is to use
    the asymptotic expansion of the symbol, which is given for classical
    operators, and explicitly evaluate the Schwartz kernel on the diagonal by
    means of the formula
    \begin{equation*}
      k_A(x,x;\mu) = \frac{1}{2\pi}\Integ{\mathbb R}{x}{a(x,x;\mu)}.
    \end{equation*}
    A reference for the formula and explanation is given in the next subsection
    as we will use it there for concrete calculations.
  \end{Proof}
\end{Theorem}
Using the existence as well as the product formula for pseudo-differential
operators, we prove the explicit formula in $V$, $W$ and their derivatives.

\subsection{Calculation of the first Coefficients}
The multi-parametric symbol of the operator $(\Delta_\lambda + z^2)$ is given by
\begin{equation}
  \label{eqn:symbol}
  \begin{split}
    a_2(x,\xi,\lambda,z) &= \xi^2 + \lambda^2 V(x) + z^2 \\
    a_0(x,\xi,\lambda,z) &= W(x),
  \end{split}
\end{equation}
with all other $a_n \equiv 0$. To not complicate the following calculations we
suppress the continuous repetition of the parameters, a prime ($a_2'$) denotes
the derivative in $x$.

We use the following product formula for symbols: Let $A$ and $B$ be
pseudo-differential operators with symbols $a$ and $b$, respectively. Then the
complete symbol of $(\Delta_\lambda + z^2)B$ is (modulo $S^\infty$) given by
\begin{equation}
  a * b := \sum_{n=0}^{\infty} \frac{1}{n!}\ \partial_\xi^n a\,D_x^n b,
  \label{eqn:product-formula}
\end{equation}
where $D_x^n := (-i)^n \partial_x^n$. A proof of this formula can be found in
\cite[Thm 3.4]{Shu:POS}.

We are interested in the symbol of the parametrix, so we demand that (as
symbols)
\begin{equation*}
  a * b \sim 1,
\end{equation*}
so the components of the asymptotic expansion of $a * b$ in the symbol spaces
are supposed to be
\begin{equation}
  \label{eqn:ab-as-symbols}
  (a * b)_n \equiv \delta_{n0}
\end{equation}
if $B$ is a parametrix of $A$. Actually this will
also work for $b * a$, i.e.\ $BA$, but this complicates further calculations as
with the chosen order we only get $\xi$-derivatives of $a = a_2 + a_0$ which
are
\begin{equation}
  \partial_{\xi} a = 2\xi \text{ and } \partial_{\xi}^2 a = 2.
\end{equation}
Thus the sum in \cref{eqn:product-formula} collapses and only three terms are
left:
\begin{equation*}
  a * b = (a_2 + a_0)b - 2\xi\mathrm i\,\partial_x b - \partial_x^2 b
  \ \mathrel{\overset{!}{\sim}}\ 1,
\end{equation*}
which gives by use of \cref{eqn:ab-as-symbols}
\begin{equation}
  \label{eqn:int-recursion}
  b_{-n} = \frac1{a_2} \left( 2\xi\mathrm i\, \partial_x b_{-(n-1)}
  + \partial_x^2 b_{-(n-2)} - a_0 b_{-(n-2)} \right),
\end{equation}
with the starting values $b_0 = 0$, $b_{-1} = 0$ and $b_{-2} = a_2^{-1}$ since
the order of the inverse has to be $-2$. We can now inductively derive the first
coefficients of the parametrix of $(\Delta_\lambda + z^2)$:
\begin{equation}
  \label{eqn:coeff-symbol}
  \begin{split}
    b_{-2} &= \frac{1}{a_2} \\
    b_{-3} &= -2\frac{\xi a_2'}{a_2^3}\\
    b_{-4} &= 4\xi^2\left(\frac{a_2''}{a_2^4}
    - 3\frac{(a_2')^2}{a_2^5}\right) - \frac{a_0}{a_2^2}
    - 2\frac{a_2''}{a_2^3} + 2\frac{(a_2')^2}{a_2^4}
  \end{split}
\end{equation}
Since we are ultimately interested in the kernel on the diagonal we will need to
integrate the resulting symbol of the parametrix, as for a pseudo-differential
operator $A$ with symbol $a$ we have
\begin{equation*}
  k_A(x,y) = \frac{1}{2\pi}\Oscint{\xi}{
    \Eto{\mathrm i\left\langle x-y, \xi\right\rangle} a(x,\xi)
  }
  \quad
  \Rightarrow
  \quad
  k(x,x) = \frac{1}{2\pi}\Integ{\mathbb{R}}{\xi}{a(x,\xi)},
\end{equation*}
where $\Oscint{\xi}{\,\cdot\,}$ denotes the oscillatory integral (as defined in
\cite[Def.~1.1]{Shu:POS}) which coincides with the usual integral over
$\mathbb{R}$ for $x = y$. The integrals over $\xi$ can be dealt with in a
uniform way using the beta function as seen in the introduction
(\cref{lem:beta_function_formula}) by letting $n$ be the order of $\xi$ in the
numerator, $m$ the exponent of $a_2$ in the denominator, $a := \mu(x) =
\sqrt{\lambda^2 V(x) + W(x)}$ and $b = 1$. We also see, that for odd $n$ the
terms vanish, as we have an integral over an odd function, while for even $n$ we
get an additional factor of $2$ for integrating over $\mathbb{R}$ instead of
$\Rplus$.

The corresponding terms in the kernel are thus
\begin{equation}
  \begin{split}
    k_{-1}(x,x,\lambda,z) &= \frac{1}{2} \frac{1}{(\lambda^2 V(x) + z^2)^{1/2}} \\
    k_{-2}(x,x,\lambda,z) &= 0 \\
    k_{-3}(x,x,\lambda,z) &= - \frac{1}{16} \frac{\lambda^2 V''(x)}{(\lambda^2
      V(x) + z^2)^{5/2}} + \frac{5}{64} \frac{\lambda^4 V'(x)^2}{(\lambda^2V(x)
      + z^2)^{7/2}} -\frac{1}{4}\frac{W(x)}{(\lambda^2 V(x) + z^2)^{3/2}}.
    \end{split}
  \label{eqn:coeff-kernel}
\end{equation}
We note, that the $(\lambda,z)$-order of the integrated terms increases by $1$
which is reflected by a change of the index.

These are the first three asymptotic components of the kernel of
$(\Delta_\lambda + z^2)^{-1}$ on the open interval $(0,1)$, i.e.\ if we define
$\mu(x) := \sqrt{\lambda^2 V(x) + z^2}$ as above we have
\begin{align*}
    \Tr(\Delta_\lambda + z^2)^{-1} \\ \SimAs{\Abs{(\lambda, z)}\to\infty} &
   \Int{x}{\left(
      \frac12\frac1{\mu(x)^{2}}
     -\frac1{16}\frac{\lambda^2 V''(x)}{\mu(x)^5}
     +\frac{5}{64}\frac{\lambda^4V'(x)^2}{\mu(x)^7}
     -\frac14\frac{W(x)}{\mu(x)^3}
     +O\left(\Abs{(\lambda, z)}^{-5}\right)
   \right)}
,
\end{align*}
which proves \thref{main:interior}. The asymptotic order of the remainder of
$-5$ follows from the fact, that by construction every other step in our
recursion formula results in the exponents of $\xi$ in the numerator being odd,
so the $\xi$ integral over $\mathbb{R}$ vanishes, in particular we get $k_{-4} =
0$.

Finally for this to give us all the information we were looking for we need to
carry out the $x$-Integration. Since by \cref{frm:shift,eqn:central-thm} we
know, that we will only differentiate the formula and integrate the result
multiplied with a $\log\lambda$-term, and all intermediate steps are integrable
we can postpone the evaluation until we have enough information on $V$ and $W$
to perform the integration.

\subsection{Homogeneity in $V$, $W$ and their Derivatives}
We see by closely looking on the generating \cref{eqn:int-recursion} that we can
(without carrying out the explicit calculations) show that the terms in the
kernel expansion have a certain structure.  First, we define an order for the
building blocks in the numerator of the terms in the expansion:
\begin{Definition}
  \label{def:order}
  We define an order on terms of $V$ and $W$ by:
  \begin{align*}
    \ord \lambda^2 V(x) &:= 0 \\
              \ord W(x) &:= 2
  \end{align*}
  For $A(x)\neq 0$ and $B(x)$ being rational expressions in $V$, $W$ and their
  derivatives, and $C\in\mathbb{C}$ we also define the following calculation
  rules:
  \begin{align*}
    \ord C &:= 0 \\
    \ord A(x)^{C} &:= C\ord A(x) \\
    \ord \partial_x A(x) &:= \ord(A(x)) + 1 \\
    \ord A(x)B(x) &:= \ord A(x) + \ord B(x)
  \end{align*}
  The sum of two terms $A(x)$ and $B(x)$ has a well-defined order if $\ord A(x)
  = \ord B(x)$, namely $\ord(A(x) + B(x)) = \ord A(x) = \ord B(x)$.
  \begin{Remark}
    The definition can be motivated from our toy example given in the
    introduction where we had
    \begin{align*}
      V(x) = f(x)^{-2} \text{ and } W(x) = \frac{f''(x)}{2f(x)} -
      \left(\frac{f'(x)}{2f(x)}\right)^2.
    \end{align*}
    If we now rescale this as $x\mapsto kx$ we get
    \begin{align*}
      V(x) &\mapsto f(kx)^{-2} = V(kx) \text{ while} \\
      W(x) &\mapsto \frac{k^2 f''(kx)}{2f(x)} -
      \left(\frac{kf'(kx)}{2f(x)}\right)^2 = k^2 W(kx),
    \end{align*}
    $\ord$ thus represents the rescaling exponent. This does also immediately
    give the calculation rules, especially the increase of $\ord$ on
    differentiation comes directly from the fact that $\partial^n
    f\circ(x\mapsto kx)(x) = k^n f(kx)$ (which also proves the well-definedness
    of $\ord$).
  \end{Remark}
\end{Definition}
We know by construction that, apart from some complex constant, only derivatives
of $\lambda^2 V(x)$ (in form of derivatives of $a_2$), $W(x)$ as well as
derivatives of $W(x)$ may appear in the numerator. The denominator is uniquely
defined by the count of $\lambda^2 V^{(l)}(x)$ terms in the numerator, i.e.\ if
there are $k$ factors of $\lambda^2 V^{(l)}$, then the denominator is
$(\lambda^2 V(x) + z^2)^{(n+2k-1)/2}$, by $(\lambda,z)$-homogeneity.

For the numerator we can see, that the order of the expression always adds up to
$n-2$ ($n$ being the $(\lambda,z)$-order, for example if we consider $n=4$ we
have the following possibilities
\begin{align*}
  V''(x), W(x), (V'(x))^2,
\end{align*}
since those are exactly the terms with order $2$ we can produce. This is what we
will prove now:
\begin{MainTheorem}
  Given the previous definition of the order $\ord$ in $V$, $W$ and their
  respective derivatives the following holds for the terms in the asymptotic
  expansion of the kernel on the diagonal of $(\Delta_\lambda + z^2)^{-1}$:
  \begin{equation}
    \label{eqn:ord-vs-mu-order}
    \ord k_{-n}(x, x) = n - 1
  \end{equation}
  \begin{Proof}
    First we note that the $\xi$-integration does not change the order of the
    expression since
    \begin{equation*}
      \ord \frac{\xi^n}{(\lambda^2 V(x) + z^2 + \xi^2)^m} = \ord (\lambda^2 V(x)
      + z^2)^{n-2m+1} = 0,
    \end{equation*}
    thus the order of the term before and after the integration is the same.
    However, since the $(\lambda,z)$-order changes, the indexation is different,
    i.e.\ $k_{-n}$ corresponds to $b_{-(n+1)}$. We
    thus prove the claim
    \begin{equation*}
      \ord k_{-(n-1)} = \ord b_{-n} = n - 2
    \end{equation*}
    by induction over $n$ on symbol level. We start with $b_{-2} = a_2^{-1} =
    (\lambda^2 V(x) + z^2 + \xi^2)^{-1}$, which clearly has order $0$. Since the
    recursion formula involves the two previous terms we also need to check
    $b_{-3}$:
    \begin{equation*}
      \ord b_{-3} = \ord\left(2\xi \frac{a_2'}{a_2^3}\right) = \ord a_2' = \ord
      \lambda^2 V'(x) = 1
    \end{equation*}
    For the induction step we only need to look closely on the recursion formula
    (\cref{eqn:int-recursion}) and use the given calculation rules:
    \begin{align*}
      &\ord 2\xi i \partial_x b_{-(n-1)} = \ord \partial_x b_{-(n-1)} = ((n-1) - 2) +
      1 = n - 2\\
      &\ord \partial_x^2 b_{-(n-2)} = ((n - 2) - 2) + 2 = n - 2 \\
      &\ord a_0 b_{-(n-2)} = \ord W(x) b_{-(n-2)} = 2 + ((n-2) - 2) = n - 2
    \end{align*}
    Since $b_{-n}$ is the sum of those terms divided by $a_2$, which has order
    $0$ this proves the homogeneity in $V$, $W$ and their derivatives.
  \end{Proof}
\end{MainTheorem}
