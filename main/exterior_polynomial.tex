\subsection{Rationality of the homogeneous Components}
We have seen in the preceding subsection that, like in the interior case, the
first homogeneous components of the multi-parametric expansion of
$\Tr(\Delta_\lambda + z^2)^{-1}$ are rational functions in $V$, $W$ and their
derivatives at $x = 0$. We want to prove that this also holds for all higher
orders in the asymptotic expansion. In particular we want to show, that we
always get a polynomial that is divided by integral powers of $\mu = \lambda^2
V(0) + W(0)$.

We know that $f_\lambda$ (\cref{def:f-lambda}) is analytic in $x$ near the
boundary since it is a composition of the analytic functions $V$ and $W$ (and
cutoff functions that are $1$ near the boundary), thus we can write with
$N\in\mathbb{N}$ and some $\xi\in[0,1]$
\begin{equation}
  \label{eqn:taylor-of-f}
  f_\lambda(x) = \sum_{n=0}^{N-1} \frac{f_\lambda^{(n)}(0)}{n!} x^n + R_N(x)
  \text{ with } R_N(x) = \frac{f^{(N)}(\xi)}{N!} x^{N}.
\end{equation}
It thus suffices to show that if we substitute each $f_\lambda$ in the
composition of our kernels by a function $x^n$ for some $n\in\mathbb{N}$ (that
may be different for each substitution) we end up with a finite asymptotic sum.
We will do this recursively by looking at suitable generalisations of the
kernels. Essentially, after splitting up the integrals, we will just evaluate
them directly for $k=0$ and use the following formula otherwise, making use of
$n\in\mathbb{N}$,
\begin{equation}
  \Integ[b]{a}{x}{\Eto{-\mu kx}x^n}
    = -n!\left(\Eto{-\mu kb}\sum_{m=0}^n \frac{b^m}{m!(\mu k)^{n+1-m}} -
              \Eto{-\mu ka}\sum_{m=0}^n \frac{a^m}{m!(\mu k)^{n+1-m}}\right),
\end{equation}
which follows directly by the substitution $x \mapsto \mu kx$ from the formula
for the (upper) incomplete gamma function given (and proven) in the
\cref{app:gamma} (\cref{lem:incomplete-gamma}). Since we need it on multiple
occasions we define for $k\ne 0$ the shortcut
\begin{equation}
  \label{def:h-n-k}
  H^n_k(x) := n!\sum_{m=0}^n \frac{x^m}{m!(\mu k)^{n+1-m}}
\end{equation}
such that we can write
\begin{equation}
  \label{eqn:ex-integ}
  \Integ[b]{a}{x}{\Eto{-\mu kx}x^n} = -\Eto{-\mu kb}H^n_k(b) + \Eto{-\mu
  ka}H^n_k(a).
\end{equation}
Note that in this context we will make use of the convention $0^0 := 1$ to
deal with the special case of $a = 0$ or $b = 0$.

The integral part of the proof lies in the following calculations that aim to
describe how the kernels behave on composition, i.e.\ we want to investigate
the kernels (cf.\ \cref{eqn:kernel-plus,eqn:kernel-minus})
\begin{align*}
  k_{R_{++}}(a,b) &= \Int{x}{\Eto{-\mu((a + x) + (x + b))}f_\lambda(x)} \text{ and }\\
  k_{R_{+-}}(a,b) &= \Int{x}{\Eto{-\mu((a + x) + \Abs{x - b})}f_\lambda(x)}.
\end{align*}
It will suffice to use a Taylor polynomial of $f_\lambda(x)$, since we can
estimate the $\mu$-order of all combinations involving the residue term $R$ in
the same way as we estimate the interesting terms. The ansatz is to show that if
we start with $R_+$ all kernel compositions that we apply from the right
result in a kernel of the form $\Eto{-\mu(a+b)}p(b)$ for some polynomial $p$
(that has specific properties regarding its degree and $\mu$-order in each term)
and is thus again similar to $k_{R_+}$. All terms that are not of this form do not
contribute to the final asymptotic expansion.
\begin{Lemma}
  \label{lem:asymp-1}
  For all $n\in\mathbb{N}_0$, $a,b\in[0,1]$ and $\mu\in\mathbb{R}$ we
  have
  \begin{align}
    \label{eqn:poly-1}
    \Int{x}{\Eto{-\mu((a + x) + (x + b))}\,x^n} &= \Eto{-\mu(a+b)}H_2^n(0) -
    \Eto{-\mu(a-b)}\Eto{-2\mu}H^n_2(1) \text{ and }\\
    \label{eqn:poly-2}
    \Int{x}{\Eto{-\mu((a + x) + \Abs{x - b})}\,x^n} &=
    \Eto{-\mu(a+b)}\left(\frac{b^{n+1}}{n+1} + H_2^n(b)\right) +
    \Eto{-\mu(a-b)}\Eto{-2\mu}H^n_2(1).
  \end{align}
  \begin{Remark}
    In particular, discarding terms that won't contribute in the asymptotic limit
    $\mu\to\infty$, the result is of the form $\Eto{-\mu(a+b)}g(b)$, with $g$
    being a polynomial.
  \end{Remark}
  \begin{Proof}
    The formulas follow by direct application of \cref{eqn:ex-integ}:
    \begin{align*}
      \Int{x}{\Eto{-\mu((a+x)+(b+x))}x^n} &= \Eto{-\mu(a+b)}\Int{x}{\Eto{-2\mu x}x^n} \\
        &= \Eto{-\mu(a+b)}\left(-\Eto{-2\mu}H^n_2(1) + H^n_2(0)\right) \\
        &= \Eto{-\mu(a+b)} H^n_2(0) - \Eto{-2\mu} H^n_2(1)
    \end{align*}
    For \cref{eqn:poly-2} we split the integral to account for the modulus in
    the exponent:
    \begin{align*}
      \Int{x}{\Eto{-\mu( (a+x) + \Abs{x - b})}f(x)} &=
      \Eto{-\mu(a+b)}\Integ[b]{0}{x}{\Eto{-\mu(x-x)}f(x)} \\ &+
      \Eto{-\mu(a-b)}\Integ[1]{b}{x}{\Eto{-\mu(x+x)}f(x)}
    \end{align*}
    For the lower term we get
    \begin{equation*}
      \Eto{-\mu(a+b)}\Integ[b]{0}{x}{x^n} =
      \Eto{-\mu(a+b)}\left(\frac{b^{n+1}}{n+1} - 0\right),
    \end{equation*}
    which gives the first part of the result. For the upper integration we
    finally have
    \begin{align*}
      \Eto{-\mu(a-b)}\Integ[1]{b}{x}{\Eto{-2\mu x}f(x)} &=
      \Eto{-\mu(a-b)}\left(-\Eto{-2\mu}H^n_2(1) + \Eto{-2\mu b}H^n_2(b)\right) \\
      &= \Eto{-\mu(a+b)}H^n_2(b) - \Eto{-\mu(a-b)}\Eto{-2\mu}H^n_2(1),
    \end{align*}
    which proves the claim.
  \end{Proof}
\end{Lemma}
To show, that the $\Eto{-2\mu}$-terms do not contribute to the asymptotic
expansion of the trace in the end we additionally state the following very
similar Lemma:
\begin{Lemma}
  \label{lem:asymp-2}
  Using the same definitions as in \cref{lem:asymp-1} we have
  \begin{align*}
    \Int{x}{\Eto{-\mu((a-x) + (x+b))}\,x^n} &= \Eto{-\mu(a+b)}\frac{1}{n+1}
    \text{ and } \\
\Int{x}{\Eto{-\mu((a-x) + \Abs{x-b})}\,x^n} &= \Eto{-\mu(a+b)}H^n_{-2}(0) -
    \Eto{-\mu(a-b)}H^n_{-2}(b).
  \end{align*}
  \begin{Proof}
    The first integral gives
    \begin{equation*}
      \Int{x}{\Eto{-\mu((a-x)+(x+b))}x^n} = \Eto{-\mu(a+b)}\frac{1}{n+1}
    \end{equation*}
    For the upper part of the second integral we immediately get
    \begin{equation*}
      \Eto{-\mu(a-b)}\Integ[1]{b}{x}{x^n} =
      \Eto{-\mu(a-b)}\frac{1-b^{n+1}}{n+1},
    \end{equation*}
    the lower part is
    \begin{align*}
      \Eto{-\mu(a+b)}\Integ[b]{0}{x}{\Eto{2\mu x}x^n} &=
      \Eto{-\mu(a+b)}\left(-\Eto{2\mu b}H^n_{-2}(b) + H^n_{-2}(0)\right) \\
      &= -\Eto{-\mu(a-b)}H^n_{-2}(b) + \Eto{-\mu(a+b)}H^n_{-2}(0).
    \end{align*}
  \end{Proof}
\end{Lemma}
Now that we know how the kernels behave when they are composed we can prove
our last Main Theorem:
\begin{MainTheorem}
  The homogeneous terms in the asymptotic expansion of the resolvent trace of
  $\Delta_\lambda$ in the parameters $(\lambda,z)$ at the boundary $x = 0$ are
  rational functions in $V$, $W$ and their respective derivatives, at $0$. Using
  the order $\ord$ defined in \thref{def:order} we see that if $-n$ is the
  $\mu$-order of a term $A$ in the asymptotic expansion then we have $\ord A = n
  - 2$.
  \begin{Remark}
    This construction works \emph{ad verbatim} for $x = 1$ with the standard
    substitution $x\mapsto 1 - x$.
  \end{Remark}
  \begin{Remark}
    We see, that $\ord$ is shifted by $1$ in comparison to the interior
    parametrix. However we will already have evaluated the trace integral for
    the boundary terms, and it seems logical that if differentiating by $x$
    raises the order by $1$, integration should lower it by 1, giving the same
    overall order in $V$, $W$ and their derivatives both in the interior and at
    the boundary of the interval.
  \end{Remark}
  \begin{Proof}
    We will reuse our $R_\sigma$-notation and derive the $\mu$- as well as the
    $V$/$W$-order ($\ord$) of $\Tr R_\sigma$. To simplify notation in the
    following exposition we also define for $n\in\mathbb{Z}$ and
    $C\in\mathbb{C}$
    \begin{equation}
      \ord_\mu C\mu^n := n
    \end{equation}
    and for $A$ being a polynomial in two parameters $a$,$b$
    \begin{equation}
      \ord_x A(a,b) := \deg A.
    \end{equation}
    The idea is to show that first of all with every composition of a kernel
    the overall $\mu$-order of the corresponding trace increases and that this
    increase directly depends directly on the differentiation order.

    The final $\mu$-order of the trace of a term in the kernel $A(a,b) :=
    (2\mu)^{-m}\Eto{-\mu(a+b)}b^n$ (i.e.\ $\ord_x A = n$ and $\ord_\mu A = m$)
    can be calculated using Watson's lemma to be
    \begin{equation}
      \Int{x}{A(x,x)} = (2\mu)^{-m}\Int{x}{\Eto{-2\mu x}x^n} \SimMu
      \frac{n!}{(2\mu)^{n+m+1}} = O(\mu^{n+m+1}).
    \end{equation}
    To find the final $\mu$-order of a term we thus have to keep track of of the
    difference $\delta := \ord_x A - \ord_\mu A$ for all of its summands.

    Now let $\sigma\in\{+,-\}^j$ for $j\in\mathbb{N}$ with $\sigma_1 = +$. In
    each kernel composition we choose one summand of the Taylor polynomial of
    $f_\lambda$ and see how the difference $\delta$ behaves.  Independent of the
    actual composition we get an additional $(2\mu)^{-1}$-term from the
    prefactor $C_\sigma$ which lowers $\ord_\mu$ by $1$. Now we choose a term in
    the Taylor polynomial of $f_\lambda$ of differentiation order $n$, i.e.\ 
    \begin{equation*}
      \frac{\lambda^2 V^{(n)}(0)}{n!} x^n \quad \text{ or } \quad
      \frac{W^{(n)}(0)}{n!}x^n.
    \end{equation*}
    In the first case the choice alone raises $\ord_\mu$ of the result by $2$
    because of the $\lambda^2$ in the numerator. The behaviour of the
    composition can be read of from \cref{lem:asymp-1}:
    
    If we are in the i$^{\text{th}}$ composition and $\sigma_i = +$ we refer to
    \cref{eqn:poly-1}. We ignore all $\Eto{-2\mu}$-terms for now, we will show
    later that they don't contribute. The introduced factor to $\Eto{-\mu(a+b)}$
    is $H^n_2(0)$ in this case which is
    \begin{equation*}
      H^n_2(0) = n! \sum_{m=0}^n \frac{0^m}{m!(2\mu)^{n+1-m}} =
      \frac{n!}{(2\mu)^{n+1}},
    \end{equation*}
    thus this lowers $\ord_\mu$ by $n+1$ and leaves $\ord_x$ unchanged.

    For $\sigma_i = -$ we get from \cref{eqn:poly-2} that the contribution is a
    bit more complicated, in this case the two factors to $\Eto{-\mu(a+b)}$ are
    $\frac{b^{n+1}}{n+1}$, which raises $\ord_x$ by $n+1$, and
    \begin{equation*}
      H^n_2(b) = n! \sum_{m=0}^n \frac{b^m}{m!(2\mu)^{n+1-m}}
    \end{equation*}
    for which each summand has $\delta = n+1$.

    Thus we see that on each kernel composition we have the following overall
    behaviour of the difference $\delta$: 
    \begin{enumerate}
      \item The composition as such lowers the difference by $1$ from the
        prefactor
      \item Each $W^{(n)}(0)$-term lowers the difference by $n+1$
      \item Each $\lambda^2 V^{(n)}(0)$-term lowers the difference by $n-1$ and
        appears only with $n > 0$ (i.e.\ it never raises the difference)
    \end{enumerate}
    Combining the last two with the first one we finally have
    \begin{enumerate}
      \item Each $W^{(n)}(0)$-term lowers the difference by $n+2$
      \item Each $\lambda^2 V^{(n)}(0)$-term lowers the difference by $n$ and
        appears only with $n > 0$
    \end{enumerate}
    Now we only need to note that we start of with a term of $(2\mu)^{-1}$,
    i.e.\ we begin with $\ord_x = 0$ and $\ord_\mu = -1$ and take into account
    that the trace evaluation subtracts another $1$ of the $\mu$-order to see
    that we have (for the case $\sigma_1 = +$) proven the claim that each term
    $A$ in the final asymptotic expansion of $\mu$-order $-n$ has $\ord A = n -
    2$. We note that all of this also holds for the Lagrange rest term in the
    Taylor polynomial (\cref{eqn:taylor-of-f}), i.e.\ by choosing the
    polynomial order high enough we see that the rest term does not contribute
    to the asymptotic components.

    To cover the cases with $\sigma_1 = -$ we note that there is always
    $i\in\mathbb{N}$ such that $\sigma_i = +$. We can thus start of the whole
    process described before from one such $i$ which will leave us with a term
    of the form $\Eto{-\mu(a+b)}P(b)$ at the end of the composition chain. We
    can then turn the integral around symbolically and employ the same process
    as before. Note that we have of course count all parameters appearing in the
    polynomial into $\ord_x$.

    The contributions of the $\Eto{-\mu(a-b)}$-terms in
    \cref{eqn:poly-1,eqn:poly-2} are completely damped away by the accompanying
    $\Eto{-2\mu}$ factor. This also doesn't change in the iteration of the
    process, as shown in \thref{lem:asymp-2}, since the overall structure
    $\Eto{-\mu}\Eto{-\mu(a+b)}$ or $\Eto{-\mu}\Eto{-\mu(a-b)}$ is invariant
    under the composition of the kernels $k_{R_+}$ and $k_{R_-}$ from the right.
  \end{Proof}
\end{MainTheorem}
The constructive part of the proof, together with \thref{lem:asymp-1}, can
easily be extended to a program that calculates all terms in the asymptotic
expansion.
