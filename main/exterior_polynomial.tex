\subsection{Rationality of the homogeneous components}
We have seen in the preceding subsection that like in the interior case the
first homogeneous components of the multi-parametric expansion of
$\Tr(\Delta_\lambda + z^2)^{-1}$ are rational functions in $V$, $W$ and their
derivatives at $x = 0$. We want to prove that this also holds for all higher
orders in the asymptotic expansion. In particular we want to show, that we
always get a polynomial that is divided by integral powers of $\mu = \lambda^2
V(0) + W(0)$.

We know, that $f_\lambda$ is analytic in $x$ near the boundary since it is a
composition of the analytic functions $V$ and $W$, thus we can write with
$N\in\mathbb{N}$ and $\xi\in(0,1)$
\begin{align*}
  f_\lambda(x) &= \sum_{n=0}^{N-1} \frac{f_\lambda^{(n)}(0)}{n!} x^n + R_N(x)
  \text{ with} \\ R_N(x) &= \frac{f^{(N)}(\xi)}{N!} x^{N}.
\end{align*}
It thus suffices to show that if we substitute each $f_\lambda$ in the
convolution of our kernels by a function $x^{n}$ for some $n\in\mathbb{N}$ (that
may be different for each substitution) we end up with a finite asymptotic sum.
We will do this recursively by looking at suitable generalisations of the
kernels.

Essentially, after splitting up the integrals, we will just evaluate the
them directly for $k=0$ and use the following formula otherwise, making use
of $n\in\mathbb{N}$,
\begin{equation}
  \Integ[b]{a}{x}{\Eto{-\mu kx}x^n}
    = -n!\left(\Eto{-\mu kb}\sum_{m=0}^n \frac{b^m}{m!(\mu k)^{n+1-m}} -
              \Eto{-\mu ka}\sum_{m=0}^n \frac{a^m}{m!(\mu k)^{n+1-m}}\right),
\end{equation}
which follows directly by the substitution $x \mapsto \mu kx$ from the formula
for the (upper) incomplete gamma function given (and proven) in the
\cref{app:gamma} (\thref{lem:incomplete-gamma}). Since we need it on multiple
occasions we define for $k\ne 0$ the shortcut
\begin{equation*}
  H^n_k(x) := n!\sum_{m=0}^n \frac{x^m}{m!(\mu k)^{n+1-m}}
\end{equation*}
such that we can write
\begin{equation}
  \label{eqn:ex-integ}
  \Integ[b]{a}{x}{\Eto{-\mu kx}x^n} = -\Eto{-\mu kb}H^n_k(b) + \Eto{-\mu
  ka}H^n_k(a).
\end{equation}
Note that in this context we will make use of the convention $0^0 := 1$ to
deal with the special case of $a = 0$ or $b = 0$.

% +- und ++ reichen aus, weil wir herumtauschen können
The integral part of the proof are the following calculations, that aim to
describe how the kernels behave on concatenation, i.e.\ we want to investigate
the kernels (see \cref{eqn:kernel-plus,eqn:kernel-minus})
\begin{align*}
  k_{++} &= \Int{x}{\Eto{-\mu((a + x) + (x + b))}f_\lambda(x)} \\
  k_{+-} &= \Int{x}{\Eto{-\mu((a + x) + \Abs{x - b})}f_\lambda(x)}
\end{align*}
It will suffice to use a Taylor polynomial of $f_\lambda(x)$, since we can
estimate the $\mu$-order of all combinations involving the residue term $R$ in
the same way as we estimate the interesting terms. The proof idea is to show
that if we start with $R_+$ all kernel concatenations in the given manner that
we apply from the right result in a kernel of the form $\Eto{-\mu(a+b)}p(b)$ for
some polynomial $p$ and is thus again similar to $R_+$. All terms that are not
of this form do not contribute to the final asymptotic expansion.
\begin{Lemma}
  \label{lem:asymp-1}
  For all $n\in\mathbb{N}_0$, $a,b\in[0,1]$ and $\mu\in\mathbb{R}$ we
  have
  \begin{align}
    \label{eqn:poly-1}
    \Int{x}{\Eto{-\mu((a + x) + (x + b))} x^n} &= \Eto{-\mu(a+b)}H_2^n(0) -
    \Eto{-\mu(a-b)}\Eto{-2\mu}H^n_2(1) \\
    \label{eqn:poly-2}
    \Int{x}{\Eto{-\mu((a + x) + \Abs{x - b})}x^n} &=
    \Eto{-\mu(a+b)}\left(\frac{b^{n+1}}{n+1} + H_2^n(b)\right) +
    \Eto{-\mu(a-b)}\Eto{-2\mu}H^n_2(1).
  \end{align}
  \begin{Remark}
    In particular, discarding terms that won't contribute in the asymptotic limit
    $\mu\to\infty$, the result is of the form $\Eto{-\mu(a+b)}g(b)$, with $g$
    being a polynomial.
  \end{Remark}
  \begin{Proof}
    The formulas follow by direct application of \eqref{eqn:ex-integ}:
    \begin{align*}
      \Int{x}{\Eto{-\mu((a+x)+(b+x))}x^n} &= \Eto{-\mu(a+b)}\Int{x}{\Eto{-2\mu x}x^n} \\
        &= \Eto{-\mu(a+b)}\left(-\Eto{-2\mu}H^n_2(1) + H^n_2(0)\right) \\
        &= \Eto{-\mu(a+b)} H^n_2(0) - \Eto{-2\mu} H^n_2(1)
    \end{align*}
    For \eqref{eqn:poly-2} we split the integral to account for the modulus in
    the exponent:
    \begin{align*}
      \Int{x}{\Eto{-\mu( (a+x) + \Abs{x - b})}f(x)} &=
      \Eto{-\mu(a+b)}\Integ[b]{0}{x}{\Eto{-\mu(x-x)}f(x)} \\ &+
      \Eto{-\mu(a-b)}\Integ[1]{b}{x}{\Eto{-\mu(x+x)}f(x)}
    \end{align*}
    For the lower term we get
    \begin{align*}
      \Eto{-\mu(a+b)}\Integ[b]{0}{x}{x^n} =
      \Eto{-\mu(a+b)}\left(\frac{b^{n+1}}{n+1} - 0\right),
    \end{align*}
    which gives the first part of the result. For the upper integration we
    finally have
    \begin{align*}
      \Eto{-\mu(a-b)}\Integ[1]{b}{x}{\Eto{-2\mu x}f(x)} &=
      \Eto{-\mu(a-b)}\left(-\Eto{-2\mu}H^n_2(1) + \Eto{-2\mu b}H^n_2(b)\right) \\
      &= \Eto{-\mu(a+b)}H^n_2(b) - \Eto{-\mu(a-b)}\Eto{-2\mu}H^n_2(1),
    \end{align*}
    which proves the claim.
  \end{Proof}
\end{Lemma}
To show, that the $\Eto{-2\mu}$-terms in the end do not contribute to the
asymptotic expansion of the trace we additionally state the following very
similar Lemma:
\begin{Lemma}
  \label{lem:asymp-2}
  Using the same definitions as in \thref{lem:asymp-1} we have
  \begin{align*}
    \Int{x}{\Eto{-\mu((a-x) + (x+b))}x^n} &= \Eto{-\mu(a+b)}\frac{1}{n+1} \\
\Int{x}{\Eto{-\mu((a-x) + \Abs{x-b})}x^n} &= \Eto{-\mu(a+b)}H^n_{-2}(0) -
    \Eto{-\mu(a-b)}H^n_{-2}(b)
  \end{align*}
  \begin{Proof}
    The first integral gives
    \begin{align*}
      \Int{x}{\Eto{-\mu((a-x)+(x+b))}x^n} &= \Eto{-\mu(a+b)}\frac{1}{n+1}
    \end{align*}
    For the upper part of the second integral we immediately get
    \begin{align*}
      \Eto{-\mu(a-b)}\Integ[1]{b}{x}{x^n} =
      \Eto{-\mu(a-b)}\frac{1-b^{n+1}}{n+1},
    \end{align*}
    the lower part is
    \begin{align*}
      \Eto{-\mu(a+b)}\Integ[b]{0}{x}{\Eto{2\mu x}x^n} &=
      \Eto{-\mu(a+b)}\left(-\Eto{2\mu b}H^n_{-2}(b) + H^n_{-2}(0)\right) \\
      &= -\Eto{-\mu(a-b)}H^n_{-2}(b) + \Eto{-\mu(a+b)}H^n_{-2}(0).
    \end{align*}
  \end{Proof}
\end{Lemma}
Now that we know how the kernels behave when they are concatenated we can prove
our last Main Theorem:
\begin{MainTheorem}
  The homogeneous terms in the asymptotic expansion of the
  resolvent-trace of $\Delta_\lambda$ in the parameters $(\lambda,z)$ at the
  boundary $x = 0$ are rational functions in $V$, $W$ and their respective
  derivatives, at $0$. Using the order $\ord$ defined in \thref{def:order} we
  see that if $-n$ is the $\mu$-order of a term $A$ in the asymptotic expansion
  then we have $\ord A = n + 1$.
  \begin{Remark}
    This construction works \emph{ad verbatim} for $x=1$ with the standard
    substitution $x\mapsto 1 - x$.
  \end{Remark}
  \begin{Remark}
    We see, that $\ord$ is shifted by 1 in comparison to the interior
    parametrix. However we will already have evaluated the trace integral for
    the boundary terms, and it seems logical that if differentiating by $x$
    raises the order by $1$, integration should lower it by 1, giving the same
    overall order in $V$, $W$ and their derivatives both in the interior and at
    the boundary of the interval.
  \end{Remark}
  \begin{Proof}
    We will reuse our $R_\sigma$-notation and derive the $\mu$- as well as the
    $V$/$W$-order ($\ord$) of $\Tr R_\sigma$.

    First let us consider a result term as it appears in \thref{lem:asymp-1},
    $\Eto{-\mu(a+b)}H^n_2(b)$. If we evaluate the asymptotics of the trace of
    such a term applying Watson's Lemma on the monomials we get (setting $a = b
    =: x$)
    \begin{align}
      \Int{x}{\Eto{-2\mu x}H^n_2(x)}
      &= -n!\sum_{m=0}^n \frac{\Int{x}{\Eto{-2\mu x} x^m}}{m!(2\mu)^{n+1-m}} \\
      &\SimMu -n!\sum_{m=0}^n \frac{m! / (2\mu)^{m+1}}{m!(2\mu)^{n+1-m}} \\
      &= -n!\sum_{m=0}^n \frac{1}{(2\mu)^{n+2}} = -\frac{(n+1)!}{(2\mu)^{n+2}}
    \end{align}
    Coincidentally this is exactly the same contribution that $-x^{n+1}$ would
    give, i.e.
    \begin{align*}
      \Int{x}{\Eto{-2\mu x}H^n_2(x)} \SimMu -\Int{x}{\Eto{-2\mu x}x^{n+1}}.
    \end{align*}
    Since also $H^n_2(0) = -n! 0^0 / (2\mu)^{n+1} = -n! (2\mu)^{-n-1}$ and thus
    \begin{align*}
      \Int{x}{\Eto{-2\mu x}H^n_2(0)} \SimMu -n! (2\mu)^{-n-2},
    \end{align*}
    all terms (apart from the $\Eto{-\mu(a-b)}$ that we are dealing with at the
    end of the proof) coming from $x^n$ have the same definite $\mu$-order.

    % Den Prozess genauer erklären, wir schieben x^n dazwischen, das hebt aber
    % nur die Gesamtordnung
    The process can be iterated, if $\sigma_1 = +$ we can contract all following
    kernel concatenations to $\Eto{-\mu(a+b)} P(x)$ for some polynomial $P$ that
    by construction still has the property, that $\mu$-order and $x$-order add
    up, where the common sum 

    We see that the $x$-order and the $\mu$-order add up in the trace
    evaluation, i.e.\ $x^n \mapsto \mu^{-(n+1)}$. Since every term in the Taylor
    expansion of $f_\lambda$ is of the form $x^n f^{(n)}_\lambda(0)$ each
    derivative lowers the $\mu$-order by 1, which establishes the relative
    connection of the $\mu$-order and $\ord$. To verify the absolute order we
    observe, that for $V$ only terms of higher derivation order than $1$ occur
    since we are actually using $\tilde V(x) = V(x) - V(0)$. Also every kernel
    concatenation is accompanied by an additional $(2\mu)^{-1}$ from $C_\sigma$,
    thus the final $\mu$-order of a contribution of $\lambda^2 V^{(n)}(0)$ is
    \begin{align*}
      (2\mu)^{-1} x^n \lambda^2 V^{(n)}(0) \mapsto O(\mu^{-1-(n+1)+2}) =
      O(\mu^{-n})
    \end{align*}
    For $W$ we get nearly the same, but $W(0)$ may appear and the order is not
    modified by a $\lambda^2$-factor, thus we have
    \begin{align*}
      (2\mu)^{-1} x^n W^{(n)}(0) \mapsto O(\mu^{-1-(n+1)}) = O(\mu^{-(n+2)}
    \end{align*}
    Combining all those facts we see, that 
    \begin{enumerate}
      \item We start with a $\mu$-order of -1
      \item Each $\lambda^2 V^{(n)}(0)$-term lowers the $\mu$-order by $n$ and
        appears only with $n > 0$
      \item Each $W^{(n)}(0)$-term lowers the $\mu$-order by $n+2$
    \end{enumerate}
    Since those are the only options we have in each concatenation step we see,
    that the final $\mu$-order of $\Tr R_\sigma$ is at most minus the length of
    the tuple $\sigma$ minus 1 ($-\#{\sigma}-1$), which is always choosing
    $\lambda^2 V'(0)$. Since we have shown that every $\Tr R_\sigma$ contributes
    to only a finite number of homogeneous components and the contributions are
    polynomials in $V$, $W$ and their derivatives with the given order relation
    we have proven the statement as well as the remark as long as all
    contributions stem from the $\Eto{-\mu(a+b)}H^n_2(x)$-terms.

    The contributions of the $\Eto{-\mu(a-b)}$-term in \cref{eqn:poly-1} are
    completely damped away by the accompanying $\Eto{-2\mu}$ factor. This also
    doesn't change in the iteration of the process, as proven in
    \thref{lem:asymp-2}.
  \end{Proof}
\end{MainTheorem}
The constructive part of the proof, together with \thref{lem:asymp-1}, can
easily be extended to a program that calculates all terms in the asymptotic
expansion.
