\subsection{Rationality of the homogeneous components}
Let
\begin{align*}
	f_\lambda(x) := \lambda^2 V(x) + W(x).
\end{align*}

We know, that $f_\lambda$ is analytic in $x$, thus we can write
\begin{align*}
  f_\lambda(x) &= \sum_{n=0}^{N-1} \frac{f_\lambda^{(n)}(0)}{n!} x^n + R_N(x) \text{
  with} \\
  R_N(x) &= \frac{f^{(N)}(\xi)}{N!} x^{N}
\end{align*}

We want to prove that the homogeneous components of the asymptotic expansion of
the exterior parametrix are again, like they were in the interior of the
interval, rational functions in $V$, $W$ and their derivatives, to be precise in
their values at 0. In particular we want to show, that we always get a
polynomial that is divided by integral powers of $\mu = \lambda^2 V(0) + W(0)$.
It suffices to show that if we substitute each $f_\lambda$ in the convolution of
our kernels by a function $x^{n}$ for some $n\in\mathbb{N}$ (that may be
different for each substitution) we end up with a finite asymptotic sum.

We will do this recursively by looking at suitable generalisations of the
kernels with $f_\lambda$ being a function of the following kind:
\begin{Definition}[Step Polynomial]
  Let $f_>, f_<\in K[x]$ be polynomials. Then we define the function $f_a\colon
  K\setminus\{a\}\to K$ as 
  \begin{align*}
    f_a(x) := \begin{cases}
      f_>(x) & x > a \\
      f_<(x) & x < a
    \end{cases}
  \end{align*}
  We define the degree of $f$ as the 2-tuple
  \begin{align*}
    \deg f := \left(\deg f_<, \deg f_>\right).
  \end{align*}
\end{Definition}

The integral part of the proof is the following Lemma:
\begin{Lemma}
  \label{lem:asymp-1}
  Let $f_b, g_b$ be step polynomials and for odd integers $k_a, k_b, K_a, K_b$
  let $e$ be a step function defined as:
  \begin{align*}
    g_{k,K}(x, b) = \begin{cases}
      k_a x + k_b b & x < b \\
      K_a x + K_b b & x > b
    \end{cases}
  \end{align*}
  % TODO: Polynomiale und \mu-Ordnung der Terme angeben
  Then we get the following asymptotics for (step) polynomials $f^A, f^{B_1}_b,
  f^{B_2}_b$ and $f^C$, each being of order $O(f_b)$ as $\mu\to\infty$ and of
  the same (or element-wise lesser) degree as $f_b$:
  \begin{align*}
    \label{eqn:poly-1}
    \newtag
    \Int{x}{\Eto{-\mu((a + x) + g_{(k_a,k_b),(K_a,K_b)}(x, b))} &f_b(x)} \\
         =&\ \Eto{-\mu(a+k_b b)}f^0_b(a) + \Eto{-\mu(a+K_b b + 1 + K_a)}f^1(a) \\
         +&\ \Eto{-\mu g_{(2+k_a, k_b),(1,1+k_a+k_b)}(a,b)} f^A(a) \\
         +&\ \Eto{-\mu g_{(2+k_a, k_b),(2+K_a,K_b)}(a,b)} f^{B_1}_b(a) \\
         +&\ \Eto{-\mu g_{(1,1+k_b+k_a),(1,1+K_a+K_b)}(a,b)} f^{B_2}_b(a) \\
         +&\ \Eto{-\mu g_{(1,1+K_a+K_b),(2+K_a,K_b)}(a,b)} f^C(a)
  \end{align*}
  And also (for a different set of $f^{*}_b$ in general):
  \begin{align*}
    \label{eqn:poly-2}
    \newtag
    \Int{x}{\Eto{-\mu (\Abs{a - x} + g_{(k_a,k_b),(K_a,K_b)}(x, b))} &f_b(x)} \\
        =&\ \Eto{-\mu(a+k_bb)}f^0_b(a) + \Eto{-\mu(-a+K_bb+1+K_a)}f^1(a) \\
        +&\ \Eto{-\mu g_{(-1,1+K_a+K_b),(1,k_b+k_a-1)}(a,b)} f^A(a) \\
         +&\ \Eto{-\mu g_{(-1,1+k_a+k_b),(K_a, K_b)}(a,b)} f^{B_1}_b(a) \\
         +&\ \Eto{-\mu g_{(k_a,k_b),(1,K_a+K_b-1)}(a,b)} f^{B_2}_b(a) \\
         +&\ \Eto{-\mu g_{(-1,1+k_a+k_b),(K_a, K_b)}} f^C(a)
  \end{align*}
  In particular, the exponents are independent of $f_b$.

  \begin{Proof}
    Essentially we will just use this formula for $k \neq 0$
    \begin{align}
      \Integ[b]{a}{x}{\Eto{-kx}x^n}
        = n!\left(\Eto{-ka}\sum_{m=0}^n \frac{a^m}{m!k^{n+1-m}} -
                  \Eto{-kb}\sum_{m=0}^n \frac{b^m}{m!k^{n+1-m}}\right),
    \end{align}
    which follows directly by the substitution $x \mapsto kx$ from the formula
    for the (upper) Incomplete Gamma Function given (and proven) in the
    appendix. The sign is intentional, and we use in this context the convention
    $0^0 := 1$ to deal with the special case of $a = 0$ or $b = 0$. For $k = 0$
    we have $\frac{b^{n+1} - a^{n+1}}{n+1}$.
    % TODO

    The first integral will be done in detail and we start of by splitting it up
    as much as needed. Let $m := \min(a,b), M := \max(a,b)$. Then we can write
    the integral (using the given abbreviation) as
    \begin{align*}
      h(x) &:= \Eto{-\mu(a+x+g_{(k_a,k_b),(K_a,K_b)}(x, b))}f_b(x) \\
      \Int{x}{h(x)} &= A + B + C \quad \text{with} \\
                 A :&= \Integ[m]{0}{x}{h(x)} \\
                 B :&= \Integ[M]{m}{x}{h(x)} \\
                 C :&= \Integ[1]{M}{x}{h(x)}.
    \end{align*}

    Now we will determine each of the integrals for monomials first.
    
    For $k_a = -1$ we have $A = \Eto{-\mu (a+k_b b)} \frac{m^{n+1}}{n+1}$,
    otherwise we can use the formula given above:
    \begin{align*}
      A &= \Integ[m]{0}{x}{\Eto{-\mu(a+x+k_a x+k_b b)}x^n}
         = \Eto{-\mu(a+k_b b)} \Integ[m]{0}{x}{\Eto{-(1+k_a)\mu}x^n} \\
        &= \Eto{-\mu(a+k_b b)} n! \left(
         % TODO Term für untere Grenze oben einfügen!
         \frac{1}{((1+k_a)\mu)^{n+1}}
         - \Eto{-\mu(1+k_a)m}\sum_{l=0}^n \frac{m^l}{l! ((1+k_a)\mu)^{n+1-l}}
        \right) \\
      &= \Eto{-\mu(a+k_b b)} \frac{n!}{((1+k_a)\mu)^{n+1}} +
        \begin{cases}
        \Eto{-\mu((2+k_a) a+k_b b)} n! \sum_{l=0}^n
        \frac{a^l}{l!((1+k_a)\mu)^{n+1-l}}
          & a < b \\
          \Eto{-\mu(a + (1+k_a+k_b)b)} n! \sum_{l=0}^n
          \frac{b^l}{l!((1+k_a)\mu)^{n+1-k}}
          & a > b
      \end{cases}
    \end{align*}
    The non-exponential part in the case differentiation is the resulting
    $f^A(a)$, which is indeed of degree $(n,1)$ and even $O(\mu^{-1})$ for
    $k_a\neq -1$, but definitely $O(1)$. We have also found terms that will form
    $\Eto{-\mu(a+k_b b)} f^0_b$, being the only contribution of $A$ for $k_a
    = -1$ and additionally the contribution from the lower end of the interval.

    For $B$ the calculation is a bit simpler, as we don't have to worry about
    the special case at 0 but only have to catch the possible $-1$'s in $k_a$
    and $K_a$:
    \begin{align*}
      B = \Eto{-\mu(1 + k_b b)} \frac{1}{n+1} \begin{cases}
        b^{n+1} - a^{n+1} & a<b \wedge k_a = -1 \\
        a^{n+1} - b^{n+1} & a>b \wedge K_a = -1,
      \end{cases}
    \end{align*}
    thus this again contributes to $f^0_b$.

    Otherwise we have for $a < b$
    \begin{align*}
      B &= \Eto{-\mu(a + k_b b)} \Integ[b]{a}{x}{\Eto{-\mu(1+k_a)}x^n}
        &= \Eto{-\mu(a + (1+k_a+k_b)b)} f^{B_2}_<(a)
         + \Eto{-\mu((2+k_a)a +k_b b)} f^{B_1}_<(a),
    \end{align*}
    and for $a > b$ (where we have $k\mapsto K$ and $a$ and $b$ switch roles)
    \begin{align*}
      B &= \Eto{-\mu(a + K_b b)} \Integ[a]{b}{x}{\Eto{-\mu(1+K_a)}x^n}
        &= \Eto{-\mu((2+K_a)a + K_b b)} f^{B_1}_>(a)
         + \Eto{-\mu(a + (1+K_a+K_b)b)} f^{B_2}_>(a).
    \end{align*}

    Now we do the same for $C$, paying attention to the upper end of the
    interval. For $K_a = -1$ we have again a contribution to $f^0_b$:
    \begin{align*}
      C = \Eto{-\mu(a+k_b b)} \frac{1}{n+1} \begin{cases}
        1 - b^{n+1} & a < b \\
        1 - a^{n+1} & a > b
      \end{cases}.
    \end{align*}

    For $K_a \neq -1$ we get:
    \begin{align*}
      C = \Eto{-\mu(a+K_b b)} n! \left(
      \Eto{-\mu(1+K_a)M} \sum_{l=0}^n \frac{M^l}{l!((1+K_a)\mu)^{n+1-l}}
      - \Eto{-\mu(1+K_a)} \sum_{l=0}^n \frac{1}{l!((1+K_a)\mu)^{n+1-l}}
      \right)
    \end{align*}
    Inserting $M=a$ or $M=b$ we get the corresponding contributions for $f^C_b$
    and the last term indeed is the $f^1$. Since we have now shown
    \eqref{eqn:poly-1} for a monomial and the integrals are linear in the
    function we can easily extend it to (step) polynomials by linear
    combination.
    
    which proves the first equation. The
    second one follows the same way if whilst case differentiating we insert the
    proper differences instead of $\Abs{a-x}$, i.e.\ $a-x$ in $A$ and $B_1$ and
    $x - a$ in $B_2$ and $C$. Also the first special case (for $f^0$ in $A$) is
    not $k_a = -1$ but $k_a = 1$ here, the second one is the same however.
    % TODO: Auswalzen, das geht so nicht! Zumindest am Ende muss hier noch etwas
    % mehr stehen.
  \end{Proof}
\end{Lemma}

Now that we know better how the kernels behave when they are concatenated we can
prove our last Main Theorem:
\begin{MainTheorem}
  The homogeneous terms in the  common asymptotic expansion of the
  resolvent-trace of $\Delta_\lambda$ at the boundary $x = 0$ is a rational
  function in $V$, $W$ and their respective derivatives at $0$.
  \begin{Remark}
    This construction works \emph{ad verbatim} for $x=1$ with the standard
    substitution $x\mapsto 1 - x$.
  \end{Remark}
  \begin{Proof}
    As explained in the beginning of this section we consider the Taylor
    expansion of $f_\lambda$ at $x=0$ with the remainder being in Lagrange form,
    which is a polynomial of degree $N$ for some $N\in\mathbb{N}$.

    Now let $R^{(n)}$ be again given by its components $R^{(n)}_{\sigma}$. We
    will now iterate over $\sigma$ using the formulas given in the Lemma above
    and gather the corresponding $k_a,k_b$ and $K_a,K_b$.

    For $\sigma_{n} = +$ we have as starting values $k_a = k_b = K_a = K_b = 1$
    and for $\sigma_{n} = -$ we have $k_a = K_a = -1$ and $k_b = K_b = 1$,
    modeling the modulus in the exponent. In each step of the iteration we use
    either \eqref{eqn:poly-1} (for $\sigma_i = +$) or \eqref{eqn:poly-2} (for
    $\sigma_i = -$) and multiply the resulting polynomials by the
    Taylor polynomial of $f_{\lambda}$, which, of course, doesn't change the
    general form of them used in further calculations. However we have to keep
    track of the polynomial degree, because this will in the end determine the
    $\mu$-order of the corresponding term. To prove, that we indeed get a proper
    asymptotic expansion with the homogeneous terms being rational functions we
    have to show, that only a finite number of terms contribute to a specific
    order. It suffices to show, that the rest term of the Taylor expansion only
    contributes up to order of $-n_N$ for some $n$, with $\lim_{N\to\infty}
    -n_N = -\infty$ and $n_N$ being monotonous.

    Finally we want to insert $a = b$ and integrate, for which to be
    well-defined we have to show, that first of all the step polynomials whose
    integrals contribute are actually polynomials, i.e.\ $f_< \equiv f_>$ on all
    of $[0,1]$ and $k_a + k_b = K_a + K_b$. Furthermore we need to show, for
    this being an asymptotic expansion, that $k_a + k_b \geq 0$, since only in
    this case the result of the integral is properly damped and by Watson's
    Lemma we have only contributions from $x = 0$.

    First we deal with the terms consisting only of $f^A$, $f^{B_*}$ and $f^C$.
    Let $k_a + k_b = K_a + K_b$ (this is given for our initial values $k =
    (1,1), K = (1,1)$ and $k = (-1,1), K=(-1,1)$). Then we have for $\sigma_i =
    +$ and for the terms belonging to $f^A$, $f^{B_*}$ and $f_C$
    \begin{align*}
      k^i_a + k^i_b &= 2 + k^{i-1}_a + k^{i-1}_b \\
      K^i_a + k^i_b &= 2 + K^{i-1}_a + K^{i-1}_b,
    \end{align*}
    and for $\sigma_i = -$
    \begin{align*}
      k^i_a + k^i_b &= k^{i-1}_a + k^{i-1}_b \\
      K^i_a + K^i_a &= K^{i-1}_a + K^{i-1}_b.
    \end{align*}
    Thus if we start with $k_a + k_b = K_a + K_b$ this will be conserved and if
    we have at least one $\sigma_i = +$ (which is given in the construction) the
    sum of the exponents will be $> 0$, thus in the trace evaluation we have a
    negative exponent. Also, as shown before, the polynomials preserve their
    degree during this step. We have to note, that the $f^{B_*}_b$-terms,
    which are step polynomials, vanish in the final step for $a = b$, as they
    correspond to the integrals $\int_a^b$ and $\int_b^a$ respectively, thus
    the step polynomials are of no concern.
    % Polynomial orders?

    The terms that contain $f^1$ are especially easy to cope with. We can see in
    the formulas for the exponents, that there is no way for $k_a, k_b$ or $K_a,
    K_b$ to \emph{both} become negative if they weren't already before. This is
    also the reason, why none of them can ever be less than $-1$.
    % TODO: Beweisen, dass K, k >= -1 sind (ist kurz, fehlt hier aber!)
    For $\sigma_i = \pm$, $f^1 \neq 0$ only for $K^{i-1}_a \neq -1 \Rightarrow
    K^{i-1}_a \geq 0$. Now this term has a factor of $\Eto{+\mu (1+K_a)}$, since
    the terms that are multiplied with this one are only of polynomial order
    they cannot cancel it and it does not contribute to the asymptotic expansion
    at all.

    Now we are left with handling $f^0_b$. We see immediately, that the term
    coming from $A$
    % TODO - Erster Term hat \mu-Ordnung n+1, also für den Restterm N+1
    %       -> trägt nicht bei
    %      - Terme aus B tragen nicht bei für a=b da sie 0 sind
    %      - Für allgemeine a, b erhöht sich die Polynomordnung um n+1, also
    %        für den Restterm auch um N+1

    % TODO Watson's Lemma anwenden um zu zeigen, dass die \mu-Ordnung überall
    %      kleiner als 0 ist, obwohl das auch in Prop2.2 drinsteckt!

    Now that all of this is done the proof is finished, as we already know that
    only a finite number of terms $R^{(n)}_\sigma$ contribute to a specific
    order and we have shown, that each $R^{(n)}_\sigma$ does also only
    contribute finitely many terms. Furthermore we have shown, that the
    contributions of the rest term in the Taylor expansion, which is of order
    $N$ in $x$, are of order $-N-1$ in $\mu$, monotonically decreasing with
    limit $-\infty$. Thus each homogeneous component is made of a finite number
    of products of coefficients of the Taylor expansion of $f_\lambda$, which
    are up to a constant the derivatives of $V$ and $W$ at $x=0$.
  \end{Proof}
\end{MainTheorem}

The constructive part of the proof, together with \thref{lem:asymp-1}, can be
extended to a program that calculates all terms in the asymptotic expansion.
