\section{Compilation and Application}
Since we saw in the previous sections that the homogeneous components of the
trace expansion of the resolvent are of the form $\frac{\lambda^n}{(\lambda^2 k
+ z^2)^m}$ (with $k$ being one of $V(0)$, $V(1)$ and $V(x)$) we get the
following formula for the coefficients of $\Tr{(\Delta_\lambda + z^2)}^{-2}$:
\begin{align}
  (-2z)^{-1} \partial_z \frac{\lambda^n}{(\lambda^2 k + z^2)^{m}}
    &= m \frac{\lambda^n}{(\lambda^2 k + z^2)^{m+1}}
\end{align}

Now that we have calculated the coefficients we only need to integrate the
homogeneous part with the logarithm, as given in the main theorem (\cite[Thm
1.4]{LV13}):
\begin{align*}
  \log\zdet\Delta - \Regsum_{\lambda=-\infty}^\infty \log\zdet\Delta_\lambda =
  -4 \Integ[\infty]{0}{\lambda}{h_2(1,\lambda)\ln(\lambda)}
\end{align*}

To do that we use the Mellin transform, since the derivative at $0$ of the
transformation integral (which can be calculated easier) is
\begin{align*}
    \left.\frac\partial{\partial r}\right|_{r=0}\Integ[\infty]{0}{x}{x^r f(x)}
    &= \Integ[\infty]{0}{x}{\ln{(x)} f(x)}
\end{align*}
if one of the integrals exist. We thus calculate again using our integration
formula from \thref{lem:beta_function_formula} with $x\mapsto \lambda$, $a \mapsto 1$,
$b \mapsto k \in\{V(0), V(1), V(x)\}$, $n \mapsto n + r$ and $m \mapsto
\tfrac{n+5}{2}$:
\begin{align*}
    \Integ[\infty]{0}{\lambda}{\frac{\lambda^{n + r}}{\left(1 +
    k\lambda^2\right)^{\frac{n+5}{2}}}} &=
      \frac{1}{2k^{n+r+1}} B\bigl(\tfrac{n+r+1}{2}, \tfrac{4-r}{2}\bigr)
\end{align*}
The beta function is symmetric, and its derivative is $\partial_a B(a,b) =
B(a,b) (\psi(a) - \psi(a+b))$, where $\psi(x) = \partial_x \ln\Gamma(x)$ is the
digamma function (see Formula~\ref{frm:beta_deriv}), so we get
\begin{align*}
  \Integ[\infty]{0}{\lambda}{
    \frac{\lambda^n}{(1 + k\lambda^2)^{\frac{n+5}{2}}}\ln\lambda
  } &=
      \frac{
        \psi\bigl(\frac{n+1}{2}\bigr) - 2\ln(k) + \gamma - 1
      }{
           (n+3)(n+1)k^{n+1}
      },
\end{align*}
where $\gamma$ is the Euler-Mascheroni constant. Noting that $n$ is always even
in our expansion we can evaluate $\psi\bigl(\tfrac{n+1}{2}\bigr)$ a bit further
by using the recurrence relation $\psi(x + 1) = \psi(x) + \tfrac{1}{x}$ and the
special value $\psi(\tfrac{1}{2}) = -\gamma - 2\ln 2$, which brings us to the
following explicit formula for the integral:
\begin{align}
  \Integ[\infty]{0}{\lambda}{
    \frac{\lambda^n}{(1 + k \lambda^2)^{\frac{n+5}{2}}}
    \ln\lambda
  } = \frac{1}{(n+1)(n+3)k^{n+1}}
  \Biggl( \sum_{i=1}^{n/2}\frac{2}{2i - 1} -1 - 2\ln(2k) \Biggr).
\end{align}

Now that we can calculate this we can give an explicit expression for the error
term as an integral over $x$:
\begin{align}
  \Integ[\infty]{0}{\lambda}{
    h_2(\lambda, 1) \ln\lambda
  }
  &=
  \label{eqn:interior_result}
  % Interior:
  \Int{x}{\Biggl(\frac{(1 + 2 \ln(2 V(x))) W(x)}{8 V(x)}
    - \frac{ (1 - 2 \ln(2 V(x)))V''(x)}{96 V(x)^{3}} \\
    &\hphantom{=\int_0^1\Biggl(}\mathbin{+} \frac{\left(\frac{5}{3} - 2\ln(2
    V(x))\right)\left(V'(x)\right)^{2}}{64 V(x)^{5}}\Biggr) } \nonumber \\
&\mathbin- 5\frac{(1 - 2\ln(2 V(1)))V'(1)}{192 V(1)^3}
+ 5\frac{(1 - 2 \ln(2 V(0)))V'(0)}{192 V(0)^3}
\label{eqn:exterior_result}
\end{align}
Here \eqref{eqn:interior_result} is the contribution of the interior terms,
while \eqref{eqn:exterior_result} is the contribution of the boundary. We have
seen, that the integrands coming from the interior parametrix stayed integrable
over $x$ during the whole process, as $V \geq \delta > 0$ since
$V\in\Cinf{[0,1]}$.
% TODO Richtig begründen!
