\section{Asymptotics of the exterior parametrix}
By using the summation formula derived in LV12 the existence and values of the
asymptotic components of the specific operator we're looking at can be
calculated directly. From Lemma bla we know, that we only have to consider $j\in
\{0, 1, 2\}$ since the requested asymptotic coefficient is of order $-5$.

In (%größtenteils)
accordance with LV12 we denote
\begin{align*}
    % TODO Anständig einführen im Setting, Unterschiede zum LV12-paper ausweisen
    % TODO Explizit darauf hinweisen, dass alle gewöhnlichen Funktionen als
    %      Multiplikationsoperator zu verstehen sind.
    % TODO Es ist gesteigert eklig, dass R_+ negatives Vorzeichen hat …
    R_- &:= \Eto{-\mu\left|x-y\right|} \\
    R_+ &:= -C(\mu,\theta)\Eto{-\mu(x+y)} \\
    R_\theta &:= \tfrac1{2\mu} (R_- + R_+) \\
    R_{(\sigma_1,\sigma_2,\ldots,\sigma_n)} &:= 
    \left(\tfrac1{2\mu}\right)^n R_{\sigma_1} \lambda(V,W) R_{\sigma_2} \dots
    \lambda(V,W) R_{\sigma_n} \\
    R^{(j)} &:= (-1)^j \psi \left( [R_\theta \lambda(V,W)]^j R_\theta -
    [R_\mathbb{R}\lambda(V,W)]^j R_\mathbb{R}\right) \phi \\
            &= (-1)^j
                \sum_{
                        \substack{
                         \sigma\in\{+,-\}^j \\
                         \sigma\neq(+,\ldots,+)
                         }
                        }
                     \psi R_\sigma
                \phi
    \label{<++>}
\end{align*}

% TODO

The calculation for $j=0$ is straightforward:
\begin{align}
    \Tr R^{(0)} =& \Integ[1]{0}{x}{k_\theta(x,x;\mu) - k_\mathbb{R}(x,x;\mu)} \\
    =& -\frac1{2\mu} C(\mu,\theta) \Integ[1]{0}{x}{\Eto{-2\mu x}} \\
%    =& -\frac1{2\mu} C(\mu,\theta) \left( \frac1{2\mu} - \frac{\Eto{-2\mu}}{2\mu} \right) \\
    \SimAs{\mu\to\infty}& -\frac1{4\mu^2}
    \label{eqn:jeq0}
\end{align}

For $j=1$ and $j=2$ we employ the following Lemma:
\begin{Lemma}[Watson]
    Let $\phi\colon [0,1] \to \mathbb{C}, \phi(t) = t^\sigma g(t)$ such that $g$ is
analytic in some neighbourhood of $t=0$, $\sigma > -1$, $\beta > 0$, and
\begin{equation*}
  \exists C, b > 0\, \forall t > 0\colon \Abs{\phi(t)} < C \Eto{bt}.
\end{equation*}
Then the exponential integral
\begin{align*}
  F(x) := \Integ[T]{0}{t}{\Eto{-\beta\,xt}\phi(t)}.
\end{align*}
is finite for all $x \geq 0$ and has the asymptotic expansion in terms of the
gamma function $\Gamma$ (cf.\ \cref{app:gamma})
\begin{equation*}
  F(x)\SimAs{x\to\infty}\sum_{n=0}^{\infty}
  \frac{g^{(n)}(0)}{(\beta x)^{n+\sigma+1}} \frac{\Gamma(n+\sigma+1)}{n!}
\end{equation*}

    \begin{proof}
        See Appendix~\ref{sec:proof-watson}.
    \end{proof}
    \begin{Remark}
        We actually don't need analyticity but only the existence of as many
        derivatives as degrees we want and additionally a finiteness condition
        on
    \end{Remark}
\end{Lemma}

% j = 1, ++
% TODO: 1/2µ einbauen
The first term for $j=1$ is as follows
\begin{align*}
    \Tr R^{(1)}_{++} &= C(\mu,\theta)^2
        \Integ[1]{0}{x}{
            \Integ[1]{0}{z}{\Eto{-\mu(x+z)}
                \lambda(z)\Eto{-\mu(z+x)}
            }
        }
        \\
        &= C(\mu,\theta)^2
            \Integ[1]{0}{x}{
                \Eto{-2\mu x}
            }
            \Integ[1]{0}{z}{\Eto{-2\mu z}\lambda(z)}
        \\
        &= C(\mu, \theta)^2 \left(
            \frac1{2\mu} - \frac{\Eto{-2\mu}}{2\mu}
           \right)
           \Integ[1]{0}{z}{\Eto{-2\mu z}\lambda(z)}
        \\
        &\SimAs{\mu\to\infty}
            \frac{(-1)^2}{2\mu}
            \sum_{n=0}^{\infty}\frac{\lambda^{(n)}(0)}{(2\mu)^{n+1}}
\end{align*}

% j = 1, +-
The second term for $j=1$ is
\begin{align*}
    \Tr R^{(1)}_{+-} &= -C(\mu,\theta)
        \Integ[1]{0}{x}{
            \Integ[1]{0}{z}{\Eto{-\mu(x+z)}
                \lambda(z)\Eto{-\mu\left|z-x\right|}
            }
        } \\
        &= -C(\mu,\theta)
            \Integ[1]{0}{x}{\left(
                \Integ[x]{0}{z}{
                    \Eto{-\mu(x+z+x-z)} \lambda(z)
                }
                +
                \Integ[1]{x}{z}{
                    \Eto{-\mu(x+z-x+z)} \lambda(z)
                }
                \right)
            } \\
        &= -C(\mu,\theta) \left(
            \Integ[1]{0}{x}{\Eto{-2\mu x} \Lambda(x)}
            + \Integ[1]{0}{x}{\Eto{-2\mu x}\lambda(x)x}
            \right) \\
        &\SimAs{\mu\to\infty}\ 
            \sum_{n=0}^\infty \frac{\Lambda^{(n)}(0)}{(2\mu)^{n+1}}
            + \sum_{n=0}^\infty \frac{n\lambda^{(n-1)}(0)}{(2\mu)^{n+1}} \\
        &= \sum_{n=0}^\infty \frac{(n+1)\lambda^{(n-1)}}{(2\mu)^{n+1}},
\end{align*}
% TODO: Beweis, dass Int_0^1 Int_y^1 f(x) dx dy = Int_0^1 x f(x) dx
where $\Lambda(x) := \Integ[x]{0}{z}{\lambda(z)}$. We also used the identities
\begin{align*}
    &\Integ[1]{0}{x}{\Integ[1]{x}{y}{f(y)}} = \Integ[1]{0}{x}{xf(x)},
    \text{ and }
    &\frac{\mathrm d^n}{\mathrm dx^n}xf(x) = xf^{(n)}(x) + nf^{(n-1)}(x).
\end{align*}

