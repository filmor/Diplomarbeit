\section{Proof of Watson's lemma}
% TODO: Geschichte?
% TODO: Beweis so ausweiten, dass er auch für k continuous derivatives funzt
\label{sec:proof-watson}
\begin{theorem}
    Let $\phi\colon [0,1] \to \mathbb{C}, \phi(t) = t^\sigma g(t)$ such that $g$ is
analytic in some neighbourhood of $t=0$, $\sigma > -1$, $\beta > 0$, and
\begin{equation*}
    \exists_{C, b > 0} \forall_{t > 0} \left|\phi(t)\right| < C \Eto{bt}.
\end{equation*}
Then the exponential integral
\begin{align*}
    F(x) := \Integ[T]{0}{t}{\Eto{-\beta\,xt}\phi(t)}
\end{align*}
is finite for all $x \geq 0$ and has the asymptotic expansion in terms of the
gamma function $\Gamma$ (see Appendix~\ref{app:gamma})
\begin{equation*}
    F(x)\SimAs{x\to\infty}\sum_{n=0}^{\infty}
    \frac{g^{(n)}(0)}{(\beta x)^{n+\sigma+1}} \frac{\Gamma(n+\sigma+1)}{n!}
\end{equation*}

    \begin{proof}
        The proof due to Miller (ref MILLER) uses the Taylor series expansion of
        $g(t)$ with a remainder term in the integrand.

        Since we only assume differentiability in some neighbourhood of $t=0$ we
        need to estimate the integral for some $0 < s < T$, with $s$ being
        sufficiently small such that $g\in C^{\infty}([0,s])$ is analytic. We
        can now split up the integral $F(x)$ as
        \begin{align*}
            F(x) = \Integ[s]{0}{t}{\Eto{-xt}\phi(t)} +
                   \Integ[T]{s}{t}{\Eto{-xt}\phi(t)}.
        \end{align*}

        The integral over $\left[s,T\right]$ can be easily estimated as
        \begin{align*}
            \left|\Integ[T]{s}{t}{\Eto{-xt}\phi(t)}\right| &\le
            \Eto{-xs} \Integ[T]{s}{t}{\left|\phi(t)\right|} \\
            &\le \Eto{-xs}\Integ[T]{0}{t}{\left|\phi(t)\right|}.
        \end{align*}
        Since we assumed this integral to be finite and $s>0$ we have for
        $x\to\infty$
        \begin{align}
            \Integ[T]st{\Eto{-xt}\phi(t)} = O(x^{-\infty}),
        \end{align}
        so we see that this term doesn't contribute to the asymptotic expansion.

        For the first term, the integral over $\left[0,s\right]$ we use the
        taylor expansion of $g$ with remainder
        \begin{align*}
            g(t) = \sum_{n=0}^{N} \frac{g^{(n)}(0)}{n!}t^n + r_N(t).
        \end{align*}

        We thus have
        \begin{align*}
            \Integ[s]0t{\Eto{-xt}t^\sigma g(t)} = \sum_{n=0}^N
            \frac{g^{(n)}(0)}{n!} \Integ[s]0t{\Eto{-xt}t^{\sigma+n}} +
            \Integ[s]0t{\Eto{-xt}t^\sigma r_N(t)},
        \end{align*}
        where we can estimate the remainder term as
        \begin{align*}
            \left|\Integ[s]0t{\Eto{-xt}t^\sigma r_N(t)}\right| &\le
            \Integ[s]0t{\Eto{-xt}t^\sigma\left|r_N(t)\right|} \\
            &\le \sup_{\tau\in\left[0,s\right]}
            \frac{\bigl|g^{(N+1)}(\tau)\bigr|}{(N+1)!}\Integ[s]0t{\Eto{-xt}t^{\sigma+N+1}}.
        \end{align*}

        It thus suffices to estimate the asymptotic behaviour of integrals of
        the type $F_\alpha(x) := \Integ[s]0t{\Eto{-xt}t^\alpha}$ for
        $x\to\infty$. We have
        \begin{align*}
            F_\alpha(x) &= \Integ[\infty]{0}{t}{\Eto{-xt}t^\alpha} -
            \Integ[\infty]{s}{t}{\Eto{-xt}t^\alpha} \\
            &= x^{-(\alpha+1)}\Integ[\infty]{s}{\tau}{\Eto{-\tau}\tau^\alpha} -
            \Integ[\infty]{s}{t}{\Eto{-xt}t^\alpha} \\
            &= x^{-(\alpha+1)}\Gamma(\alpha + 1) - 
            \Integ[\infty]{s}{t}{\Eto{-xt}t^\alpha}.
        \end{align*}
        For the second term we can use the Hölder equation in the special case
        $p=q=2$, where we split the exponential as $\Eto{-xt} =
        \Eto{-xt/2}\Eto{-xt/2}$, to get
        \begin{align*}
            \Integ[\infty]{s}{t}{\Eto{-xt}t^\alpha} &\le
            \sqrt{\Integ[\infty]{s}{t}{\Eto{-xt}}} \cdot
            \sqrt{\Integ[\infty]{s}{t}{\Eto{-xt}t^{2\alpha}}} \\
            &= \Eto{-xs/2} \sqrt{\Integ[\infty]{s}{t}{\Eto{-xt}t^{2\alpha}}}.
        \end{align*}
        The second term in the product decreases with $x$, and since we are
        interested in the case $x\to\infty$ we can assume $x>1$ to estimate the
        square of the second factor as
        \begin{align*}
            \Integ[\infty]{s}{t}{\Eto{-xt}t^{2\alpha}} &\underset{x>1}\le
            \Integ[\infty]{s}{t}{\Eto{-t}t^{2\alpha}} \\
            &=: C_{s,\alpha}
        \end{align*}
        and thus since $s>0$ we have the following asymptotic behaviour for
        $x\to\infty$:
        \begin{align*}
            \Integ[\infty]{s}{t}{\Eto{-xt}t^\alpha} &\le \Eto{-xs/2}
            \sqrt{\Integ[\infty]{s}{t}{\Eto{-t}t^{2\alpha}}} \\
            &\le C_{s,\alpha}\,\Eto{-xs/2} = O(x^{-\infty}),
        \end{align*}
        thus, again, this term doesn't contribute to the asymptotic expansion.

        Using this we have
        \begin{align*}
            F_\alpha(x) = x^{-(\alpha+1)}\,\Gamma(\alpha + 1) + O(x^{-\infty}),
        \end{align*}
        and inserting $\alpha = \sigma + N + 1$ for the remainder and $\alpha =
        \sigma + n$ for the main term we arrive at
        \begin{align*}
            F(x) &= \sum_{n=0}^N \frac{g^{(n)}(0)}{n!} F_{\sigma+n} + F_{\sigma +
            N + 1} \\
            &= \sum_{n=0}^N \frac{g^{(n)}(0)}{n!}
                 \Gamma(\sigma + n +1)\,x^{-(\sigma + n + 1)} + \Gamma(\sigma +
                 N + 2)\,x^{-(\sigma + N + 2)} \\
            &= \sum_{n=0}^N \frac{g^{(n)}(0)}{n!}\Gamma(\sigma + n +
            1)\,x^{-(\sigma + n + 1)} + O\bigl(x^{-(\sigma + N + 2)}\bigr).
        \end{align*}
        This proves the statement.
    \end{proof}
    \begin{remark}
        There is also a very simple proof using the so called Moment Expansion
        by Estrada (ref).
    \end{remark}
\end{theorem}
