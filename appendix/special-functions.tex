\section{Special Functions}
\subsection{Gamma Function}
\label{app:gamma}
The gamma function is defined on $\mathbb{R}_{+}$ as the exponential integral
\begin{align}
    \Gamma(x) := \Integ[\infty]{0}{t}{\Eto{-t}t^{x-1}},
    \label{def:gamma}
\end{align}
which coincides with $(x-1)!$ for $x\in\mathbb{N}$. In the text we use the so
called (upper) incomplete gamma function, defined by the integral
\begin{align}
  \Gamma(x,b) := \Integ[\infty]{b}{t}{\Eto{-t}t^{x-1}},
\end{align}
for which we will now prove the following lemma:
\begin{Lemma}
  \label{lem:incomplete-gamma}
  Let $n\in\mathbb{N}_0$, $k\in\mathbb{R}$. Then the following formula holds:
  \begin{align}
    \Gamma(n+1, b) = \Integ[b]{\infty}{t}{\Eto{-t} t^n} = n!\ \Eto{-b}\sum_{m=0}^n
    \frac{b^m}{m!}
  \end{align}
  \begin{Proof}
    We will prove this by induction over $n$. For $n = 0$ we are left with
    \begin{align*}
      \Gamma(1, b) = \Integ[b]{\infty}{t}{\Eto{-t}} = \Eto{-b}.
    \end{align*}
    For $n$ we partially integrate
    \begin{align*}
      \Gamma(n+1, b) &= \Integ[b]{\infty}{t}{\Eto{-t} t^n} 
                     = \left.-\Eto{-t} t^n \right|_{b}^{\infty} +
      n\Integ[b]{\infty}{t}{\Eto{-t}t^{n-1}}
      = \Eto{-b} b^n + n\Gamma(n, b) \\
      &= \Eto{-b} b^n + n (n-1)!\ \Eto{-b}\sum_{m=0}^{n-1} \frac{b^m}{m!} \\
      &= n!\ \Eto{-b} \sum_{m=0}^n \frac{b^m}{m!},
    \end{align*}
    where we inserted the induction requirement in the last step.
  \end{Proof}
\end{Lemma}

\subsection{Digamma Function}
The digamma function is the logarithmic derivative of the gamma function
\begin{align}
    \psi(x) := \frac{\mathrm d}{\mathrm dx}\log{\Gamma(x)}
            = \frac{\Gamma'(x)}{\Gamma(x)},
    \label{def:digamma}
\end{align}
which is sometimes also denoted by the ancient greek letter $\digamma$, which is
actually called ``digamma''.

We will need the following property of the digamma function:
\begin{align}
  \psi(x + 1) = \psi(x) + \frac1x,
\end{align}
which is easily proven using the recursion rule of the gamma function:
\begin{align*}
  \psi(x+1) &= \frac{\mathrm d}{\mathrm dx}\log{\Gamma(x+1)} = \frac{\mathrm
d}{\mathrm dx}\log{x\Gamma(x)} \\
  &= \frac{\mathrm d}{\mathrm dx}\log{x} + \frac{\mathrm d}{\mathrm
  dx}\log{\Gamma(x)} \\
  &= \frac1x + \psi(x).
\end{align*}
Furthermore we need the special value at $x = \frac12$, $\psi(\frac12) =
-\gamma-\log 2$, which can be found using the Gauss Digamma Theorem REF.

\subsection{Beta Function}
\label{sec:beta_function}
The beta function is defined for $x,y\in\mathbb{C}$ with $\Re(x),\Re(y) > 0$ as
\begin{align}
    B(x,y) = \Integ[1]{0}{t}{t^{x-1}(1-t)^{y-1}}.
    \label{def:beta}
\end{align}
It is also sometimes called \textit{Euler integral of the first kind}. In this
thesis we use two specific representations of the beta function, namely
\begin{align}
    B(x, y) &= \frac{\Gamma(x)\cdot\Gamma(y)}{\Gamma(x+y)} \quad \text{and}
    \label{frm:beta_rep1} \\
    B(x, y) &= \Integ[\infty]{0}{t}{\frac{t^{x-1}}{(1 + t)^{x+y}}}
    \label{frm:beta_rep2}
\end{align}

We can derive \cref{frm:beta_rep2} easily by substituting $t \mapsto
\frac{t'}{1+t'}$. \Cref{frm:beta_rep1} is a bit more subtle:
\begin{align*}
    \Gamma(x) \Gamma(y) &= \Integ[\infty]{0}{t_1}{\Eto{-t_1}t_1^{x-1}}
                           \Integ[\infty]{0}{t_2}{\Eto{-t_2}t_2^{y-1}} \\
    &= \Integ[\infty]{0}{t_1}{
        \Integ[\infty]{0}{t_2}{
            \Eto{-t_1-t_2}t_1^{x-1}t_2^{y-1}
            }
        },\text{ substitute $t_1 = zt$ and $t_2 = z(1-t)$} \\
    &= \Integ[\infty]{0}{z}{
        \Integ[1]{0}{t}{
            \Eto{-z}(zt)^{x-1}(z(1-t))^{y-1}z
            }
        } \\
        &= \Integ[\infty]{0}{z}{ \Eto{-z}z^{x+y-1} }
            \Integ[1]{0}{t}{t^{x-1}(1-t)^{y-1}} \\
    \Leftrightarrow \Gamma(x)\Gamma(y) &= \Gamma(x+y)B(x,y)
\end{align*}

From the first representation we can also calculate the derivative of the beta
function:
\begin{align}
    \frac{\partial}{\partial x} B(x, y)
    &= \frac{
                \Gamma'(x)\Gamma(y)\Gamma(x+y) - \Gamma(x)\Gamma(y)\Gamma'(x+y) 
            }{
                \Gamma(x+y)^2
            } \nonumber \\
    &= \frac{
                (\Gamma(x)\psi(x))\Gamma(y)\Gamma(x+y) -
                \Gamma(x)\Gamma(y)(\Gamma(x+y)\psi(x+y))
            }{
                \Gamma(x+y)^2
            } \nonumber \\
    &= \frac{\Gamma(x)\cdot\Gamma(y)}{\Gamma(x+y)}\left(\psi(x) - \psi(x+y)\right)
        \nonumber \\
    &= B(x,y)\left(\psi(x) - \psi(x+y)\right)
    \label{frm:beta_deriv}
\end{align}
